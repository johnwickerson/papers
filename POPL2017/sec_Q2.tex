\section{Application: Comparing MCMs (\Q2)}
\label{sec:discrep}

In this section, we use Alloy to generate litmus tests that
distinguish three proposed variants of the C11 MCM
(\S\ref{sec:Q2_c11_sra_simp}, \S\ref{sec:kyndylan},
\S\ref{sec:Q2_c11_simp_orig}). Such distinguishing tests, particularly
the \emph{simplest} distinguishing tests, are difficult to find by
hand, but are very useful for illustrating the proposed changes. We go
on to check two generic properties of MCMs -- multi-copy atomicity
(\S\ref{sec:Q2_mca}) and SC-DRF (\S\ref{sec:scdrf}) -- by encoding the
properties as MCMs themselves and comparing them against software- and
architecture-level MCMs.

\subsection{Strong Release/Acquire Semantics in C11}
\label{sec:Q2_c11_sra_simp}
Lahav et al. have proposed a stronger semantics for release/acquire
atomics~\cite{lahav+16}. For the release/acquire fragment of C11 (no
non-atomics and no relaxed or SC atomics), their semantics amounts to
adding the axiom $\acyclic(sb \cup co \cup rf)$. We used Alloy to
compare their MCM, which we call \mm{C11-SRA}, to \mm{C11} over the
release/acquire fragment.

% Alloy
% demonstrated that the classic `2+2W' litmus test~\cite{?}
% \begin{center}
% \begin{tikzpicture}[inner sep=1pt, baseline=(a.base)]
% \node (a) at (0,0.8) {$\evW{"rel"}{"x"}{1}$};
% \node (b) at (0,0) {$\evW{"rel"}{"y"}{2}$};
% \node (c) at (2.5,0.8) {$\evW{"rel"}{"y"}{1}$};
% \node (d) at (2.5,0) {$\evW{"rel"}{"x"}{2}$};
% \draw[edgemo] (b) to [auto, pos=0.8] node {$co$} (c);
% \draw[edgemo] (d) to [auto, swap, pos=0.8] node {$co$} (a);
% \draw[edgesb] (a) to [auto] node {$sb$} (b);
% \draw[edgesb] (c) to [auto] node {$sb$} (d);
% \node[sthd, fit=(a)(b)] {};
% \node[sthd, fit=(c)(d)] {};
% \node at (5.0,0.4) {$\litmustestIsmall{atomic\_int x=0,y=0;}{
% \quad\texttt{x.store(1,REL);} \\
% \quad\texttt{y.store(2,REL);} \\[-1.6mm]
% \quad\rule{22mm}{0.4pt} \\[-2.7mm]
% \quad\rule{22mm}{0.4pt} \\
% \quad\texttt{y.store(1,REL);} \\
% \quad\texttt{x.store(2,REL);} \\
% }{x==1 \&\& y==1}$};
% \end{tikzpicture}
% \end{center}
% %
% can pass under \mm{C11-Simp} but not under \mm{C11-SRA} -- just as
% Lahav et al. report. We also performed the comparison under the
% additional constraint that the postcondition should not need to refer
% to shared locations, only to registers -- this corresponds to a
% technical requirement in Lahav et al.'s work. The example that Lahav
% et al. provide in this context is a 10-event
% execution~\cite[Fig.~5]{lahav+16}; under the same constraints, Alloy
% found a distinguishing execution with just six events
Lahav et al. provide a 10-event (and 4-location) execution that
distinguishes the MCMs~\cite[Fig.~5]{lahav+16}. Alloy, on the other
hand, found a execution that requires just 6 events (and 2 locations)
and serves the same purpose.\footnote{Lahav et al. impose an additional technical requirement
that postconditions should not need to refer to shared locations (only
to registers), which rules out the even-simpler `2+2W' litmus
test~\cite{sarkar+11}.}
%
\begin{center}
\begin{tikzpicture}[inner sep=1pt, baseline=(a.base)]
\node (a) at (0,0.8) {$\evW{\morel}{"x"}{1}$};
\node (b) at (0,0) {$\evW{\morel}{"y"}{2}$};
\node (c) at (0,-0.8) {$\evR{\moacq}{"y"}{1}$};
\node (d) at (2.5,0.8) {$\evW{\morel}{"y"}{1}$};
\node (e) at (2.5,0) {$\evW{\morel}{"x"}{2}$};
\node (f) at (2.5,-0.8) {$\evR{\moacq}{"x"}{1}$};
\draw[edgemo] (b) to [auto, pos=0.8] node {$co$} (d);
\draw[edgemo] (e) to [auto, swap, pos=0.8] node {$co$} (a);
\draw[edgerf] (a) to [auto, swap, pos=0.7] node {$rf$} (f);
\draw[edgerf] (d) to [auto, pos=0.7] node {$rf$} (c);
\draw[edgesb] (a) to [auto] node {$sb$} (b);
\draw[edgesb] (b) to [auto] node {$sb$} (c);
\draw[edgesb] (d) to [auto] node {$sb$} (e);
\draw[edgesb] (e) to [auto] node {$sb$} (f);
\node[sthd, fit=(a)(b)(c)] {};
\node[sthd, fit=(d)(e)(f)] {};

\node at (5.0,-0.0) {$\litmustestIsmall{atomic\_int x=0,y=0;}{
\quad\texttt{x.store(1,\morel);} \\
\quad\texttt{y.store(2,\morel);} \\
\quad\texttt{r0=y.load(\moacq);} \\[-1.6mm]
\quad\rule{22mm}{0.4pt} \\[-2.7mm]
\quad\rule{22mm}{0.4pt} \\
\quad\texttt{y.store(1,\morel);} \\
\quad\texttt{x.store(2,\morel);} \\
\quad\texttt{r1=x.load(\moacq);} \\
}{r0==1 \&\& r1==1}$};
\end{tikzpicture}
\end{center}

\subsection{Forbidding Reading/Synchronisation Cycles in C11}
\label{sec:kyndylan}

Nienhuis et al.~\cite[Fig.~12]{nienhuis+16} have suggested
strengthening the C11 MCM with the axiom $\acyclic(sw\cup rf)$. Let us
call that MCM \mm{C11-SwRf}. We used Alloy to search for litmus tests
that could witness such a change, and discovered a solution requiring
12 events and 6 threads
%
\begin{equation*}
\small
\begin{tikzpicture}[inner sep=0.5mm, baseline=(a.base)]
\node (a) at (0.0,1.6) {$\evRMW{\morlx}{"y"}{4}{5}$};
\node (b) at (0.0,0.8) {$\evF{\morel}$};
\node (c) at (0.0,0.0) {$\evW{\morlx}{"x"}{1}$};
\node (d) at (2.1,1.6) {$\evRMW{\morel}{"y"}{2}{3}$};
\node (e) at (2.1,0.8) {$\evW{\morlx}{"y"}{4}$};
\node (f) at (2.1,0.0) {$\evRMW{\moacq}{"x"}{1}{2}$};
\node (g) at (4.2,1.6) {$\evRMW{\morel}{"x"}{2}{3}$};
\node (h) at (4.2,0.8) {$\evW{\mosc}{"x"}{4}$};
\node (i) at (4.2,0.0) {$\evRMW{\mosc}{"y"}{1}{2}$};
\node (j) at (6.3,1.6) {$\evRMW{\morel}{"x"}{4}{5}$};
\node (k) at (6.3,0.8) {$\evF{\moar}$};
\node (l) at (6.3,0.0) {$\evW{\morlx}{"y"}{1}$};
\draw[edgesb] (a) to [auto,pos=0.4] node {$sb$} (b);
\draw[edgesb] (b) to [auto,pos=0.4] node {$sb$} (c);
\draw[edgesb, shift left=0.5mm] (d) to [auto,pos=0.4] node {$sb$} (e);
\draw[edgesb, shift right=0.5mm] (g) to [auto,swap,pos=0.4] node {$sb$} (h);
\draw[edgesb] (j) to [auto,pos=0.4] node {$sb$} (k);
\draw[edgesb] (k) to [auto,pos=0.4] node {$sb$} (l);
\draw[edgemo, shift right=0.5mm] (d) to [auto,swap] node {$co$} (e);
\draw[edgemo, shift left=0.5mm] (g) to [auto] node {$co$} (h);
\draw[edgemo,shift left=0.5mm] (e) to [auto] node {$co$} (a);
\draw[edgerf,shift right=0.5mm] (e) to [auto,swap] node {$rf$} (a);
\draw[edgemo,shift right=0.5mm] (c) to [auto,swap, inner sep=1mm] node {$co$} (f);
\draw[edgerf,shift left=0.5mm] (c) to [auto] node {$rf$} (f);
\draw[edgemo,shift right=0.5mm] (f) to [auto,swap,pos=0.2] node {$co$} ([xshift=-5mm]g.south);
\draw[edgerf,shift left=0.5mm] (f) to [auto,pos=0.1] node {$rf$} ([xshift=-5mm]g.south);
\draw[edgemo,shift right=0.5mm] (h) to [auto,swap] node {$co$} (j);
\draw[edgerf,shift left=0.5mm] (h) to [auto] node {$rf$} (j);
\draw[edgemo,shift left=0.5mm] (l) to [auto, inner sep=1mm] node {$co$} (i);
\draw[edgerf,shift right=0.5mm] (l) to [auto,swap] node {$rf$} (i);
\draw[edgemo,shift left=0.5mm] (i) to [auto,pos=0.2] node {$co$} ([xshift=5mm]d.south);
\draw[edgerf,shift right=0.5mm] (i) to [auto,swap,pos=0.1] node {$rf$} ([xshift=5mm]d.south);
\node[sthd_qtight, fit=(a)(b)(c)] {};
\node[sthd_qtight, fit=(d)(e)] {};
\node[sthd_qtight, fit=(f)] {};
\node[sthd_qtight, fit=(g)(h)] {};
\node[sthd_qtight, fit=(i)] {};
\node[sthd_qtight, fit=(j)(k)(l)] {};
\end{tikzpicture}
\end{equation*}
%
that is virtually identical to the one provided by Nienhuis et al., if
a little less symmetrical. We sought smaller solutions, and found none
with fewer than 8 events that could distinguish \mm{C11-SwRf} from
\mm{C11}. For executions with 8 to 11 events, the SAT solver could not
return a result in a reasonable time.

\subsection{Simplifying the SC Axioms in C11}
\label{sec:Q2_c11_simp_orig}
Batty et al. have proposed a change to the C11 consistency axioms that
enables them to be simplified, and also avoids the need for a total
order, $S$, over all SC events~\cite{batty+16}. Having already
incorporated their proposal in our Fig.~\ref{fig:c11_predicates}, let
us call the original MCM \mm{C11-Orig}.

Batty et al. present a litmus test to distinguish \mm{C11}
from \mm{C11-Orig}, which uses 7 instructions across 4
threads~\cite[Example~1]{batty+16}. Alloy, on the other hand, found
one (below) that needs only 5 instructions and 3 threads.
%
\begin{equation*}
\begin{tikzpicture}[inner sep=1pt]
\node[event, anchor=west](a) at (0.2,2) 
{$\evW{\morlx}{"x"}{1}$};

\node[event, anchor=west](b) at (2,2) 
{$\evRMW{\mosc}{"x"}{1}{2}$};

\node[event, anchor=west](c) at (2,1.2) 
{$\evR{\mosc}{"y"}{0}$};

\node[event, anchor=west](d) at (0.2,1.2) 
{$\evW{\mosc}{"y"}{1}$};

\node[event, anchor=west](e) at (0.2,0.4) 
{$\evR{\mosc}{"x"}{1}$};

\foreach \i/\j in {b/c, d/e}
\draw[edgeS] ([xshift=5mm]\i.south west) to[auto,pos=0.4]
node{$S$} ([xshift=5mm]\j.north west -| \i.south west);

\foreach \i/\j in {b/c, d/e}
\draw[edgesb] ([xshift=4mm]\i.south west) to[auto,swap,pos=0.4]
node{$sb$} ([xshift=4mm]\j.north west -| \i.south west);

\draw[edgeS] (c) to[auto,pos=0.5] node{$S$} (d);

\draw[edgemo] ([yshift=.5mm]a.east) to[auto] node{$co$} ([yshift=.5mm]b.west);

\draw[edgerf, overlay] (a.south west) to[auto, bend right=20, pos=0.05] node{$rf$}
(e.north west);

\draw[edgerf] ([yshift=-.5mm]a.east) to[auto, swap] node{$rf$}
([yshift=-.5mm]b.west);

\node[sthd, fit=(a)] {};
\node[sthd, fit=(b)(c)] {};
\node[sthd, fit=(d)(e)] {};

\node at (6.4,1.3) {$\litmustestIsmall{atomic\_int x=0,y=0;}{
\quad\texttt{x.store(1,\morlx);} \\[-1.6mm]
\quad\rule{30.5mm}{0.4pt} \\[-2.7mm]
\quad\rule{30.5mm}{0.4pt} \\
\quad\texttt{r0=x.cas(1,2,\mosc,\morlx);} \\
\quad\texttt{r1=y.load(\mosc);} \\[-1.6mm]
\quad\rule{30.5mm}{0.4pt} \\[-2.7mm]
\quad\rule{30.5mm}{0.4pt} \\
\quad\texttt{y.store(1,\mosc);} \\
\quad\texttt{r2=x.load(\mosc);} \\
}{r0==true\,\&\&\,r1==0\,\&\&\,r2==1}$};
\end{tikzpicture}
\end{equation*}
%
%
% \begin{equation*}
% \litmustestIIIsmall{atomic\_int x,y=0;}{
%    \texttt{x.store(1,RLX);} 
%  & \texttt{r0=x.cas(1,2,SC,RLX);} 
%  & \texttt{y.store(1,SC);} \\
%  & \texttt{r1=y.load(SC);} 
%  & \texttt{r2=x.load(SC);} 
% \\
% }{r0==true \&\& r1==0 \&\& r2==1}
% \end{equation*}
%

\subsection{Multi-copy atomicity}
\label{sec:Q2_mca}
% Cite peter?
The property of \emph{multi-copy atomicity}\footnote{This is also
known as \emph{store atomicity}~\cite{sorin+11} or \emph{write
atomicity}~\cite{adve+96}.} (MCA)~\cite{collier92} ensures that, in
the absence of thread-local reordering, different threads cannot
observe writes in conflicting orders -- i.e., there is a single copy
of the memory that serialises all writes. The canonical MCA violation
is given by the IRIW test~\cite{boehm+08} (below) where thread-local
reordering has been disabled by inducing `preserved program order'
($ppo$) edges, perhaps using dependencies or fences. That the final
reads observe $0$ betrays a disagreement in the order the writes
occurred:
%
\begin{equation}
\label{eq:iriw}
\begin{tikzpicture}[inner sep=0.5mm, baseline=(a.base)]
\node (a) at (0.0,1.6) {$\evW{}{"y"}{1}$};
\node (b) at (2.1,1.6) {$\evR{}{"y"}{1}$};
\node (c) at (2.1,0.8) {$\evR{}{"x"}{0}$};
\node (d) at (4.2,1.6) {$\evR{}{"x"}{1}$};
\node (e) at (4.2,0.8) {$\evR{}{"y"}{0}$};
\node (f) at (6.3,1.6) {$\evW{}{"x"}{1}$};
\draw[edgesb] (b) to [auto,pos=0.4] node {$ppo$} (c);
\draw[edgesb] (d) to [auto,pos=0.4] node {$ppo$} (e);
%\draw[edgesb, shift right=0.5mm] (g) to [auto,swap,pos=0.4] node {$sb$} (h);
\draw[edgerf,shift right=0.5mm] (a) to [auto,swap] node {$rf$} (b);
\draw[edgerf,shift left=0.5mm] (f) to [auto] node {$rf$} (d);
\node[sthd, fit=(a)] {};
\node[sthd, fit=(b)(c)] {};
\node[sthd, fit=(d)(e)] {};
\node[sthd, fit=(f)] {};
\end{tikzpicture}
\end{equation}

\begin{figure}[t]
\begin{myFrame}{$consistent_{\rm MCA}(ppo)$}
\begin{mathpar}
\labelb{eq:mca:uniproc}~
\acyclic((sb \cap \sloc) \cup co)
\and
\labelb{eq:mca:wo}~ 
\acyclic(wo(ppo))
\end{mathpar}
\end{myFrame}
\begin{mathpar}
wo(ppo) \eqdef 
    (((rfe \join ppo \join rfe^{-1}) - \id ) \join co) \cup (rfe \join ppo \join fr_{\rm init})
\end{mathpar}
\caption{Multi-copy atomicity as an MCM}
\label{fig:mca_predicates}
\end{figure}

We formalise MCA -- for the first time in the axiomatic style -- in
Fig.~\ref{fig:mca_predicates}. The model comprises write/write
coherence~\refb{eq:mca:uniproc}~\cite{sorin+11} and the acyclicity of
the \emph{write order} relation, $wo$~\refb{eq:mca:wo}. Write order
captures the intuition that if two reads, say $r_1$ and $r_2$ (with
$(r_1,r_2)\in ppo$), observe two writes, say $w_1$ and $w_2$
respectively, then any write $co$-later than $w_2$ must follow $w_1$
in the single copy of memory. The $wo$ relation is the union of two
cases: in the first, both reads observe write events, and in the
second, one reads from the initial value (reusing our definition of
$fr_{\rm init}$ from Def.~\ref{def:fromread}). Note that MCA is
parameterised by the given model's $ppo$.\footnote{We have an
alternative formulation for happens-before-based models.}

With this formal definition of MCA, we can seek executions allowed by
a given MCM but disallowed by MCA. We tested x86 and Power, and as
expected, Power does not guarantee MCA (Alloy finds a counterexample
similar to \eqref{eq:iriw}) but x86 does.


\subsection{Seeking SC-DRF Violations}
\label{sec:scdrf}

We used Alloy to seek violations of the SC-DRF
guarantee~\cite{adve+90} in an early draft of the C11
MCM~\cite{c++11draft}. The SC-DRF guarantee, in the C11 context, states
that if a program is race-free and contains no non-SC atomic operations,
then it has only SC semantics.

\begin{figure}[t]
\begin{myFrame}{$consistent_{\mm{SC}}$}
\begin{mathpar}
\labelb{eq:sc}~
acyclic(rf \cup co \cup fr \cup sb)
\end{mathpar}
\end{myFrame}
\caption{The \mm{SC} MCM (following Shasha et al.~\cite{shasha+88})}
\label{fig:sc}
\end{figure}

We sought an execution $X$ that is dead (so that its corresponding
litmus test is race-free), uses no non-SC atomics, and is consistent
under the (draft) C11 MCM but inconsistent under \mm{SC}:
\[
\field{X}{A} \subseteq \field{X}{sc}
\wedge X\in dead_{\mm{C11}}
\cap consistent_{\mm{C11-Draft}} - consistent_{\mm{SC}}
\]
where the \mm{SC} MCM is characterised in Fig.~\ref{fig:sc} and
\mm{C11-Draft} has all the axioms listed by Batty et
al.~\cite[Def.~11]{batty+16}, minus their `S4' axiom. Alloy found an
example similar to that reported by Batty et al.~\cite[\S4,
\emph{Sequential consistency for SC atomics}]{batty+11}. The non-SC
execution is consistent under \mm{C11-Draft} because no rule forbids
SC reads to observe initial writes. The `S4' axiom was
added~\cite[\S2.7]{batty+11} to fix exactly this issue.


%%% Local Variables:
%%% mode: latex
%%% TeX-master: "paper"
%%% End: