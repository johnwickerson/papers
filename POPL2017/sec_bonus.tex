\appendix
\section{Bonus stuff, probably won't fit in}

\subsection{Monotonicity violation in C11}
Alloy finds the following pair of 6-event executions that witness a
monotonicity violation in C11:
\begin{equation*}
\begin{tikzpicture}[inner sep=1pt]
\node[event, anchor=west](a) at (1,2.5) 
{$\evR{\morlx}{"x"}{1}$};

\node[event, anchor=west](b) at (1,1.5) 
{$\evR{"na"}{"a"}{1}$};

\node[event, anchor=west](c) at (1,0.5) 
{$\evW{\mosc}{"y"}{1}$};

\node[event, anchor=west](d) at (2.7,2.5) 
{$\evR{\morlx}{"y"}{1}$};

\node[event, anchor=west](e) at (2.7,1.5) 
{$\evW{\morlx}{"x"}{1}$};

\node[event, anchor=west](f) at (-0.5,2) 
{$\evW{"na"}{"a"}{1}$};

\node[event, anchor=west](a') at (5,2) 
{$\evR{\morlx}{"x"}{1}$};

\node[event, anchor=west](b') at (5,1) 
{$\evR{"na"}{"a"}{1}$};

\node[event, anchor=west](c') at (5,0) 
{$\evW{\mosc}{"y"}{1}$};

\node[event, anchor=west](d') at (6.5,2) 
{$\evR{\morlx}{"y"}{1}$};

\node[event, anchor=west](e') at (6.5,1) 
{$\evW{\morlx}{"x"}{1}$};

\node[event, anchor=west](f') at (5,3) 
{$\evW{"na"}{"a"}{1}$};

\foreach \i/\j in {a/b, b/c, d/e, a'/b', b'/c', d'/e'} {
\draw[edgesb] ([xshift=5mm]\i.south west) to[auto,swap,pos=0.4]
node{$sb$} ([xshift=5mm]\j.north west -| \i.south west);
\draw[edgesb] ([xshift=6mm]\i.south west) to[auto,pos=0.4]
node{$cd$} ([xshift=6mm]\j.north west -| \i.south west);
}
\draw[edgesb] ([xshift=5mm]f'.south west) to[auto,swap,pos=0.4]
node{$sb$} ([xshift=5mm]a'.north west -| a'.south west);

\draw[edgerf] (e) to[auto, swap, pos=0.7] node{$rf$} (a);
\draw[edgerf] (c) to[auto, pos=0.3, swap, out=30, in=210] node{$rf$} (d);
\draw[edgerf] (f) to[auto, pos=0.4] node{$rf$} (b);

\draw[edgerf] (e') to[auto, swap, pos=0.7] node{$rf$} (a');
\draw[edgerf] (c') to[auto, swap, pos=0.3, out=30, in=210] node{$rf$} (d');
\draw[edgerf] (f') to[auto, swap, pos=0.2, bend right=70] node{$rf$} (b');

\draw[edgepi] (a) to[auto,swap,pos=0.8] node{$\pi$} (a');
\draw[edgepi] (b) to[auto,swap,pos=0.8] node{$\pi$} (b');
\draw[edgepi] (c) to[auto] node{$\pi$} (c');
\draw[edgepi] (d) to[auto, pos=0.15] node{$\pi$} (d');
\draw[edgepi] (e) to[auto, pos=0.1] node{$\pi$} (e');
\draw[edgepi] (f) to[auto, pos=0.7,out=45,in=180] node{$\pi$} (f');

\node[sthd, fit=(f)] {};
\node[sthd, fit=(a)(b)(c)] {};
\node[sthd, fit=(d)(e)] {};
\node[sthd, fit=(f')(a')(b')(c')] {};
\node[sthd, fit=(d')(e')] {};
\end{tikzpicture}
\end{equation*}
where left-hand execution is inconsistent (because the write to "a"
does not happen-before the read that observes its value) but the
right-hand execution (obtained by sequentialising two of the threads)
is consistent (since "a"'s write now happens-before its read). This is
essentially the same execution as that discovered by Vafeiadis et
al.~\cite[Fig.~1]{vafeiadis+15}; it is just a little less elegant:
Alloy chose the write to "y" to be sequentially-consistent when a
relaxed write would suffice.

\subsection{Generating monotonic programs} We rely on the following
property of our program-generating function:
\[
\stack{
\text{if}~"strengthen"(X, Y, rf, co) \\
\text{then}~ltest(X,rf,co) \sqsubseteq ltest(Y,rf,co)
}
\]
The $\sqsubseteq$ relation is defined as the smallest partial order
that is compatible with the syntax of litmus tests and satisfies:
\begin{mathpar}
\inferrule{ }{\mathbf{st}(x,v,\mathbf{rel}) \sqsubseteq \mathbf{st}(x,v,\mathbf{rlx})}
\and
\inferrule{ }{\mathbf{st}(x,v,\mathbf{sc}) \sqsubseteq \mathbf{st}(x,v,\mathbf{rel})}
\and
\inferrule{ }{\mathbf{ld}(r,x,\mathbf{sc}) \sqsubseteq \mathbf{ld}(r,x,\mathbf{acq})}
\and
\inferrule{ }{\mathbf{ld}(r,x,\mathbf{acq}) \sqsubseteq \mathbf{ld}(r,x,\mathbf{rlx})}
\and
\inferrule{ }{\mathbf{xc}(r,x,v,\mathbf{sc}) \sqsubseteq \mathbf{xc}(r,x,v,\mathbf{ar})}
\and
\inferrule{ }{\mathbf{xc}(r,x,v,\mathbf{ar}) \sqsubseteq \mathbf{xc}(r,x,v,\mathbf{acq})}
\and
\inferrule{ }{\mathbf{xc}(r,x,v,\mathbf{ar}) \sqsubseteq \mathbf{xc}(r,x,v,\mathbf{rel})}
\and
\inferrule{ }{\mathbf{xc}(r,x,v,\mathbf{acq}) \sqsubseteq \mathbf{xc}(r,x,v,\mathbf{rlx})}
\and
\inferrule{ }{\mathbf{xc}(r,x,v,\mathbf{rel}) \sqsubseteq \mathbf{xc}(r,x,v,\mathbf{rlx})}
\and
\inferrule{ }{C_1 ; C_2 \sqsubseteq C_1 + C_2}
\and
\inferrule{ }{C_1 ; C_2 \sqsubseteq C_1 \parallel C_2}
\end{mathpar}

\subsection{Compiling C11 to x86}
A mapping $\pi$ implements the C11-to-x86 compilation scheme if it
satisfies the following constraints:

\paragraph{\upshape $"mapping"(\pi,X,Y)$:}
\begin{mathpar}
\labelb{eq:c11x86:pibij}~
(\pi^{-1}\join\pi) = \stor{Y."ev"}
\and
\labelb{eq:c11x86:sb}~
(X."sb" \join \pi) = (\pi \join Y."sb")
\and
\labelb{eq:c11x86:sthd}~
(X.\sthd \join \pi) = (\pi\join Y.\sthd)
\and
\labelb{eq:c11x86:sloc}~
(X.\sloc \join \pi) = (\pi\join Y.\sloc)
\and
\labelb{eq:c11x86:cd}~
(X."cd" \join \pi) = (\pi\join Y."cd")
\and
\labelb{eq:c11x86:ad}~
(X."ad" \join \pi) = (\pi\join Y."ad")
\and
\labelb{eq:c11x86:dd}~
(X."dd" \join \pi) = (\pi\join Y."dd")
\and
\labelb{eq:c11x86:r}~
\pi (X."R") = Y."R"
\and 
\labelb{eq:c11x86:w}~
\pi(X."W") = Y."W"
\and
\labelb{eq:c11x86:a}~
\pi(X."sc") = Y."A"
\and 
\labelb{eq:c11x86:scf}~
\pi(X."F" \cap X."sc") = Y."F" 
\and
\labelb{eq:c11x86:notscf}~
\pi(X."F" - X."sc") \ncap Y."F" 
\end{mathpar}

This mapping is one-to-one~\refb{eq:c11x86:pibij}. It preserves
sequenced-before~\refb{eq:c11x86:sb}, threads~\refb{eq:c11x86:sthd},
locations~\refb{eq:c11x86:sloc}, and
dependencies~\refb{eq:c11x86:cd}~\refb{eq:c11x86:ad}~\refb{eq:c11x86:dd}.
Reads are mapped to reads~\refb{eq:c11x86:r} and writes are mapped to
writes~\refb{eq:c11x86:w}. SC events are mapped to atomic
events~\refb{eq:c11x86:a}. SC fences are mapped to
fences~\refb{eq:c11x86:scf}, and non-SC fences are mapped to
no-ops~\refb{eq:c11x86:notscf}.


\subsection{Generating executable code}

\begin{figure}
\[
\renewcommand\arraystretch{1.2}
\begin{array}{@{}r@{~~}c@{~~}l@{}}
\cof{x} &=& \C{\&} x \\
\cof{v} &=& v \\
\cof{e + 0 \times r} &=& \C{(} \cof{e} \C{+0*} r \C{)} \\
\cof{\mathbf{st}(e_x, e_v, \mathrm{None})} &=& \C{*} \cof{e_x} \C{=} \cof{e_v} 
\\
\cof{\mathbf{st}(e_x, e_v, \mathrm{Some}~mo)} &=& \\
\multicolumn{3}{@{}r@{}}{
\C{(atomic\_store((atomic\_int*)} \cof{e_x} \C{,}
\cof{e_v} \C{,} mo \C{),0)}} 
\\
\cof{\mathbf{ld}(r,e_x,\mathrm{None})} &=& r \C{=((int*)}\cof{e_x} \C{)} \\
\cof{\mathbf{ld}(r,e_x,\mathrm{Some}~mo)} &=& r \C{=atomic\_load(}
\cof{e_x} \C{,} mo \C{)} 
\\
\cof{\mathbf{cas}(r,e_x,v,e_{v'},\mathrm{Some}~mo)} &=& \\
\multicolumn{3}{@{}r@{}}{\stack{r \C{=(atomic\_compare\_exchange\_strong(} \\\quad \C{(atomic\_int*)}
\cof{e_x} \C{,} v \C{,} \cof{e_{v'}} \C{,} mo \C{,relaxed))?} v \C{:0}}} 
\\
\cof{\mathbf{fence}(\mathrm{Some}~mo)} &=& \C{atomic\_thread\_fence(} mo \C{)}
\\
\cof{\mathbf{if}~r = v~\mathbf{then}~a} &=& r\C{==}v\C{?}\cof{a}\C{:0}
\\
\cof{C_1 + C_2} &=& \cof{C_1}\C{+}\cof{C_2} 
\\
\cof{C_1 \join C_2} &=& \C{(}\cof{C_1}\C{,}\cof{C_2}\C{)}
\end{array}
\]
\caption{Generating executable C code \JWComment{Needs checking}}
\label{fig:generating_C}
\end{figure}

It is quite straightforward to convert from a term in $\Prog{\tau}$ to
a piece of executable C11 or assembly code. For example,
Fig.~\ref{fig:generating_C} demonstrates how we generate C11 code.

\subsection{Monotonicity results for x86} 
To extend the $strengthen$ relation to
the x86 setting, we add the constraint $\field{X}{locked} \subseteq
\field{Y}{locked}$ -- this allows the stronger execution to have
additional `locked' events. We used Alloy to search for monotonicity
violations in the x86-TSO model, as formalised by Alglave et
al.~\cite{alglave+14} -- that is, we sought solutions for
$\GeneralExec(\mm{x86-TSO},\mm{x86-TSO}, strengthen_{\rm x86})$. Alloy
discovered the following pair of executions:
%
\begin{equation*}
\begin{tikzpicture}[inner sep=1pt]
\node[event, anchor=west](a) at (2,2) 
{\evtlbl{$a$}$\evW{}{"x"}{1}$};

\node[event, anchor=west](b) at (2,1) 
{\evtlbl{$b$}$\evR{"lock"}{"y"}{0}$};

\node[event, anchor=west](c) at (4,2) 
{\evtlbl{$c$}$\evW{"lock"}{"y"}{1}$};

\node[event, anchor=west](d) at (4,1) 
{\evtlbl{$d$}$\evR{}{"x"}{0}$};

\node[event, anchor=west](a') at (6.5,2) 
{\evtlbl{$a'$}$\evW{}{"x"}{1}$};

\node[event, anchor=west](b') at (6.5,1) 
{\evtlbl{$b'$}$\evR{"lock"}{"y"}{0}$};

\node[event, anchor=west](c') at (8.5,2) 
{\evtlbl{$c'$}$\evW{"lock"}{"y"}{1}$};

\node[event, anchor=west](d') at (8.5,1) 
{\evtlbl{$d'$}$\evR{"lock"}{"x"}{0}$};

\foreach \i/\j in {a/b, a'/b',c/d, c'/d'}
\draw[edgesb] ([xshift=8mm]\i.south west) to[auto,swap,pos=0.4]
node{$sb$} ([xshift=8mm]\j.north west -| \i.south west);

\node[sthd, fit=(a)(b)] {};
\node[sthd, fit=(c)(d)] {};
\node[sthd, fit=(a')(b')] {};
\node[sthd, fit=(c')(d')] {};
\end{tikzpicture}
\end{equation*}
%
The left-hand execution is forbidden because a write cannot be
reordered with a later read if one of them is locked. The right-hand
execution is allowed because of a bug in Alglave et al.'s
formalisation of \mm{x86-TSO}: they neglect to forbid write/read
reordering when both the write ($c'$) and the read ($d'$) are
locked. Upon amending their model and rerunning our test, Alloy could
find no further monotonicity violations.

\subsection{Amusing remark about Prolog}

\begin{remark}Recalling Prolog's `green cuts' and `red
cuts'~\cite{bratko86}, we could describe a postcondition as `red' if
it rules out an execution that is consistent and racy, and `green'
otherwise. Herd and cppmem allow users the freedom to generate red
postconditions, for pedagogical reasons. In our work, we must only
generate green postconditions. \end{remark}

\subsection{Compiling C11 to Power}

\begin{figure}[t]
\begin{myFrame}{$compile_{\rm C11/PPC}(\pi,X,Y)$}
\begin{mathpar}
\labelb{eq:c11ppc:pi1}~
\stor{\field{X}{E}} = (\pi \join \pi^{-1})
\and
\labelb{eq:c11ppc:pi2}~
\stor{\field{Y}{E}} \subseteq (\pi^{-1} \join \pi)
\and
\labelb{eq:c11ppc:sb}~
\stack{
\forall (e_1,e_2)\in \field{X}{sb} \ldotp (\{e_1\}\join \pi) \times
(\{e_2\}\join\pi) \subseteq \field{Y}{sb}
}
\and
\labelb{eq:c11ppc:sloc}~
\field{X}{sloc} = \pi \join \field{Y}{sloc} \join \pi^{-1}
\and
\labelb{eq:c11ppc:sthd}~
\field{Y}{sthd} = \pi^{-1} \join \field{X}{sthd} \join \pi
\and
\labelb{eq:c11ppc:rf}~
\field{X}{rf} = \pi \join \field{Y}{rf} \join \pi^{-1}
\and
\labelb{eq:c11ppc:co}~
\field{X}{co} = \pi \join \field{Y}{co} \join \pi^{-1}
\and
\labelb{eq:c11ppc:Rrlx}~
\forall e\in \field{X}{R} - \field{X}{W} - \field{X}{acq}\ldotp
  \exists e' \in \field{Y}{R} \ldotp 
  (\{e\}\join\pi)  = \{e'\} 
\and
\labelb{eq:c11ppc:Racq}~
\stack{\forall e\in \field{X}{R} \cap \field{X}{acq} - \field{X}{W} -
\field{X}{sc}\ldotp {}\\{}
  \exists e'_1 \in \field{Y}{R}\ldotp 
  \exists e'_2 \in \field{Y}{isync}\ldotp
  (\{e\}\join\pi)  = \{e'_1\}\uplus\{e'_2\} \wedge {}\\{}
  (e'_1,e'_2) \in imm(\field{Y}{sb}) \cap \field{Y}{cd} -
  \field{Y}{ad} - \field{Y}{dd} \wedge {}\\{}
  (\forall e'\ldotp (e'_1,e') \in
  \field{Y}{sb} \Rightarrow (e'_1,e') \in \field{Y}{cd})}
\and
\labelb{eq:c11ppc:Rsc}~
\labelb{eq:c11ppc:Wrlx}~
\labelb{eq:c11ppc:Wrel}~
\labelb{eq:c11ppc:Wsc}~
\labelb{eq:c11ppc:RMWrlx}~
\labelb{eq:c11ppc:RMWacq}~
\labelb{eq:c11ppc:RMWrel}~
\labelb{eq:c11ppc:RMWar}~
\labelb{eq:c11ppc:RMWsc}~
\labelb{eq:c11ppc:Frlx}~
\labelb{eq:c11ppc:Far}~
\labelb{eq:c11ppc:Fsc}~\text{[see full version]}
\and
\labelb{eq:c11ppc:ad}~
\field{X}{ad} = \pi \join \field{Y}{ad} \join \pi^{-1}
\and
\labelb{eq:c11ppc:dd}~
\field{X}{dd} = \pi \join \field{Y}{dd} \join \pi^{-1}
\and
\labelb{eq:c11ppc:cd1}~
\stack{
\forall (e_1,e_2) \in \field{X}{cd} \ldotp \exists e'_1 \in
(\{e_1\}\join\pi)\ldotp {}\\{} (\{e'_1\}\times
(\{e_2\}\join\pi)) \subseteq \field{Y}{cd}
}
\and
\labelb{eq:c11ppc:cd2}~
\stack{
\forall (e'_1,e'_2) \in \field{Y}{cd} \ldotp \exists (e_1,e_2) \in
(\field{X}{cd})^{?} \ldotp {}\\{}
e'_1 \in (\{e_1\}\join\pi) \wedge 
e'_2 \in (\{e_2\}\join\pi)
}
\end{mathpar}
\end{myFrame}
\caption{Compilation scheme from C11 to Power}
\label{fig:c11_ppc}
\end{figure}

\label{sec:Q4_c11_ppc}
A C execution $X$ is deemed to compile to a Power execution $Y$ if
there exists a relation $\pi$ between the events of the former and
those of the latter such that $compile_{\rm C11/PPC}(\pi,X,Y)$ holds,
as defined in Fig.~\ref{fig:c11_ppc}.

The $\pi$ relation is injective, surjective and left-total (i.e., a
multivalued function)~\refb{eq:c11ppc:pi1}~\refb{eq:c11ppc:pi2}. 

The mapping must not remove $sb$ edges, but can add
them~\refb{eq:c11ppc:sb}. New $sb$ edges are created either when a
partial $sb$ is linearised, or when a single software-level event maps
to a chain of architecture-level events.

A non-atomic or relaxed read maps to a single
read~\refb{eq:c11ppc:Rrlx}. An acquire read maps to a pair of
architecture-level events, say $e'_1$ and $e'_2$, where $e'_1$ is a
read, $e'_2$ is an "isync", $e_1$ and $e_2$ are sequenced
consecutively, and every event in $Y$ that is sequenced after $e'_1$
becomes control-dependent on $e'_1$~\refb{eq:c11ppc:Racq}. The full
version of our mapping also includes SC reads~\refb{eq:c11ppc:Rsc},
relaxed/release/SC
writes~\refb{eq:c11ppc:Wrlx}~\refb{eq:c11ppc:Wrel}~\refb{eq:c11ppc:Wsc},
RMWs~\refb{eq:c11ppc:RMWrlx}~\refb{eq:c11ppc:RMWacq}~\refb{eq:c11ppc:RMWrel}~\refb{eq:c11ppc:RMWar}~\refb{eq:c11ppc:RMWsc},
and
fences~\refb{eq:c11ppc:Frlx}~\refb{eq:c11ppc:Far}~\refb{eq:c11ppc:Fsc}.

The mapping does not affect address or data
dependencies~\refb{eq:c11ppc:ad}~\refb{eq:c11ppc:dd}. \JWComment{Need
to think more about control
dependencies~\refb{eq:c11ppc:cd1}~\refb{eq:c11ppc:cd2}.} The mapping
does not change locations~\refb{eq:c11ppc:sloc} or
threading~\refb{eq:c11ppc:sthd}, and the $rf$ and $co$ maps are
unaffected~\refb{eq:c11ppc:rf}~\refb{eq:c11ppc:co}.

\paragraph{Results} We used Alloy to search for C11/Power compilation
bugs involving software-level executions of up to four events and
architecture-level executions of up to eight events. (Beyond this, the
SAT solver timed out.) We found no bugs.

We also used Alloy to search for bugs in two other, broadly similar,
C11 compilation schemes. The target memory models, which cannot be
named, are currently being developed by two major chip companies.
Alloy found a bug in one of the models, which led us to the
observation that the model was failing to include control dependencies
(though we have been unable to confirm this with the developers as an
actual bug).

\subsection{Results for initial writes / predicate calculus}

\begin{center}
\begin{tikzpicture}
\begin{semilogxaxis}[
  legend pos=south east,
  xtickten={3,4,5},
  ytick={1,3,5},
  x label style={at={(axis description cs:0,0.335)},anchor=north west},
  ylabel near ticks,
  ylabel={Task},
  xlabel={Solve time /s:},
  minor tick num=0,
  height=3cm,
  width=9.1cm,
]
\addplot[only marks, mark size=0pt]
plot [error bars/.cd, x dir=both, x explicit, error bar style={line
width=3pt, draw=green}] 
table [y=task, x=init, x error minus expr=\thisrow{init}-\thisrow{initmin}, x
error plus expr=\thisrow{initmax}-\thisrow{init}, col sep=comma]
{../models/tests/results_init_exp/Sheet 1-Table 2.csv};

\addplot[only marks, mark size=0pt]
plot [error bars/.cd, x dir=both, x explicit, error bar style={line
width=3pt, draw=red}] 
table [y expr=\thisrow{task}-0.5, x=normal, x error minus expr=\thisrow{normal}-\thisrow{normalmin}, x
error plus expr=\thisrow{normalmax}-\thisrow{normal}, col sep=comma]
{../models/tests/results_init_exp/Sheet 1-Table 2.csv};

\addplot[only marks, mark size=0pt]
plot [error bars/.cd, x dir=both, x explicit, error bar style={line
width=3pt, draw=blue}] 
table [y expr=\thisrow{task}+0.5, x=exp, x error minus
expr=\thisrow{exp}-\thisrow{expmin}, x
error plus expr=\thisrow{expmax}-\thisrow{exp}, col sep=comma]
{../models/tests/results_init_exp/Sheet 1-Table 2.csv};
%\legend{initial writes, normal, predicate}
\end{semilogxaxis}
\end{tikzpicture}
\end{center}

\subsection{Other nice things about Alloy}

Alloy has several qualities that
make it ideal, we believe, for this kind of work:

\begin{description}

\item[Modularity] Alloy's type system supports subtyping, which we
exploit by having, for instance, \emph{x86 executions} inherit the
constraints of \emph{architecture-level executions}, which inherit the
constraints of \emph{executions}. This, together with Alloy's simple
module system that allows generic definitions, preserves the
readability of our models. 

\item[Debuggability] If Alloy fails, it fails `visibly' (to coin a
phrase by Padon et al.~\cite{padon+16}). That is, if Alloy produces a
bogus solution, it is typically straightforward to identify
constraints to strengthen. And if Alloy fails to find a known
solution, it helpfully highlights the minimal subset of unsatisfiable
constraints, which simplifies the subsequent search for constraints to
weaken.

\item[Maturity] Having been in development for over fifteen
years~\cite{jackson+01}, Alloy has a large user base and support
community. It is also open-source and offers an easy-to-use API. 
%
\end{description}

\subsection{Architecture-level executions}

\begin{figure}[t]
\paragraph{Sets of events ($\subseteq \E$):} ~\\
\quad\begin{tabular}{ll}
$atom$ & relates the components of an RMW
\end{tabular} \\~

\paragraph{Well-formedness constraints:}
\begin{mathpar}
\labelb{eq:hwexec:sbtot}~ \sthd \subseteq sb^? \cup (sb^{-1})^?
\and
\labelb{eq:hwexec:nona}~ nal = \emptyset
\and
\labelb{eq:hwexec:noRMW}~ R \ncap W
\and
\labelb{eq:hwexec:atom}~ atom \subseteq (R\times W) \cap \imm(sb)
\end{mathpar}
\caption{Architecture-level executions, $\ExecH$ (extending $\Exec$)}
\label{fig:hw_exec}
\end{figure}

An architecture-level execution is a structure that inherits all the
fields and well-formedness conditions from basic executions, and
adds those listed in Fig.~\ref{fig:hw_exec}. Since the structure of
assembly code does not permit ambiguous evaluation orders,
sequenced-before becomes a total relation within each
thread~\refb{eq:hwexec:sbtot}. There is no concept of a non-atomic
location at the architecture level~\refb{eq:hwexec:nona}. It is
commonplace in architecture-level MCMs to separate the read and the
write of an RMW into separate events~\refb{eq:hwexec:noRMW}, and
connect them with the $atom$
relation~\refb{eq:hwexec:atom}~\cite{alglave+14}. This differs from
the C11 approach, where an RMW is a single event belonging to both $R$
and $W$.

\subsection{Relation calculus vs. predicate calculus}

Alloy supports concise relation calculus expressions like
$irreflexive(rf\join hb)$ and more powerful predicate calculus
expressions like $\forall e_1,e_2\ldotp (e_1,e_2)\in rf \Rightarrow
(e_2,e_1)\notin hb$. We measured the impact on solving time when a few
tasks (1, 3, and 5) were converted to use the latter style, and found
it to be negligible. 

\subsection{Bit where we work through a few examples of `generating
litmus tests'}

The following example illustrates
how this technique can obtain conformance tests (\Q1) for the C11
memory model.

\begin{Example}
Consider the following C11 execution, in which ${\rm W}_{"rlx"}$ and
${\rm R}_{"rlx"}$ denote atomic write and read events with `relaxed'
memory order.
%
\begin{equation}
\label{exec:goodtest}
\begin{tikzpicture}[inner sep=1pt, baseline=(a.base)]
\node (a) at (2,0.8) {$\evR{"rlx"}{"x"}{1}$};
\node (b) at (2,0) {$\evR{"rlx"}{"x"}{0}$};
\node (c) at (0,0.8) {$\evW{"rlx"}{"x"}{1}$};
\draw[edgerf] (c) to [auto] node {$rf$} (a);
\draw[edgesb] (a) to [auto] node {$sb$} (b);
\node[sthd, fit=(a)(b)] {};
\node[sthd, fit=(c)] {};
\end{tikzpicture}
\end{equation}
The execution is inconsistent in C11, because of a coherence violation. It can be
reverse-engineered into a litmus test
\begin{equation*}
\litmustestII{atomic\_int x=0;}{
\texttt{x.store(1,RLX);} & \texttt{r0=x.load(RLX);} \\
                         & \texttt{r1=x.load(RLX);} \\
}{r0==1 \&\& r1==0}
\end{equation*}
that is both useful and conclusive.
\end{Example}

Unfortunately, litmus tests generated in this way are not always
\emph{conclusive}. If a particular execution is \emph{allowed} under a
memory model, we can easily construct a litmus test that \emph{can
pass}, but if a particular execution is \emph{forbidden} under a
memory model, it is not straightforward to construct a litmus test
that \emph{must fail}, because the constructed litmus test may still
be able to pass via a different execution that is allowed. The
following two examples demonstrate why it is not enough for the
execution merely to be inconsistent (and hence why we further impose
deadness).

\begin{Example}
\label{ex:badtest_racy}
Consider the following C11 execution, in which "na" indicates events
corresponding to non-atomic operations.
\begin{equation}
\label{exec:badtest_racy}
\begin{tikzpicture}[inner sep=1pt, baseline=(a.base)]
\node (a) at (0,0) {$\evW{"na"}{"a"}{1}$};
\node (b) at (2,0) {$\evR{"na"}{"a"}{1}$};
\draw[edgerf] (a) to [auto] node {$rf$} (b);
\node[sthd, fit=(a)] {};
\node[sthd, fit=(b)] {};
\end{tikzpicture}
\end{equation}
It is inconsistent in C11 (because a read of a non-atomic location
must only observe a write that happens before it), but the litmus test
obtained from it
\begin{equation*}
\litmustestII{int a=0;}{
\texttt{a=1;} & \texttt{r0=a;} \\
}{r0==1}
\end{equation*}
is not conclusive because it admits the execution
\begin{equation*}
\begin{tikzpicture}[inner sep=1pt]
\node (a) at (0,0) {$\evW{"na"}{"a"}{1}$};
\node (b) at (2,0) {$\evR{"na"}{"a"}{0}$};
%\draw[edgerf] (a) to [auto] node {$rf$} (b);
\node[sthd, fit=(a)] {};
\node[sthd, fit=(b)] {};
\end{tikzpicture}
\end{equation*}
%
which is both consistent and racy. (Any behaviour is allowed for a
program that admits a consistent and racy execution.) \end{Example}

\begin{Example} 
\label{ex:badtest_co}
Consider the following C11 execution, in which $co$
denotes the coherence order on writes.
\begin{equation}
\label{exec:badtest_co}
\begin{tikzpicture}[inner sep=1pt, baseline=(a.base)]
\node (a) at (0,0.8) {$\evtlbl{$a$}\evW{"rlx"}{"x"}{2}$};
\node (b) at (0,0) {$\evtlbl{$b$}\evW{"rlx"}{"x"}{1}$};
\node (c) at (2.5,0.8) {$\evtlbl{$c$}\evW{"rlx"}{"x"}{3}$};
\node (d) at (2.5,0) {$\evtlbl{$d$}\evR{"rlx"}{"x"}{3}$};
\draw[edgerf] (c) to [auto, bend right, swap] node {$rf$} (d);
\draw[edgesb] (a) to [auto, bend left] node {$sb$} (b);
\draw[edgesb] (c) to [auto, bend left] node {$sb$} (d);
\draw[edgemo] (b) to [auto, bend left] node {$co$} (a);
\draw[edgemo] (a) to [auto] node {$co$} (c);
\node[sthd, fit=(a)(b)] {};
\node[sthd, fit=(c)(d)] {};
\end{tikzpicture}
\end{equation}
It is inconsistent in C11 (because of a coherence violation), but
the litmus test obtained from it
\begin{equation*}
\litmustestII{atomic\_int x=0;}{
\texttt{x.store(2,RLX);} & \texttt{x.store(3,RLX);} \\
\texttt{x.store(1,RLX);} & \texttt{r0=x.load(RLX);} \\
}{x==3 \&\& r0==3}
\end{equation*}
is not conclusive because its final state can be obtained via
a consistent execution that simply reverses the $co$ edge
in~\eqref{exec:badtest_co} from $(b,a)$ to $(a,b)$. 
\end{Example}

The purpose of the `$dead$' condition in
Def.~\ref{def:general_problem_executions} is to rule out executions
like \eqref{exec:badtest_racy} that give rise to litmus tests that can
pass because they are racy, as well as executions like
\eqref{exec:badtest_co} whose litmus tests can pass via another
consistent execution, but to permit executions
like~\eqref{exec:goodtest}. With a suitable definition of `$dead$'
(\S\ref{sec:safety}), and careful restrictions on the target
programming language (\S\ref{sec:language}), our approach ensures that
if $(X,Y)$ solves $\GeneralExec$, and if $X$ and $Y$ are maximal
candidate executions of programs $P$ and $Q$ (respectively) that reach
$\sigma$, then $(P,Q,\sigma)$ solves the original $\GeneralProg$
problem, as required. We find that obtaining $(P,Q,\sigma)$ in this
indirect way is a much more efficient strategy than solving
$\GeneralProg$ directly.

\subsection{Constructing coherence}
\label{sec:sezgin}

Sezgin~\cite{sezgin04} has asked whether $co$ can constructed from the
other relations of an execution, such as $sb$ and $rf$, rather than
simply declared to exist. He has shown that an attempt to characterise
SC executions in this way falls short, by exhibiting a 16-event
execution~\cite[Fig.~4.3]{sezgin04} that is judged SC according to the
`constructed $co$' approach, but is nonetheless not SC. We encoded his
$co$ construction into Alloy and sought a discrepancy with the SC
MCM. Unfortunately, Sezgin's 16-event example proved to be out of
Alloy's reach. This problem is much more computationally expensive
than the problem considered in \S\ref{sec:kyndylan}, because to show
that an execution is \emph{not} SC, we must universally quantify over
all possible $co$ relations.


\subsection{Handling (mutually) recursive definitions}
\label{sec:fixpoints}

The axiomatic Power MCM derives four relations (called $ic$, $ii$,
$ci$ and $cc$) via mutual recursion. In the ".cat"
format~\cite{alglave+14}, one can write `"let rec $ic = f_0(ic,ii,ci,cc)$ and $ii =
f_1(ic,ii,ci,cc)$ and $ci = f_2(ic,ii,ci,cc)$ and $cc = f_3(ic,ii,ci,cc)$"', but Alloy does not provide recursive
definitions. Accordingly, we expand the recursive construction explicitly, using the constraints
\begin{mathpar}
\labelb{ppcrec:0}~
f_0(ic,ii,ci,cc) \subseteq ic
\and
\labelb{ppcrec:1}~
f_1(ic,ii,ci,cc) \subseteq ii
\and
\labelb{ppcrec:2}~
f_2(ic,ii,ci,cc) \subseteq ci
\and
\labelb{ppcrec:3}~
f_3(ic,ii,ci,cc) \subseteq cc
\and
\labelb{ppcrec:all}~
\stack{\forall ic',ii',ci',cc'\ldotp\\
\quad (f_0(ic',ii',ci',cc') \subseteq ic' \wedge
f_1(ic',ii',ci',cc') \subseteq ii' \wedge {}\\
\quad \phantom(f_2(ic',ii',ci',cc') \subseteq ci' \wedge
f_3(ic',ii',ci',cc') \subseteq cc') \Rightarrow {}\\
\quad\quad (ci \subseteq ci' \wedge ii \subseteq ii' \wedge cc \subseteq cc' \wedge ic \subseteq ic')}
\end{mathpar}
%
to express that $(ic,ii,ci,cc)$ is both a
pre-fixpoint~\refb{ppcrec:0}~\refb{ppcrec:1}~\refb{ppcrec:2}~\refb{ppcrec:3}
and the least such~\refb{ppcrec:all}. This last constraint involves
universal quantification over relations and hence requires the
higher-order AlloyStar solver. A first-order alternative is simply to
unroll the recursive definitions a few times; that is, to define
$ic = f_0^k(\emptyset,\emptyset,\emptyset,\emptyset)$ (and $ii$, $ci$,
and $cc$ similarly) for a fixed $k$. We found, for a search scope of 4
software events and 8 hardware events, that $k\ge 2$ is sufficient for
avoiding false positives. Figure~\ref{fig:fixpoint_results} shows that
the proper fixpoint construction (i.e., $k=\infty$) is considerably
more expensive than a fixed unrolling.

\subsection{MP bug in OpenCL/AMD}

\begin{figure}
\newcommand\ST[1]{\begin{array}{@{}c@{}}#1\end{array}}
\newcommand\LG[2]{{$\scriptsize\left(\ST{\sigma_{#1}\\
\sigma_{#2}}\right)$}}
\begin{tikzpicture}[inner sep=1pt]

\node[event, anchor=west](a) at (-4.9,0) 
{$\evW{\na}{"x"}{1}$};

\node[event, anchor=west](b) at (-4.9,-0.8) 
{$\evW{\morel,\msdv}{"y"}{1}$};

\node[event, anchor=west](c) at (-3.3,0) 
{$\evR{\moacq,\msdv}{"y"}{1}$};

\node[event, anchor=west](d) at (-3.3,-0.8) 
{$\evR{\na}{"x"}{0}$};

\node[sthd, fit=(a)(b)] {};
\node[sthd, fit=(c)(d)] {};

\draw[edgerf] (b) to[auto, pos=0.4] node{$rf$} (c);
\foreach \i/\j in {a/b}
\draw[edgesb] ([xshift=8mm]\i.south west) to[auto,swap,pos=0.4]
node{$sb$} ([xshift=8mm]\j.north west -| \i.south west);
\foreach \i/\j in {c/d}
\draw[edgesb] ([xshift=8mm]\i.south west) to[auto,swap,pos=0.6]
node{$sb$,$cd$} ([xshift=8mm]\j.north west -| \i.south west);

\node (inv1) at (2.3,0) {\LG04$\evInv$\LG04};
\node (efet1) [below=4mm of inv1] {\LG04$\evEFet{"x"}$\LG34};
\node (sto1) at (0,0) {\LG04$\evW{}{"x"}{1}$\LG14};
\node (eflu1) [below=4mm of sto1] {\LG14$\evEFlu{"x"}$\LG25};
\node (flu1) [below=4mm of eflu1] {\LG25$\evFlu$\LG25};
\node (sto2) [below=4mm of flu1] {\LG25$\evW{}{"y"}{1}$\LG65};
\node (eflu2) [below=4mm of sto2] {\LG65$\evEFlu{"y"}$\LG77};
\node (efet2) [below=4mm of efet1] {\LG37$\evEFet{"y"}$\LG87};
\node (loa1) [below=4mm of efet2] {\LG87$\evR{}{"y"}{1}$\LG87};
\node (loa2) [below=4mm of loa1] {\LG87$\evR{}{"x"}{0}$\LG87};

\draw[edgerf] (sto2) to[auto, pos=0.1, bend left=40] node[inner sep=0]{$rf$} (loa1);

\coordinate (wg1nw) at ([yshift=2mm]sto1.north west);
\coordinate (wg1ne) at ([yshift=2mm]sto1.north east);
\coordinate (wg1sw) at ([yshift=-2mm]eflu2.south west);
\coordinate (wg1se) at ([yshift=-2mm]eflu2.south east);

\coordinate (wg2nw) at ([yshift=2mm]inv1.north west);
\coordinate (wg2ne) at ([yshift=2mm]inv1.north east);
\coordinate (wg2sw) at ([yshift=-2mm]loa2.south west);
\coordinate (wg2se) at ([yshift=-2mm]loa2.south east);

\node[sthd_tight, fit=(sto1)(eflu2)] {};
\node[sthd_tight, fit=(inv1)(loa2)] {};

\draw[edgepi, overlay] (a) to[auto, out=30, in=180,pos=0.8] node{$\pi$} (sto1);
\draw[edgepi] (b) to[auto, out=330, in=180,pos=0.4] node{$\pi$} (flu1);
\draw[edgepi] (b) to[auto, out=330, in=160] (sto2);
\draw[edgepi] (c) to[auto, out=350, in=200, swap, pos=0.2] node{$\pi$} (inv1);
\draw[edgepi] (c) to[auto, out=350, in=120] (loa1);
\draw[edgepi] (d) to[auto, bend right=10, pos=0.1] node{$\pi$} (loa2);

\draw[edgethen] (inv1) to (efet1);
\draw[edgethen] (efet1) to[out=175,in=360] (sto1);
\draw[edgethen] (sto1) to (eflu1);
\draw[edgethen] (eflu1) to (flu1);
\draw[edgethen] (flu1) to (sto2);
\draw[edgethen] (sto2) to (eflu2);
\draw[edgethen] (eflu2) to[out=0,in=180] (efet2);
\draw[edgethen] (efet2) to (loa1);
\draw[edgethen] (loa1) to (loa2);

\node[anchor=north west] at (-5.0,-1.25) {$\begin{array}{@{}r@{\,}c@{\,}l@{}} 
\sigma_0 &=& \{\} \\
\sigma_1 &=& \{\ST{"x"\mapsto_{\rm dv}1}\} \\
\sigma_2 &=& \{\ST{"x"\mapsto_{\rm cv}1}\} \\
\sigma_3 &=& \{\ST{"x"\mapsto_{\rm cv}0}\} \\
\sigma_4 &=& \{\ST{"x"\mapsto_{\rm cv}0, 
                   "y"\mapsto_{\rm cv}0}\} \\
\sigma_5 &=& \{\ST{"x"\mapsto_{\rm cv}1, 
                   "y"\mapsto_{\rm cv}0}\} \\
\sigma_6 &=& \{\ST{"x"\mapsto_{\rm cv}1, 
                   "y"\mapsto_{\rm dv}1}\} \\
\sigma_7 &=& \{\ST{"x"\mapsto_{\rm cv}1, 
                   "y"\mapsto_{\rm cv}1}\} \\
\sigma_8 &=& \{\ST{"x"\mapsto_{\rm cv}0, 
                   "y"\mapsto_{\rm cv}1}\}
\end{array}$};
\end{tikzpicture}
\caption{Message-passing bug in OpenCL/AMD compilation}
\label{fig:mp_bug}
\end{figure}

See Fig.~\ref{fig:mp_bug}.

\subsection{Other related work}

\begin{itemize}

\item Horn et al.~\cite{horn+15} have a paper that is something to do
with SAT-solving and weak memory, and also mentions the N-free
property.

\item Lau et al. have looked at how to reverse-engineer a program
by examining executions~\cite{lau+03}.

\item Maybe mention future application to crash-consistency models
(Bornholt et al, ASPLOS 16, http://homes.cs.washington.edu/~bornholt/papers/ferrite-asplos16.pdf)

\item We remark that consistency models for distributed systems often
resemble those for memory systems, and we expect that our comparison
techniques may prove helpful in that domain too -- potentially using
Cerone et al.'s axiomatic treatment of various notions of consistency
as a starting point~\cite{cerone+15}.

\item Regarding the SC-DRF guarantee (\Q2): Batty et al. have proved it
manually for C11, while Bouajjani et al.~\cite{bouajjani+11} have
analysed the complexity of automatically checking it for the TSO
memory model~\cite{owens+09}.

\item Jade's SFM lecture notes?~\cite{alglave15}

\end{itemize}