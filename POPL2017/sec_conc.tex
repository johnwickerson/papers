\section{Conclusion}
\label{sec:conc}

By solving relational constraints between executions and then lifting
solutions to litmus tests, Alloy can generate conformance tests,
compare MCMs (against each other and against general properties like
SC-DRF and multi-copy atomicity), and check monotonicity and compiler
mappings. As such, we believe that Alloy should become an essential
component of the memory modeller's toolbox. Indeed, we are already
working with two large processor vendors to apply our technique to
their recent and upcoming architectures and languages. Other future
work includes applying our technique to more recent MCMs that
are defined in a non-axiomatic style~\cite{jeffrey+16,
pichon-pharabod+16, flur+16, kang+17}.

Although Alloy's lightweight, automatic approach cannot give the same
universal assurance as fully mechanised theorems, we have found it
invaluable in practice, because even (and perhaps \emph{especially})
in the complex and counterintuitive world of non-SC MCMs, Jackson's
maxim~\cite{jackson12a} holds true: \emph{Most bugs have small
counterexamples}.
