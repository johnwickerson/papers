\section{Application: Checking Monotonicity (\Q3)}
\label{sec:monotonicity}

In this section, we rediscover two monotonicity violations in the C11
MCM, whereby a new behaviour is enabled either by sequentialisation
(\S\ref{sec:Q3_c11_seq}) or by strengthening a memory order
(\S\ref{sec:Q3_c11_mo}).

\begin{figure}
\centering
\begin{myFrame}{$strengthen(X,Y)$}
\begin{mathpar}
\labelb{eq:monot:ev}~
\field{X}{E} = \field{Y}{E}
\and
\labelb{eq:monot:r}~
\field{X}{R} = \field{Y}{R}
\and 
\labelb{eq:monot:w}~
\field{X}{W} = \field{Y}{W}
\and 
\labelb{eq:monot:f}~
\field{X}{F} = \field{Y}{F} 
\and
\labelb{eq:monot:nal}~
\field{X}{nal} = \field{Y}{nal} 
\and
\labelb{eq:monot:sb}~
\field{X}{sb} \subseteq \field{Y}{sb}
\and
\labelb{eq:monot:ad}~
\field{X}{ad} \subseteq \field{Y}{ad}
\and
\labelb{eq:monot:cd}~
\field{X}{cd} \subseteq \field{Y}{cd}
\and
\labelb{eq:monot:dd}~
\field{X}{dd} \subseteq \field{Y}{dd}
\and
\labelb{eq:monot:sthd}~
\field{X}{\sthd} \subseteq \field{Y}{\sthd}
\and
\labelb{eq:monot:sloc}~
\field{X}{\sloc} = \field{Y}{\sloc}
\and
\labelb{eq:monot:rf}~
\field{X}{rf} = \field{Y}{rf}
\and
\labelb{eq:monot:co}~
\field{X}{co} = \field{Y}{co}
\end{mathpar}
\end{myFrame}
\caption{`Strengthening' an execution}
\label{fig:strengthen}
\end{figure}

Checking monotonicity requires the relation defined in
Fig.~\ref{fig:strengthen}, which holds when one execution is
`stronger' than another. There is almost an isomorphism between $X$
and $Y$, except that $Y$ may have extra $sb$ and dependency
edges~\refb{eq:monot:sb}~\refb{eq:monot:ad}~\refb{eq:monot:cd}~\refb{eq:monot:dd},
and it may interleave multiple threads together~\refb{eq:monot:sthd}.

% When Alloy finds $X$ and $Y$ that witness non-monotonicity at the
% execution level ($X$ is inconsistent, $Y$ is consistent, and $Y$ is
% stronger than $X$), we search for programs $P$ and $Q$ that witness
% non-monotonicity at the source-code level. We constrain $P$ and $Q$
% to have $X$ and $Y$ as maximal candidate executions respectively,
% and $Q$ to be stronger than $P$ in the sense that $P$ and $Q$
% contain the same instructions, control dependencies and sequencing
% between instructions in $P$ are preserved in $Q$, and instructions
% in the same thread in $P$ are in the same thread in $Q$.

\subsection{Monotonicity of C11 w.r.t. Sequencing}
\label{sec:Q3_c11_seq}
One way to extend the $strengthen$ relation to the C11 setting is to
add the following constraints:
%
\begin{mathpar}
\field{X}{A} = \field{Y}{A} 
\and
\field{X}{acq} = \field{Y}{acq} 
\and
\field{X}{rel} = \field{Y}{rel} 
\and
\field{X}{sc} = \field{Y}{sc}.
\end{mathpar}
%
With this notion of strengthening, Alloy found a pair of 6-event
executions that witness a monotonicity violation in C11. They are
almost identical to those given by Vafeiadis et
al.~\cite[Fig.~1]{vafeiadis+15}, though slightly less elegant: Alloy
chose one of the writes to be SC when a relaxed write would have
sufficed.

\subsection{Monotonicity of C11 w.r.t. Memory Orders}
\label{sec:Q3_c11_mo}
Another way to extend the $strengthen$ relation is to add
%
\begin{mathpar}
\field{X}{A} = \field{Y}{A} 
\and
\field{X}{acq} \subseteq \field{Y}{acq} 
\and
\field{X}{rel} \subseteq \field{Y}{rel} 
\and
\field{X}{sc} \subseteq \field{Y}{sc}
\and
\field{X}{sb} = \field{Y}{sb}
\and
\field{X}{ad} = \field{Y}{ad}
\and
\field{X}{cd} = \field{Y}{cd}
\and
\field{X}{dd} = \field{Y}{dd}
\and
\field{X}{\sthd} = \field{Y}{\sthd}
\end{mathpar}
%
which allows memory orders to be strengthened but forbids changes
to sequencing or threading. Using this notion of strengthening, Alloy
found a 7-event monotonicity violation in C11: the execution
previously given in Example~\ref{ex:c11_not_mono}. That execution is
inconsistent, but if the relaxed fence is strengthened to a release
fence, it becomes consistent. This example is similar to one due to
Vafeiadis et al.~\cite[\S3, \emph{Strengthening is
Unsound}]{vafeiadis+15}, but it is simpler: where theirs requires 8
events, 4 locations, and 3 threads, our Alloy-generated example
requires only 7 events, 3 locations, and 2 threads.

%%% Local Variables:
%%% mode: latex
%%% TeX-master: "paper"
%%% End:
