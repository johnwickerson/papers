\documentclass[preprint]{sigplanconf}

\usepackage{amsmath}
\usepackage{amssymb}
%\usepackage{datetime}
\usepackage[usenames,dvipsnames,table]{xcolor}
\usepackage[T1]{fontenc} % for proper _ in tt mode
\usepackage[utf8]{inputenc}
\usepackage{mathpartir}
\mprset{andskip=0.8em plus 0.5fil}
\makeatletter
\def\mpr@paroptions{\mpr@lesslineskip}
\makeatother
\usepackage{mathtools}
\usepackage{bbm}
\usepackage{amsthm}
\usepackage{framed}
\usepackage{fixltx2e} % fixes \textsubscript macro
\usepackage{stmaryrd}
\usepackage{colortbl}
\usepackage{booktabs}
\usepackage{tabularx}
\usepackage[ruled, linesnumbered]{algorithm2e}
%\usepackage[noend]{algpseudocode}
%\renewcommand\textfraction{0}

% Ticks and crosses {
\usepackage{pifont}% http://ctan.org/pkg/pifont
\newcommand\cmark{\ding{51}}%
\newcommand\xmark{\ding{55}}%
% }

% Inline lists {
\usepackage[inline]{enumitem}
\newenvironment{inlineitemize}{
\begin{itemize*}[label={},afterlabel={},before=\unskip]
}{
\end{itemize*}
}
\newenvironment{inlineenumerate}{
\begin{enumerate*}[label={\arabic*.}]
}{
\end{enumerate*}
}
% }

% Fiddling with the style of {description} lists. {
\renewcommand{\descriptionlabel}[1]{%
\hspace{\labelsep}\emph{#1}\hspace*{1.5mm}%
}
% }

% Reduce spacing above/below figures {
\setlength{\floatsep}{16pt plus 4pt minus 2pt}
\setlength{\textfloatsep}{16pt plus 4pt minus 3pt}
% }

% Fancy frames {

\definecolor{shadecolor}{HTML}{EEEEEE}

% \newcommand\myFrame[2]{%
% \noindent
% \begin{tikzpicture}
% \node[inner sep=2mm, draw=black!20, line width=0.8mm, text
% width=\dimexpr\linewidth-4mm-\pgflinewidth\relax] (T)
% {\vspace*{1mm}\par #2};
% \node[inner sep=1mm, draw=none,fill=white, align=left,
% anchor=west] (H) at (T.north west)[xshift=3mm] {#1};
% \end{tikzpicture}%
% \par
% }

% \newcommand\myFrame[2]{%
% \noindent
% \definecolor{shadecolor}{HTML}{FFFFFF}
% \begin{framed}
% \smash{\raisebox{0mm}{\hspace*{1mm}\tikz\node[draw=black, fill=red]{#1};}} #2
% \end{framed}
% \par}

\usepackage{mdframed}
\newenvironment{myFrame}[1]{%
\mdfsetup{%
frametitle={%
\tikz[baseline=(current bounding box.east),outer sep=0pt]
\node[anchor=east,rectangle,fill=white] {#1};},
innertopmargin=-10pt,linecolor=black!20,linewidth=2pt,topline=true,
frametitleaboveskip=\dimexpr-\ht\strutbox\relax}
\begin{mdframed}[]\relax%
}{\end{mdframed}}
% }


% Theorems {
\newtheorem{theorem}{Theorem}
\newtheorem{lemma}{Lemma}
\theoremstyle{definition}
\newtheorem{definition}[lemma]{Definition}
\newtheorem{remark}[lemma]{Remark}
\newtheorem{example}{Example}
\newtheorem*{notation}{Notation}
\newenvironment{Solution}{\begin{proof}[Solution]}{\end{proof}}
\newenvironment{commentary}{\begin{proof}[Commentary]\phantom\qedhere}{\end{proof}}
\newenvironment{Example}{\begin{snugshade}\begin{example}}{\end{example}\end{snugshade}}
\setlength\FrameSep{1.5mm}
\newenvironment{Theorem}{\begin{oframed}\begin{theorem}}{\end{theorem}\end{oframed}}
% }

% Easy line breaks in displaymath mode {
\newcommand\stack[2][l]{{\renewcommand\arraystretch{1}\begin{array}[t]{@{}#1@{}}#2\end{array}}}
% }

% Start of 'ignore natbib' hack
\let\bibhang\relax
\let\citename\relax
\let\bibfont\relax
\let\Citeauthor\relax
\expandafter\let\csname ver@natbib.sty\endcsname\relax
% End of 'ignore natbib' hack
\usepackage[backend=bibtex, sortcites, maxnames=10]{biblatex}

\renewcommand*{\postnotedelim}{\space}
\let\oldcite\cite
\renewcommand\cite[2][]{%
\ifx\\#1\\%
  \oldcite{#2}%
\else% 
  \oldcite[(#1)]{#2}%
\fi}

%\addbibresource{/Users/jpw48/Dropbox/John.bib}
% cd ~/git/memalloy/paper && bibexport -o memalloy.bib paper.aux
\addbibresource{memalloy.bib}

\usepackage{tikz}
\usetikzlibrary{matrix,fit,calc,positioning,backgrounds,arrows,patterns,decorations}
\tikzset{event/.style={draw=none, inner sep=0.4mm}}
% {
\pgfdeclaredecoration{simple line}{initial}{
\state{initial}[width=\pgfdecoratedpathlength-1sp]{%
  \pgfmoveto{\pgfpointorigin}}
\state{final}{\pgflineto{\pgfpointorigin}}
}
\tikzset{
   shift left/.style={%
     decorate,decoration={simple line,raise=#1}},
   shift right/.style={%
     decorate,decoration={simple line,raise=-1*#1}},
}
% }

\usepackage{pgfplots}
\pgfplotsset{
  discard if not/.style 2 args={
    x filter/.code={
      \edef\tempa{\thisrow{#1}}
      \edef\tempb{#2}
      \ifx\tempa\tempb
      \else
        \def\pgfmathresult{inf}
      \fi
    }
  }
}

% Coloring edges in execution graphs {
\definecolor{colormo}{HTML}{3366CC}
\definecolor{colorrf}{HTML}{FF0000}
\definecolor{colorfr}{HTML}{006600}
\definecolor{colorhb}{HTML}{660000}
\definecolor{colorS}{HTML}{990099}

\tikzset{
  edgemo/.style={colormo,-latex},
  edgerf/.style={colorrf,-latex},
  edgefr/.style={colorfr,dashed,-latex},
  edgeS/.style={colorS, -latex},
  edgesb/.style={black, -latex},
  edgethd/.style={black, latex-latex},
  edgepi/.style={black!40, dashed, -latex},
  sthd/.style={draw, dotted, inner sep=1mm, rounded corners},
  sthd_qtight/.style={draw, dotted, inner sep=0.2mm, rounded corners},
  swg/.style={draw, dotted, inner sep=1mm, rounded corners},
  sthd_tight/.style={draw, dotted, inner sep=0mm, rounded corners},
  edgethen/.style={black, line width=1pt, ->},
}
% }

% Memory orders {
\newcommand\morlx{{\tt RLX}}
\newcommand\moacq{{\tt ACQ}}
\newcommand\mocon{{\tt CON}}
\newcommand\morel{{\tt REL}}
\newcommand\moar{{\tt AR}}
\newcommand\mosc{{\tt SC}}
\newcommand\msdv{{\tt DV}}
\newcommand\mswg{{\tt WG}}
\newcommand\na{{\tt na}}
% }

% Memory events {
\newcommand\evW[3]{{\rm W}_{#1}\,{#2}\,{#3}}
\newcommand\evR[3]{{\rm R}_{#1}\,{#2}\,{#3}}
\newcommand\evRMW[4]{{\rm C}_{#1}\,{#2}\,{#3}/{#4}}
\newcommand\evF[1]{{\rm F}_{#1}}
\newcommand\evEFlu[1]{\mathit{flu}\,{#1}}
\newcommand\evEFet[1]{\mathit{fet}\,{#1}}
\newcommand\evFlu{{\rm Flu}}
\newcommand\evInv{{\rm Inv}}
\newcommand\evRemInv{{\rm InvA}}
\newcommand\evLk[1]{{\rm Lk}\,{#1}}
\newcommand\evUk[1]{{\rm Uk}\,{#1}}
\newcommand\evWCL[4]{{\rm W}_{#1,#4}\,{#2}\,{#3}}
\newcommand\evRCL[4]{{\rm R}_{#1,#4}\,{#2}\,{#3}}

% }

\newcommand\evtlbl[1]{\mbox{#1:~}}

\usepackage[overlay,absolute]{textpos}

\newcommand\badgeshift{\hspace*{5mm}}

\title{Automatically Comparing Memory Consistency Models}
\authorinfo{John Wickerson}{Imperial College London, UK}{j.wickerson@imperial.ac.uk}
\authorinfo{Mark Batty\badgeshift}{University of Kent, UK\badgeshift}{m.j.batty@kent.ac.uk\badgeshift}
\authorinfo{Tyler Sorensen}{Imperial College London, UK}{t.sorensen15@imperial.ac.uk}
\authorinfo{George A. Constantinides\badgeshift}{Imperial College London,
UK\badgeshift}{g.constantinides@imperial.ac.uk\badgeshift}

\makeatletter
\newcommand{\customlabel}[2]{%
\protected@write \@auxout {}{\string \newlabel {#1}{{#2}{}}}}
\makeatother

\newcommand\blacknum[1]{%
\begin{tikzpicture}[baseline=-0.3em]%
\node[circle, draw=none, fill=black, inner sep=0pt, outer sep=0pt, minimum
size=0.9em](a){%
\clap{\textcolor{white}{\tiny\sf\bfseries #1}}
};%
\end{tikzpicture}%
}

\newcounter{ctrb}
\setcounter{ctrb}{0}
\newcommand\labelb[1]{
\stepcounter{ctrb}
\customlabel{#1}{\thectrb}
\blacknum{\thectrb}
}
\newcommand\refb[1]{\blacknum{\ref{#1}}}

\newif\iftodos
\todostrue % Turn on/off TODO comments
\newcommand\JWComment[1]{\iftodos{\textcolor{Purple}{{\bf [[JW:}
#1{\bf ]]}}}\fi}
\newcommand\TSComment[1]{\iftodos{\textcolor{blue}{{\bf [[TS:}
#1{\bf ]]}}}\fi}
\newcommand\MBComment[1]{\iftodos{{\color{Green}{{\bf [[MB:}
#1{\bf ]]}}}}\fi}
\newcommand\GCComment[1]{\iftodos{{\color{Purple}{{\bf [[GC:}
#1{\bf ]]}}}}\fi}


% Auto-detect multi-letter variable names in math-mode 
% From: http://tex.stackexchange.com/a/218659
% (
\everymath{\it}\everydisplay{\it}
\def\mathcodes#1{\mathcode`#1=\numexpr\mathcode`#1-"7000\relax 
   \ifx#10\else\expandafter\mathcodes\fi}
\mathcodes1234567890
% )

\mathchardef\mhyphen="2D

% Define "foo" as shorthand for {\tt foo} (
\catcode`"=\active
\def"#1"{\texttt{#1}}
% )


\newcommand\id{\mathit{id}}
\newcommand\stor[1]{{[#1]}}
\newcommand\mm[1]{\textsf{#1}}
\newcommand\acyclic{\mathit{acyclic}}
\newcommand\irreflexive{\mathit{irreflexive}}
\newcommand\equivalence{\mathit{equiv}}
\newcommand\spo{\mathit{strict\mhyphen{}po}}
\newcommand\sthd{sthd}
\newcommand\sloc{sloc}
\newcommand\intransitive{\mathit{intransitive}}
\newcommand\imm{\mathit{imm}}
\newcommand\preexecs[1]{\mathit{pre}(#1)}
\newcommand\pre[1]{{#1}_{\rm pre}}
\newcommand\ncap{\mathbin{\not\hspace{-0.25em}\cap}}
\newcommand\join{\mathbin{;}}
\newcommand\tightjoin{{;}}
\newcommand\comp[1]{\overline{#1}}
\renewcommand\parallel{\mathbin{\textsf{ll}}}
\newcommand\E{\mathbb{E}}
\newcommand\HE{\mathbb{E}_{\rm arch}}
\newcommand\SE{\mathbb{E}_{\rm sw}}
\newcommand\lit{\textbf{\itshape lit}}
\newcommand\litmax{\textbf{\itshape lit}^{\rm min}}
\newcommand\litmaxdet{\textsc{Lit}}
\newcommand\litgen{\textbf{\itshape lit}'}
\newcommand\obs{\mathbf{obs}}
\newcommand\Inst{\mathbb{I}}
\newcommand\Thread{\mathbb{C}}
\newcommand\Prog{\mathbb{P}}
\newcommand\Reg{\mathbb{R}}
\newcommand\Loc{\mathbb{L}}
\newcommand\ALoc{\mathbb{L}_{\rm a}}
\newcommand\NALoc{\mathbb{L}_{\rm na}}
\newcommand\Val{\mathbb{V}}
\newcommand\Expr[1]{\mathbb{A}_{#1}}
\newcommand\Exec{\mathbb{X}}
\newcommand\ExecC{\mathbb{X}_{\rm C11}}
\newcommand\ExecOpenCL{\mathbb{X}_{\rm OpenCL}}
\newcommand\ExecPTX{\mathbb{X}_{\rm PTX}}
\newcommand\ExecH{\mathbb{X}_{\rm H}}
\newcommand\cmdpar{\textsc{cmd}'}
\newcommand\cmdseq{\textsc{cmd}}
\newcommand\sem[1]{{\llbracket #1 \rrbracket}}
\newcommand\cof[1]{\lfloor #1 \rfloor}
%\newcommand\roundsem[1]{\llparenthesis #1 \rrparenthesis}
\newcommand\cand[1]{\sem{#1}_0}
\newcommand\C[1]{\textcolor{blue}{\texttt{#1}}}
\newcommand\Alloystar{AlloyStar}
\newcommand\eqdef{\stackrel{\text{\tiny def}}{=}}
\newcommand\dom{\mathrm{dom}}
\newcommand\ran{\mathrm{ran}}
\renewcommand\gets{:=}
\newcommand\getsany{:\in}
\newcommand\kw[1]{\textbf{#1}}
\newcommand\field[2]{{#2}_{#1}}

\newcommand\Fwg[1]{{\rm F}_{"WG"}(#1)}
\newcommand\Fdv[1]{{\rm F}_{"DV"}(#1)}
\newcommand\Wnarlx[1]{{\rm W}_{\na,\morlx}(#1)}
\newcommand\Wrel[1]{{\rm W}_{\morel}(#1)}
\newcommand\Wsc[1]{{\rm W}_{\mosc}(#1)}
\newcommand\Rna[1]{{\rm R}_{\na}(#1)}
\newcommand\Ra[1]{{\rm R}_{\morlx,\moacq,\mosc}(#1)}
\newcommand\membar[3]{{\rm F}(#1,#2,#3)}

\newcommand\litmustestI[3]{
\begin{tabular}{|l|}
\hline
\multicolumn{1}{|l|}{\texttt{#1}} \\
#2
\multicolumn{1}{|l|}{\texttt{#3}} \\ 
\hline
\end{tabular}}

\newcommand\litmustestIsmall[3]{{\small
\begin{tabular}{|@{\,}l@{\,}|}
\hline
\multicolumn{1}{|@{\,}l@{\,}|}{\texttt{#1}} \\
#2
\multicolumn{1}{|@{\,}l@{\,}|}{\texttt{#3}} \\ 
\hline
\end{tabular}}}

\newcommand\litmustestII[3]{
\begin{tabular}{|l||l|}
\hline
\multicolumn{2}{|c|}{\texttt{#1}} \\
#2
\multicolumn{2}{|c|}{\texttt{#3}} \\ 
\hline
\end{tabular}}

\newcommand\litmustestIIIsmall[3]{
\small\begin{tabular}{@{}|@{~}l@{~~}||@{~~}l@{~~}||@{~~}l@{~}|@{}}
\hline
\multicolumn{3}{@{}|c|@{}}{\texttt{#1}} \\
#2
\multicolumn{3}{@{}|c|@{}}{\texttt{#3}} \\ 
\hline
\end{tabular}}

\newcommand\Q[1]{{\bf Q#1}}
\newcommand\GeneralProg{g}
\newcommand\GeneralExec{\tilde{g}}

% Artifact badge (
\usepackage[firstpage]{draftwatermark}
\SetWatermarkText{\hspace*{8in}\raisebox{4.5in}{\includegraphics[scale=0.1]{aec-badge-popl}}}
\SetWatermarkAngle{0}
% )

\usepackage{flushend}

\begin{document}
\toappear{}

\copyrightdoi{TODO}
\maketitle

\begin{abstract}

A \emph{memory consistency model} (MCM) is the part of a programming
language or computer architecture specification that defines which
values can legally be read from shared memory locations. Because MCMs
take into account various optimisations employed by architectures and
compilers, they are often complex and counterintuitive, which makes
them challenging to design and to understand.

We identify four tasks involved in designing and understanding MCMs:
generating conformance tests, distinguishing two MCMs, checking
compiler optimisations, and checking compiler mappings. We show that
all four tasks are instances of a general constraint-satisfaction
problem to which the solution is either a program or a pair of
programs. Although this problem is intractable for automatic
solvers when phrased over programs directly, we show how to solve
analogous constraints over program \emph{executions}, and then
construct programs that satisfy the original constraints.

Our technique, which is implemented in the Alloy modelling framework,
is illustrated on several software- and architecture-level MCMs, both
axiomatically and operationally defined. We automatically recreate
several known results, often in a simpler form, including:
distinctions between variants of the C11 MCM; a failure of the `SC-DRF
guarantee' in an early C11 draft; that x86 is `multi-copy atomic' and
Power is not; bugs in common C11 compiler optimisations; and bugs in a
compiler mapping from OpenCL to AMD-style GPUs. We also use our
technique to develop and validate a new MCM for NVIDIA GPUs that
supports a natural mapping from OpenCL.

\end{abstract}
\category
{C.1.4}{Processor Architectures}{Parallel processors}
\category
{D.3.4}{Programming Languages}{Compilers}
\category
{F.3.2}{Logics and Meanings of Programs}{Semantics of Programming
Languages}


\keywords 
C/C++,
constraint solving,
graphics processor (GPU),
model checking,
OpenCL,
program synthesis,
shared memory concurrency,
weak memory models

\vspace*{-1mm}

\section{Introduction}
\label{sec:intro}
\label{sec:intro:context}

In the specification of a concurrent programming language or a
parallel architecture, the \emph{memory consistency model} (MCM)
defines which values can legally be read from shared memory
locations~\cite{adve+96}. MCMs have to be general enough to enable
portability, but specific enough to enable efficient
implementations. They must also admit optimisations employed by architectures (such as
store buffering and instruction reordering~\cite{hennessy+12}) and by
compilers (such as common subexpression elimination and constant
propagation~\cite{aho+06}). This profusion of design goals has led to MCMs for
languages (such as C11~\cite{c11} and OpenCL~2.0~\cite{opencl20}), for
CPU architectures (such as x86, ARM, and IBM Power), and for GPU
architectures (such as AMD and NVIDIA), that are complicated and
counterintuitive. In particular, all of these MCMs permit executions
that are not \emph{sequentially consistent} (SC), which means that
they do not correspond to a simple interleaving of concurrent
instructions~\cite{lamport79}. As a result, designing and reasoning
about MCMs is extremely challenging.

Responding to this challenge, researchers have built numerous
automatic tools (see~\S\ref{sec:related}). These typically address the
question of whether a program $P$, executed under an MCM $M$, can
reach the final state $\sigma$. Put another way: can the \emph{litmus
test} $(P,\sigma)$ \emph{pass} under $M$? While useful, there are
several other questions whose answers are valuable for MCM reasoning
and development. Four that have appeared frequently in the literature
are:

\begin{description}
%
\item[\Q1] Which programs can be run to test whether a
compiler or machine conforms to a given MCM?~\cite{darbari+16,
alglave+10}
%
\item[\Q2] Is one MCM more permissive than another? That is, is there
a litmus test that can pass under one but must fail under the
other?~\cite{owens+09, mador-haim+12, mador-haim+10, batty+16,
nienhuis+16, lahav+16, alglave+10}
%
\item[\Q3] Can `strengthening' a program (syntactically) ever enable
additional behaviours?  For instance, can we take a litmus test that
must fail, impose additional sequencing or dependencies between its
instructions (or, in the C11 case, give an atomic operation a stronger
`memory order'~\cite[§7.17.3]{c11}), and thereby allow it to
pass?~\cite{sevcik+08, sevcik11, morisset+13, vafeiadis+15,
chakraborty+16, burckhardt+10}
%
\item[\Q4] Is a given software/architecture compiler mapping correct?
Or is there a litmus test that must fail under the software-level MCM,
but which, when compiled, can pass under the architecture-level
MCM?~\cite{batty+11, batty+12, wickerson+15a, sevcik+11, lustig+14, lustig+15}
%
\end{description}
%
% We note that each work cited above addresses one question in
% \emph{isolation} and, except for \Q1 and Lustig et al.'s
% work~\cite{lustig+14, lustig+15}, provides no automated methods for exploring
% the question.

\subsection{Key Idea 1: Generalising the Question}
\label{sec:intro:contribs}

Our first key idea is the observation that all four of the questions
listed above can be answered, sometimes positively and sometimes negatively, by exhibiting programs $P$ and $Q$ and state
$\sigma$ such that the litmus test $(P,\sigma)$ \emph{must fail} under
a given MCM $M$, $P$ and $Q$ are related by a given binary relation
$\blacktriangleright$ on programs, and $(Q,\sigma)$ \emph{can pass}
under a given MCM $N$. That is, each corresponds to finding inhabitants of
the following set.

\begin{definition}[General problem] 
\label{def:general_problem_programs}
%
$\GeneralProg(M,N,\blacktriangleright) \eqdef {}$
\[
\{(P,Q,\sigma) \mid \stack{\sigma \notin \obs_{M}(P) \wedge 
P\blacktriangleright Q \wedge \sigma \in \obs_{N}(Q)\}}
\] 
where $\obs$ returns the set of final states that can be observed
after executing a given program under a given MCM.\footnote{Elements of $\GeneralProg(M,N,\blacktriangleright)$
can also be seen as counterexamples to 
$\blacktriangleright$ implying observational
refinement~\cite{hoare72}:
$P\blacktriangleright Q \nRightarrow \obs_{N}(Q)\subseteq
\obs_{M}(P)$.}
\end{definition}


Our four questions correspond to the following specialisations of
$\GeneralProg$'s parameters:
%
\begin{description}
%
\item[\Q1] $\GeneralProg(M,\mm{0},\id)$ consists of tests that check conformance to $M$, where $\mm{0}$ is the MCM
that allows all executions and $\id$ is the identity relation;
%
\item[\Q2] $\GeneralProg(M,N,\id)$ consists of tests that cannot pass
under $M$ but can under $N$;
%
\item[\Q3] $\GeneralProg(M,M,\text{`is weaker than'})$ consists of
monotonicity violations in $M$, given a relation that holds when one
program is (syntactically) `weaker than' another; and
%
\item[\Q4] $\GeneralProg(M,N,\text{`compiles to'})$ consists of bugs
in a compiler mapping from $M$ to $N$, given a relation that holds when one program `compiles to' another.
%
\end{description}

\subsection{Key Idea 2: Constraining Executions, not Programs} 
\label{sec:intro:constraining_execs}

Having captured in Def.~\ref{def:general_problem_programs} our four
questions, we now ask whether answers can be generated using an
automatic constraint solver. Unfortunately,
Def.~\ref{def:general_problem_programs} is problematic because showing
that a litmus test must fail under $M$ requires universal
quantification over executions. That is, we must show that there
\emph{exists} a litmus test $(P,\sigma)$ such that \emph{all} possible
executions of $P$ under $M$ do not reach $\sigma$. As Milicevic et al.
observe, `this ``$\exists\forall$'' quantifier pattern is a
particularly difficult instance of higher-order
quantification'~\cite{milicevic+15}, and our pilot experiments
confirmed it to be intractable in practice.

\begin{figure}[t]
\centering
\begin{tikzpicture}[inner sep=0.5mm]
\node(P) at (1,1.2) {\strut $(P,\sigma)$};
\node(Q) at (5,1.2) {\strut $(Q,\sigma)$};
\node(X) at (1,0) {\strut \rlap{$X \notin consistent_M$}\phantom{$X$}};
\node(X') at (0,0) {\strut $X'$};
\node(Y) at (5,0) {\strut \rlap{$Y \in consistent_N$}\phantom{$Y$}};
\node at (5,0) {\strut 
\phantom{$Y \in consistent_N$}};
\node(Y') at (4,0) {\strut $Y'$};
\draw[->] (P) to (X);
\draw[->] (P) to (X');
\draw[->] (Q) to (Y);
\draw[->] (Q) to (Y');
\draw[->] (X) to[auto,bend left=30] node{$\triangleright$} (Y);
\draw[->] (P) to[auto] node{$\blacktriangleright$} (Q);
\end{tikzpicture}\hspace*{1cm}
\caption{Diagram for explaining Key Idea 2}
\label{fig:keyidea2}
\end{figure}

Our second `key idea', then, involves rephrasing the constraints to be
not over \emph{programs}, but over \emph{program executions}, where
they become much cheaper to solve (see
Def.~\ref{def:general_problem_executions}, below), and then to recover
litmus tests from these executions. We explain how this is possible
with reference to Fig.~\ref{fig:keyidea2}.

We start by finding individual executions $X$ and $Y$, such that $X$
is inconsistent under $M$, $Y$ is consistent under $N$, and
$X\triangleright Y$.  From $X$, we construct a litmus test
$(P,\sigma)$ that can behave like $X$, and from $Y$, we construct a
litmus test $(Q,\sigma)$ that can behave like $Y$. The
$\triangleright$ relation between $X$ and $Y$ ensures that these
litmus tests will have the same final state $\sigma$ and that
$P\blacktriangleright Q$. We now have a litmus test that \emph{can
fail} under $M$ and another that \emph{can pass} under $N$. But
$\GeneralProg$ requires a litmus test that \emph{must fail} under $M$,
and perhaps $(P,\sigma)$ can still pass by taking a different
execution $X'$. Example~\ref{ex:lb_without_cd} illustrates how such a
situation can arise.

\begin{Example}
\label{ex:lb_without_cd}
%
The diagram below (left) depicts a C11 execution. Dotted rectangles group
events into threads; ${\rm W}$ and ${\rm R}$ denote write and read
events; $\morel$ and $\moacq$ denote atomic `release' and `acquire'
accesses; $\na$ denotes a non-atomic memory access; $rf$ is the `reads
from' relation; and $sb$ means `sequenced before'. We extract a litmus
test from this executions (below right), by mapping write events to
store instructions, reads to loads, and having the final state enforce
the desired $rf$ relation.
%
\begin{center}
\begin{tikzpicture}[inner sep=1pt, baseline=(a.base)]
\node (a) at (0,0.8) {\evtlbl{$a$}$\evR{\na}{"a"}{1}$};
\node (b) at (0,0) {\evtlbl{$b$}$\evW{\morel}{"x"}{1}$};
\node (c) at (2.2,0.8) {\evtlbl{$c$}$\evR{\moacq}{"x"}{1}$};
\node (d) at (2.2,0) {\evtlbl{$d$}$\evW{\na}{"a"}{1}$};
\draw[edgerf] (b) to [auto,pos=0.7] node {$rf$} (c);
\draw[edgerf] (d) to [auto,pos=0.3] node {$rf$} (a);
\draw[edgesb] (a) to [auto] node {$sb$} (b);
\draw[edgesb] (c) to [auto] node {$sb$} (d);
\node[sthd, fit=(a)(b)] {};
\node[sthd, fit=(c)(d)] {};

\node at (5.4,0.4) {$\litmustestIsmall{int a=0; atomic\_int x=0;}{
\quad\texttt{r0=a; x.store(1,\morel);} \\[-1.6mm]
\quad\rule{31mm}{0.4pt} \\[-2.7mm]
\quad\rule{31mm}{0.4pt} \\
\quad\texttt{r1=x.load(\moacq); a=1;} \\
}{r0==1 \&\& r1==1}$};
\end{tikzpicture}
\end{center}
%
The execution is deemed inconsistent in C11. This is because the
successful release/acquire synchronisation implies that $b$
`happens-before' $c$, hence that $a$ happens-before $d$, and hence
that it is impossible for $a$ to read from $d$. As a result, it is
tempting to use the litmus test derived from this execution for
conformance testing (\Q1). In fact, the litmus test is not useful for
this purpose because it has a data race: the non-atomic store to "a"
goes ahead even if the release/acquire synchronisation on "x"
fails. Racy C11 programs have undefined semantics, so non-conformance
cannot be deduced from the test passing.
\end{Example}

To guard against situations like the one above, we require that $X$ is
also `dead'. Semantically, $X$ is dead if whenever $X$ is
inconsistent, and $(P,\sigma)$ is a `minimal' litmus test constructed
from $X$, then no execution of $P$ that leads to $\sigma$ is allowed
(which implies, in particular, that $P$ is race-free). This ensures
that $P$ not only \emph{can fail}, but \emph{must fail}, as
required. We obtain semantic deadness via a syntactic approximation
($dead_M$), which simply involves a few MCM-specific constraints on
the shape of $X$. For instance, we require that whenever two
non-atomic accesses are prevented from racing by release/acquire
synchronisation (as $a$ and $d$ are in
Example~\ref{ex:lb_without_cd}), one of the accesses must have a
control dependency on the acquire event. That is, if we add a control
dependency edge from $c$ to $d$, then this execution would be in
$dead_M$. This ensures semantic deadness because the resultant litmus
test, now having \texttt{if(r1==1) a=1} rather than just \texttt{a=1},
is now race-free. It therefore becomes a useful conformance test.

% \footnote{This
% recalls (the inverse of) strategies used in non-termination
% proving~\cite{cook+14} and temporal logic~\cite{cook+13} that reduce
% $\exists$-properties to $\forall$-properties.}

Formally, we reduce our general constaint-solving problem to finding
inhabitants of the following set.
%
\begin{definition}[General problem over executions]
\label{def:general_problem_executions}
$\GeneralExec(M,N,\triangleright) \eqdef {}$
\[
\{(X,Y)\in\Exec^2\mid 
\stack{X \notin consistent_{M} \wedge {}\\ X \in dead_{M} \wedge
X \triangleright Y \wedge Y \in consistent_{N}\}.}
\] 
\end{definition}

Analogies to our four specific problems, \Q1--\Q4, can be obtained by
specialising $\GeneralExec$'s parameters like we did in
\S\ref{sec:intro:contribs}.

We remark that deadness ensures the \emph{soundness} of our solving
strategy, but because we obtain semantic deadness via a syntactic
approximation, it may spoil its
\emph{completeness}. Nonetheless, although our technique is
incomplete, we demonstrate in the following subsection that it is
\emph{useful}.

\subsection{Applications}
\label{sec:applications}

We have implemented our technique in the Alloy modelling
framework~\cite{jackson12a}. An Alloy model comprises a set of
classes plus a set of constraints that relate objects and fields in
those classes. If further provided with upper bounds on the
number of objects in each class, Alloy can compile the constraints
down to a SAT query, and then invoke a SAT solver to search for a
satisfying instance.

We have applied our technique to a range of MCMs, including both
software-level and architecture-level MCMs, both CPU and GPU
varieties, and both operationally-defined and
axiomatically-defined. Our results fall into two categories: automatic
recreations of results that have previously been manually generated,
and new results.

\paragraph{Recreated results} We have rediscovered litmus tests that witness:
%
\begin{itemize}
\item the impact of three proposed changes to the C11 axioms
(\S\ref{sec:Q2_c11_sra_simp}, \S\ref{sec:kyndylan},
\S\ref{sec:Q2_c11_simp_orig}) -- and our distinguishing litmus tests
are substantially simpler than the originals in two cases;

\item a violation of the supposedly guaranteed sequentially-consistent
semantics for data-race-free programs (the `SC-DRF
guarantee'~\cite{adve+90}) in a early draft of the C11 standard
(\S\ref{sec:scdrf}) -- similar to that reported by Batty et
al.~\cite{batty+11};

\item that x86 is `multi-copy atomic'~\cite{collier92, sorin+11} but
Power is not (\S\ref{sec:Q2_mca});

\item the C11 MCM behaving non-monotonically, by allowing tests to
pass only if sequencing is added or a
memory order is strengthened (\S\ref{sec:monotonicity}) -- and in the
second case, our litmus test is simpler than that found
manually by Vafeiadis et al.~\cite{vafeiadis+15}; and

\item two bugs in a published compiler mapping from OpenCL to
AMD-style GPUs~\cite{orr+15} (\S\ref{sec:Q4_opencl_amd}) -- one of
which is substantially simpler than the original found by Wickerson et
al.~\cite{wickerson+15a}.

\end{itemize} 

\paragraph{New results} Our main new result
(\S\ref{sec:Q4_opencl_ptx}) concerns the mapping of OpenCL to PTX,
an assembly-like language for NVIDIA GPUs~\cite{nvidia15}. We
first use \Q4 to show that a `natural' OpenCL/PTX compiler mapping is
unsound for an existing formalisation of the PTX MCM by Alglave et
al.~\cite{alglave+15}, but sound for a stronger PTX MCM that we
propose. We then use \Q2 to generate litmus tests that distinguish the
two PTX MCMs, which we use to validate our stronger MCM experimentally
against actual NVIDIA GPUs.

% Our second new result, which concerns mapping C11 to x86, is given
% below as an illustrative example of our technique.

% \begin{figure}
% \centering
% \begin{tikzpicture}[inner sep=1pt]
% \node[anchor=east] at (4.9,0.5) {\textcolor{Red}{\xmark} Inconsistent in \mm{C11}};

% \node[event](a) at (2.2,1.8) 
% {\evtlbl{$a$}$\evW{\mosc}{"x"}{1}$};

% \node[event](b) at (2.2,1) 
% {\evtlbl{$b$}$\evR{\mosc}{"y"}{0}$};

% \node[event](c) at (4,1.8) 
% {\evtlbl{$c$}$\evW{\mosc}{"y"}{1}$};

% \node[event](d) at (4,1) 
% {\evtlbl{$d$}$\evR{\mosc}{"x"}{0}$};

% \node[anchor=east] at (9.8,0.5) {\textcolor{Green}{\cmark} Consistent in \mm{x86}};

% \node[event](a') at (6.7,1.8) 
% {\evtlbl{$a'$}$\evW{"lock"}{"x"}{1}$};

% \node[event](b') at (6.7,1) 
% {\evtlbl{$b'$}$\evR{"lock"}{"y"}{0}$};

% \node[event](c') at (8.8,1.8) 
% {\evtlbl{$c'$}$\evW{"lock"}{"y"}{1}$};

% \node[event](d') at (8.8,1) 
% {\evtlbl{$d'$}$\evR{"lock"}{"x"}{0}$};

% \foreach \i/\j in {a/b, a'/b'}
% \draw[edgesb] ([xshift=8mm]\i.south west) to[auto,swap,pos=0.4]
% node{$sb$} ([xshift=8mm]\j.north west -| \i.south west);

% \foreach \i/\j in {c/d, c'/d'}
% \draw[edgesb] (\i) to[auto,swap,pos=0.4]
% node{$sb$} (\j);

% \draw[edgepi] (a) to[auto, bend right=11, pos=0.83] node{$\pi$} (a');
% \draw[edgepi,overlay] (c) to[auto, bend left=11, pos=0.17] node{$\pi$} (c');
% \draw[edgepi] (b) to[auto, bend right=11, pos=0.83] node{$\pi$} (b');
% \draw[edgepi] (d) to[auto, bend left=11, pos=0.17] node{$\pi$} (d');

% \node[sthd, fit=(a)(b)] {};
% \node[sthd, fit=(c)(d)] {};
% \node[sthd, fit=(a')(b')] {};
% \node[sthd, fit=(c')(d')] {};

% \node[anchor=north west] (l1) at (1.3,4.9) 
% {\small $\litmustestI{atomic\_int x=0,y=0;}{
% \quad\texttt{x.store(1,SC);} \\
% \quad\texttt{r0=y.load(SC);} \\[-1.6mm]
% \quad\rule{20.5mm}{0.4pt} \\[-2.7mm]
% \quad\rule{20.5mm}{0.4pt} \\
% \quad\texttt{y.store(1,SC);} \\
% \quad\texttt{r1=x.load(SC);} \\
% }{r0==0 \&\& r1==0}$};

% \node[anchor=north west] (l2) at (6.6,4.9)
% {\small $\litmustestI{x=0,y=0;}{
% \quad\texttt{LOCK MOV [x],\$1} \\
% \quad\texttt{LOCK MOV EAX,[y]} \\[-1.6mm]
% \quad\rule{23.5mm}{0.4pt} \\[-2.7mm]
% \quad\rule{23.5mm}{0.4pt} \\
% \quad\texttt{LOCK MOV [y],\$1} \\
% \quad\texttt{LOCK MOV EBX,[x]} \\
% }{EAX==0 \&\& EBX==0}$};

% \draw[->](l1) to[auto] node{compiles to} (l2);

% \coordinate (foo1) at (3.1,2.1);
% \coordinate (foo2) at (7.75,2.1);

% \draw[<-](foo1) to[auto, pos=0.6, swap, inner sep=0.5mm]
% node{can behave like} (l1.south -| foo1);

% \draw[<-](foo2) to[auto, pos=0.6, inner sep=0.5mm] node
% {can behave like} (l2.south -| foo2);


% \end{tikzpicture}
% \caption{Miscompilation of C11 (left) to x86 (right)?}
% \label{fig:c11_x86_bug}
% \end{figure}

% \begin{Example}[C11/x86 compiler mapping]

% To check the correctness of a C11/x86 compiler mapping, we first
% create Alloy models that represent the x86 MCM (ported from a ".cat"
% model distributed with Alglave et al.'s \textsf{herd}
% tool~\cite{alglave+14}\footnotemark),
% the C11 MCM (ported from a ".cat" model by Batty et
% al.~\cite{batty+16}) and the mapping itself
% (the $\triangleright_{\mm{C11}/\mm{x86}}$ relation, which follows Batty et
% al.~\cite{batty+11}). We then seek elements of
% $\GeneralExec(\mm{C11},\mm{x86},\triangleright_{\mm{C11}/\mm{x86}})$.

% Alloy finds a solution in less than a second
% (Fig.~\ref{fig:c11_x86_bug}, bottom). Dotted lines group events into
% threads, ${\rm W}$ and ${\rm R}$ denote write and read events, $rf$
% denotes the `reads from' relation, $sb$ means `sequenced before', and
% $\pi$ witnesses the correspondence between source events and compiled
% events. We extract litmus tests from these executions
% (Fig.~\ref{fig:c11_x86_bug}, top), by mapping write events to store
% instructions, reads to loads, and having the final state enforce the
% desired $rf$ relation.

% The C11 execution ($a$,$b$,$c$,$d$) is inconsistent because there is
% no interleaving of the "SC" events from the two threads that allows
% both $b$ and $d$ to read zero. Yet the corresponding x86 execution
% ($a'$,$b'$,$c'$,$d'$) is consistent, at least according to Alglave et
% al.'s formalisation of the x86 MCM, because $sb$ edges between two
% locked instructions are not necessarily respected. In fact, this is a
% bug in Alglave et al.'s formalisation, which we have confirmed with
% the authors and fixed. In the fixed MCM, Alloy found no compiler
% mapping violations involving up to 5 events.
% % 
% \end{Example}
% \footnotetext{\url{http://diy.inria.fr/tst/doc/x86tso.cat}}

\medskip\noindent In summary, we make the following contributions.
%
\begin{enumerate}

\item We show that four frequently-asked questions about MCMs can be
viewed as instances of one general formula ($\GeneralProg$,
Def.~\ref{def:general_problem_programs}).

\item We rephrase the formula to constrain executions rather than
litmus tests ($\GeneralExec$,
Def.~\ref{def:general_problem_executions}), so that it can be
tractably explored using a constraint-solving tool.

\item We implement our approach in Alloy, and use it to automatically
reproduce several results obtained manually in previous work, often
finding simpler examples.

\item We present a new, experimentally-validated MCM for PTX, and an
OpenCL/PTX compiler mapping, and use Alloy to validate the mapping
against the PTX MCM.

% \item We evaluate how our various design decisions affect the time
% Alloy takes to obtain solutions (\S\ref{sec:eval}).

\end{enumerate}

Our supplementary material~\cite{popl17supplementary} contains our
Alloy models and our PTX testing results.

%%% Local Variables:
%%% mode: latex
%%% TeX-master: "paper"
%%% End:

\section{Executions}
\label{sec:executions}

The semantics of a program is a set of \emph{executions}. This section
describes our formalisation of executions, both in general
(\S\ref{sec:form_executions}) and specifically for C11, OpenCL, and
PTX (\S\ref{sec:language-specific-executions}), and then
explains how MCMs decide which executions are allowed
(\S\ref{sec:c11_consistency}).

Program executions are composed of \emph{events}, each representing
the execution of a program instruction. Most existing MCM frameworks
embed several pieces of information within each event, such as the
location it accesses, the thread it belongs to, and the value it reads
and/or writes (e.g.,~\cite{adve+90, alglave+10, batty+11, morisset+13,
sarkar+09}). In our work, events are \emph{pure}, in Needham's
sense~\cite{needham89}: they are given meaning simply by their
membership of, for instance, the `read events' set, or the `same
location' relation. This formulation brings three benefits. First, it
means that we can easily build a hierarchy of executions to unify
several levels of abstraction. For instance, starting with a top-level
`execution' class, we can obtain a class of `C11 executions' just by
adding extra fields such as `the set of events with acquire
semantics'; we need not modify the type of events. Second, it means
that the same events can appear (with different meanings) in two
executions, thus reducing the total number of events needed in our
search space. Third, we avoid the need to define (and therefore, in a
bounded search query, set the number of) locations, threads, and
values.

\paragraph{Notation} Our MCMs are written in Alloy's modelling
language, but in this paper we opt for more conventional set-theoretic
notation. For a binary relation $r$, $\comp{r}$ is its complement,
$r^{-1}$ is its inverse, $r^?$ is its reflexive closure, $r^+$ is its
transitive closure, and $r^*$ is its reflexive, transitive
closure. The $\spo$ predicate holds of binary relations that are
acyclic and transitive, and $\equivalence(r,s)$ holds when $r$ is a
subset of $s^2$, reflexive over $s$, symmetric, and transitive. We
compose an $m$-ary relation $r_1$ with an $n$-ary relation
$r_2$ (where $m,n\ge 1$) via
$r_1\join r_2 \eqdef \{(x_1,\dots, x_{m-1},z_1,\dots, z_{n-1})\mid \exists
y\ldotp (x_1,\dots, x_{m-1}, y)\in r_1 \wedge (y,z_1,\dots, z_{n-1})\in
r_2\}$,
and we lift a set to a subset of the identity relation via
$\stor{s} \eqdef \{(e,e)\mid e\in s\}$. We write $\imm(r)$ for
$r-(r\join r^+)$, and $s_1\ncap s_2$ for $s_1\cap s_2 = \emptyset$.

\begin{figure}[t]
\begin{tabular}{@{}l@{~}l@{~~~~~~~~}l}
$E$ & $\subseteq \E$ & all events in the execution \\
$R$, $W$ & $\subseteq \E$ & events that read (resp. write) a location \\
$F$ & $\subseteq \E$ & fence events \\
$nal$ & $\subseteq \E$ & events that access a non-atomic
location \\
$sb$ & $\subseteq \E^2$ & sequenced-before \\
$ad$, $cd$, $dd$ & $\subseteq \E^2$ & address/control/data dependency \\
$\sthd$, $sloc$ & $\subseteq \E^2$ & same thread, same location \\
$rf$, $co$ & $\subseteq \E^2$ & reads-from, coherence order \\
$rfe$ & $\subseteq \E^2$ & $\eqdef rf - sthd$
\end{tabular}
%\par\vspace*{0.5mm}\rule{\linewidth}{0.4pt}
\begin{mathpar}
\labelb{eq:basicexec:ev}~ R \cup W \cup F \cup nal \subseteq E
\and
\labelb{eq:basicexec:fences}~ (R\cup W) \ncap F
\and
\labelb{eq:basicexec:sbthd}~ sb \subseteq \sthd
\and
\labelb{eq:basicexec:sbspo}~ \spo(sb)
\and
\labelb{eq:basicexec:ad}~ ad \subseteq \stor{R}\join
sb\join\stor{R\cup W}
\and
\labelb{eq:basicexec:dd}~ dd \subseteq \stor{R}\join sb\join\stor{W}
\and
\labelb{eq:basicexec:cd}~ cd \subseteq \stor{R}\join sb
\and
\labelb{eq:basicexec:thdequiv}~ \equivalence(\sthd, E)
\and
\labelb{eq:basicexec:locequiv}~ \equivalence(\sloc, R\cup W)
\and
\labelb{eq:basicexec:locnal}~ nal \join \sloc = nal
\and
\labelb{eq:rfdom}~ rf \subseteq (W \times R) \cap \sloc
\and
\labelb{eq:rfunique}~ rf \join rf^{-1} \subseteq \id
\and
\labelb{eq:cospo}~ \spo(co)
\and
\labelb{eq:cotot}~ co \cup co^{-1} = (W - nal)^2 \cap \sloc - \id 
\end{mathpar}\par\vspace*{-2mm}
\caption{Basic executions, $\Exec$}
\label{fig:basic_exec}
\end{figure}

\subsection{Basic Executions}
\label{sec:form_executions}
Let $\E$ be a set of events. 

\begin{definition}[Basic executions] 
%
The set $\Exec$ of basic executions comprises structures with fourteen
fields, governed by well-formed\-ness constraints
(Fig.~\ref{fig:basic_exec}). We write $f_X$ for the field $f$ of
execution $X$, omitting the subscript when it is clear from the
context. The constraints can be understood as follows. The subsets
$R$, $W$, $F$, and $nal$ are all drawn from the events $E$ that appear
in the execution~\refb{eq:basicexec:ev}. In particular,
compound read-modify-write (RMW) events belong to \emph{both} $R$ and $W$.
Fences are distinct from reads and writes~\refb{eq:basicexec:fences}.
\emph{Sequenced before} is an intra-thread strict partial
order~\refb{eq:basicexec:sbthd}~\refb{eq:basicexec:sbspo}. (We allow
$sb$ within a thread to be partial because in C-like languages, the
evaluation order of certain components, such as the operands of the
"+"-operator, is not specified.) Address dependencies are either
read-to-read or read-to-write~\refb{eq:basicexec:ad}, data
dependencies are read-to-write~\refb{eq:basicexec:dd}, and control
dependencies originate from reads~\refb{eq:basicexec:cd}. The $\sthd$
relation forms an equivalence among all
events~\refb{eq:basicexec:thdequiv}, while $\sloc$ forms an
equivalence among reads and writes~\refb{eq:basicexec:locequiv}. We
allow a distinction between `atomic' and `non-atomic' locations; MCMs
that do not make this distinction (such as architecture-level MCMs) simply constrain the set of
non-atomic locations to be empty. The $nal$ set consists only of
complete $\sloc$-classes~\refb{eq:basicexec:locnal}. A relation $rf$
is a well-formed \emph{reads-from} if it relates writes to reads at
the same location~\refb{eq:rfdom} and is injective~\refb{eq:rfunique}.
The inter-thread reads-from ($rfe$) is derived from $rf$. A relation
$co$ is a well-formed \emph{coherence order} if when restricted to
writes on a single atomic location it forms a strict total order; that
is, it is acyclic and transitive~\refb{eq:cospo}, and it relates every
pair of distinct writes to the same atomic location~\refb{eq:cotot}.
\end{definition}

\begin{remark}
%
We emphasise that elements of $\E$ have no intrinsic structure, only
identity. Nevertheless, when drawing executions, we tag events with
their type: ${\rm R}$ for elements of $R-W$, ${\rm W}$ for elements of
$W-R$, and ${\rm C}$ (for `compound') for elements of $R\cap W$. We
indicate $sthd$ equivalence classes with dotted rectangles, and $sloc$
equivalence classes using named representatives, e.g. "x" and "y". We
also tag events with the values read/written, but this is purely for
readability.
%
\end{remark}

\begin{remark}[Initial writes] %
\label{rem:initial_writes} %
Like some prior MCM formalisations~\cite{sezgin04,mador-haim+12}, but
unlike most (e.g.~\cite{batty+11, alglave+13, alglave+10, lahav+16}),
our executions exclude initial writes. We found that Alloy's
performance degrades rapidly as the upper bound on $\lvert \E \rvert$
increases; by avoiding initial writes, this bound can be
lowered. Removing initial writes makes $rf^{-1}$ a \emph{partial}
function, which complicates the definition of \emph{from-read}, as
described below. %
\end{remark}
%
\begin{definition}
\label{def:fromread}
\emph{From-read} relates each read to all of the writes that are
$co$-later than the write that the read observed~\cite{ahamad+93}:
%
\[
fr \eqdef ((rf^{-1} \join co) \cup fr_{\rm init}) - \id
\]
%
where
$fr_{\rm init} \eqdef (\stor{R} - (rf^{-1} \join rf)) \join \sloc
\join \stor{W}$.
In the absence of initial writes, $fr_{\rm init}$ handles reads that observe the initial value.
%
\end{definition}
%
\subsection{Language-Specific Executions} 
\label{sec:language-specific-executions}

We can obtain executions for various languages as subclasses of
$\Exec$.

\begin{figure}[t]
\begin{tabular}{@{}l@{~}l@{~~~~}l}
$A$   & $\subseteq\E$ & atomic events \\
$acq$, $rel$ & $\subseteq\E$ & events that have acquire (resp. release) semantics \\
$sc$ & $\subseteq\E$ & events that have SC semantics
\end{tabular}
%\par\vspace*{0.5mm}\rule{\linewidth}{0.4pt}
\begin{mathpar}
\labelb{eq:c11exec:a}~ acq \cup rel \cup sc \cup (R\cap W) \cup F \subseteq A \subseteq E
\and
\labelb{eq:c11exec:acq}~ R \cap sc \subseteq acq \subseteq
R \cup F
\and
\labelb{eq:c11exec:rel}~ W \cap sc \subseteq rel \subseteq W \cup F
\and
\labelb{eq:c11exec:scf}~ F \cap sc \subseteq acq \cap rel
\and
\labelb{eq:c11exec:anal}~R-A \subseteq nal \subseteq E-A
\end{mathpar}
\par\vspace*{-2mm}
\caption{C11 executions, $\ExecC$ (extending $\Exec$)}
\label{fig:c11_exec}
\end{figure}

\begin{definition}[C11 executions] C11 executions ($\ExecC$) are
structures that inherit all the fields and well-formedness conditions
from basic executions, and add those listed in
Fig.~\ref{fig:c11_exec}. The new fields originate from the `memory
orders' that are attached to atomic operations in
C11~\cite[§7.17.3]{c11}. Acquire events, release events, SC events,
RMWs, and fences are all atomic~\refb{eq:c11exec:a}. Atomic events
that are neither acquires nor releases correspond to C11's `relaxed'
memory order. Acquire semantics
is given to \emph{all} SC reads and \emph{only} reads and
fences~\refb{eq:c11exec:acq}. Release semantics is given to \emph{all}
SC writes and \emph{only} writes and fences~\refb{eq:c11exec:rel}. SC
fences have both acquire and release semantics~\refb{eq:c11exec:scf}.
Non-atomic reads access only non-atomic locations, and atomic
operations never access non-atomic locations~\refb{eq:c11exec:anal}.
\end{definition}

\begin{figure}[t]
\begin{tabular}{@{}l@{~}l@{~~~~~~~~}l}
$dv$ & $\subseteq \E$ & events that have whole-device scope \\
$swg$ & $\subseteq \E^2$ & same workgroup \\
\end{tabular}
%\par\vspace*{0.5mm}\rule{\linewidth}{0.4pt}
%\par\vspace*{-5mm}
\begin{mathpar}
\labelb{eq:openclexec:swg}~ 
sthd \subseteq swg
\and
\labelb{eq:openclexec:swgequiv}~ 
\equivalence(swg, E)
\and
\labelb{eq:openclexec:dv}~
dv \subseteq A
\end{mathpar}
\par\vspace*{-2mm}
\caption{OpenCL executions, $\ExecOpenCL$ (extending $\ExecC$)}
\label{fig:opencl_exec}
\end{figure}

\begin{definition}[OpenCL executions] 
%
OpenCL~\cite{opencl20} extends C11 with hierarchical models of
execution and memory that reflect the GPU architectures it was
primarily designed to target. Accordingly, OpenCL executions
($\ExecOpenCL$, Fig.~\ref{fig:opencl_exec}) extend C11 executions
first by partitioning threads into one or more \emph{workgroups} via
the $swg$
equivalence~\refb{eq:openclexec:swg}~\refb{eq:openclexec:swgequiv},
and second by allowing some atomic operations to be tagged as `device
scope'~\refb{eq:openclexec:dv}, which ensures they are visible
throughout the device. Other atomics (i.e., with only `workgroup
scope') can efficiently synchronise threads within a workgroup but
are unsuitable for inter-workgroup synchronisation. We do not
consider OpenCL's local memory in this work, and we restrict our
attention to the single-device case.
%
\end{definition}

\begin{figure}[t]
\begin{tabular}{@{}l@{~}l@{~~~~~~~~}l}
$dv$ & $\subseteq \E$ & events that have whole-device scope \\
$swg$ & $\subseteq \E^2$ & same workgroup (`co-operative thread array') \\
\end{tabular}
%\par\vspace*{0.5mm}\rule{\linewidth}{0.4pt}
\begin{mathpar}
\labelb{eq:ptxexec:swg}~ 
sthd \subseteq swg
\and
\labelb{eq:ptxexec:swgequiv}~ 
\equivalence(swg, E)
\and
\labelb{eq:ptxexec:dvF}~ 
dv \subseteq F
\and
\labelb{eq:ptxexec:nonal}~ 
nal = \emptyset
\and
\labelb{eq:ptxexec:totalsb}~ 
sthd - \id \subseteq sb \cup sb^{-1}
\and
\labelb{eq:ptxexec:rmw}~ 
R \ncap W
\end{mathpar}
\par\vspace*{-2mm}
\caption{PTX executions, $\ExecPTX$ (extending $\Exec$)}
\label{fig:ptx_exec}
\end{figure}

\begin{definition}[PTX executions] 
%
Like OpenCL executions, PTX executions ($\ExecPTX$,
Fig.~\ref{fig:ptx_exec}) gather threads into
groups~\refb{eq:ptxexec:swg}~\refb{eq:ptxexec:swgequiv}. Some PTX
fences ("membar.gl") have whole-device scope~\refb{eq:ptxexec:dvF},
and others ("membar.cta") have only workgroup scope. There are no
non-atomic locations~\refb{eq:ptxexec:nonal}, sequencing is total
within each thread~\refb{eq:ptxexec:totalsb}, and we do not consider
RMWs~\refb{eq:ptxexec:rmw}.
%
\end{definition}

\begin{remark} When drawing C11 executions, we identify the sets $A$,
$acq$, $rel$ and $sc$ by tagging events in $E-A$ with $\na$, those in
$A - acq - rel$ with $\morlx$, those in $acq - rel - sc$ with
$\moacq$, those in $rel - acq - sc$ with $\morel$, those in
$acq \cap rel - sc$ with $\moar$, and those in $sc$ with $\mosc$. In
OpenCL or PTX executions, we tag events in $A - dv$ with $\mswg$, and
those in $dv$ with $\msdv$. \end{remark}

\subsection{Consistent, Race-Free, and Allowed Executions}
\label{sec:c11_consistency}

Each MCM $M$ defines sets $consistent_M$ and $racefree_M$ of
executions. (For architecture-level MCMs, which do not define races,
the latter contains all executions.) The sets work together to define
the executions allowed under $M$, as follows.

\begin{definition}[Allowed executions] 
\label{def:allowed_execs}
The executions of a program $P$
that are allowed under an MCM $M$ are
%
\[
\sem{P}_M \eqdef \begin{cases} 
\mathrlap{\cand{P} \cap consistent_M}\quad\quad\quad\quad \\ & \text{if}~\cand{P}\cap
consistent_M \subseteq racefree_M \\
\top & \text{otherwise}
\end{cases}
\]
%
where $\top$ stands for an appropriate universal set. Here, $\cand{P}$
is the set of $P$'s \emph{candidate executions}. These can be thought
of as the executions allowed under an MCM that imposes no constraints,
and are discussed separately (see
Def.~\ref{def:candidate_executions}). The equation above says that the
allowed executions of $P$ are its consistent candidates, unless a
consistent candidate is racy, in which case \emph{any} execution is
allowed. (This is sometimes called `catch-fire'
semantics~\cite{boehm+08}.) \end{definition}

\newcommand\dashboxed[1]{\begin{tikzpicture}[baseline=(a.base)]
\node[anchor=base, draw, dashed, inner sep=1mm, rounded corners](a)
{$#1$};
\end{tikzpicture}}

\begin{figure}[t]
\begin{myFrame}{$consistent_{\mm{C11}}$}
\begin{mathpar}
\labelb{eq:c11c:uniproc}~
\acyclic((hb \cap \sloc) \cup rf \cup co \cup fr)
\and
\labelb{eq:c11c:narf}~ 
rf \join \stor{nal} \subseteq \imm(\stor{W} \join (hb \cap \sloc))
\and
\labelb{eq:c11c:ssimp}~
\acyclic((Fsb^? \join (co \cup fr \cup hb) \join
    sbF^?) \cap sc^2 \dashboxed{{}\cap incl})
\end{mathpar}
\end{myFrame}
\vspace*{-2mm}
\begin{myFrame}{$racefree_{\mm{C11}}$}
\begin{mathpar}
\labelb{eq:c11r:dr}~
cnf - A^2 - \sthd \subseteq hb \cup hb^{-1}
\and
\labelb{eq:c11r:ur}~ 
cnf \cap \sthd \subseteq sb \cup sb^{-1}
\and
\dashboxed{\labelb{eq:c11r:hr}~ 
cnf - incl \subseteq hb \cup hb^{-1}}
\end{mathpar}
\end{myFrame}
\begin{mathpar}
Fsb \eqdef \stor{F}\join sb
\and
sbF \eqdef sb \join \stor{F}
\and
rs \eqdef (sb \cap \sloc)^* \join rf^* 
\and
sw \eqdef \stor{rel} \join Fsb^? \join \stor{W\cap A} \join rs
\join rf \join \stor{R\cap A} \join sbF^? \join \stor{acq}
\and
\dashboxed{incl \eqdef dv^2 \cup swg}
\and
hb \eqdef ((sw\, \dashboxed{{}\cap incl} - \sthd) \cup sb)^+
\and
cnf \eqdef ((W \times W) \cup (R \times W) \cup (W \times R)) \cap \sloc - \id
\end{mathpar}
\caption{The C11 \protect\dashboxed{\mbox{and OpenCL}} MCMs}
\label{fig:c11_predicates}
\end{figure}

The consistency and race-freedom axioms for C11 and OpenCL (minus
`consume' atomics) are defined in Figure~\ref{fig:c11_predicates} and
explained below. We have included some recently-proposed
simplifications. In particular, the simpler release sequence (proposed
by Vafeiadis et al.~\cite{vafeiadis+15}) makes deadness easier to
define (\S\ref{sec:safety}), and omitting the total order `$S$' over
SC events (as proposed by Batty et al.~\cite{batty+16}) avoids having
to iterate over all total orders when showing an execution to be
inconsistent.

\emph{Happens before} ($hb$) edges are created by sequencing and by a
release \emph{synchronising with} ($sw$) an acquire in a different
thread. Synchronisation begins with a release write (or a release
fence that precedes an atomic write) and ends with an acquire read (or
an acquire fence that follows an atomic read) and the read observes
either that write or a member of the write's \emph{release sequence}
($rs$). An event's release sequence comprises all the writes to the
same location that are sequenced after the event, as well as the RMWs
(which may be in another thread) that can be reached from one of those
writes via a chain of $rf$ edges~\cite[\S4.3]{vafeiadis+15}. In
OpenCL, synchronisation only occurs between events with
\emph{inclusive scopes} ($incl$), which means that if the events are
in different workgroups they must both be annotated with `device' scope.
Happens-before edges between events accessing the same location,
together with $rf$, $co$, and $fr$ edges, must not form
cycles~\refb{eq:c11c:uniproc}~\cite[\S5.3]{vafeiadis+15}. A read of a
non-atomic location must observe a write that is still \emph{visible}
($vis$)~\refb{eq:c11c:narf}. The \emph{sequential consistency} (SC)
axiom~\refb{eq:c11c:ssimp} requires, essentially, that the $co$, $hb$
and $fr$ edges between $sc$ events do not form
cycles~\cite[\S3.2]{batty+16} An execution has a \emph{data race}
unless every pair of conflicting ($cnf$) events in different threads,
not both atomic, is in $hb$~\refb{eq:c11r:dr}. It has an
\emph{unsequenced race} unless every pair of conflicting events in the
same thread is in $sb$~\refb{eq:c11r:ur}. An OpenCL execution has a
\emph{heterogeneous race}~\cite{hower+14} unless every pair of
conflicting events with non-inclusive scopes is in
$hb$~\refb{eq:c11r:hr}.

% \begin{remark}
% \label{rem:S}
% The SC axiom above is from a revised version of the C11 MCM
% suggested by Batty et al.~\cite{batty+16}. In the original MCM, the
% consistency of an execution depends on the existence of a suitable
% total order, say $S$, over SC events. We prefer the revised version in
% this work because when we later search for an \emph{inconsistent} C11
% execution, we need not iterate over all $S$ orders.
% \end{remark}

% \begin{remark}
% \label{rem:c11_initial_writes}
% We have made a few departures from Batty et al.'s presentation of
% these axioms~\cite{batty+16}, as necessitated by our omission of
% initial writes (Rem.~\ref{rem:initial_writes}). First, happens-before
% ($hb$) no longer needs to explicitly connect initial writes to other
% events. Second, now that $rf^{-1}$ is no longer a total function, the
% coherence axiom can no longer be phrased as Batty et al. had it
% ($\irreflexive((rf^{-1})^?\join co \join rf^? \join hb)$), and must
% instead be stated using
% from-read~\refb{eq:c11c:coh}.\footnote{Somewhat more obscurely: the
% `S4' axiom in the original C11 MCM~\cite{batty+16} must also be
% updated to account separately for initial reads. That is, we require
% not just
% $\irreflexive((S - (co\join S)) \join rf^{-1} \join (hb \cap \sloc)
% \join [W])$,
% but now
% $\irreflexive((S - (co\join S)) \join (\stor{R} - (rf^{-1} \join rf))
% \join fr)$ as well.}
% \end{remark}


%%% Local Variables:
%%% mode: latex
%%% TeX-master: "paper"
%%% End:

\section{Obtaining Litmus Tests}
\label{sec:language}

If we find executions that solve $\GeneralExec$
(Def.~\ref{def:general_problem_executions}), we need to `lift' these
executions up to the level of programs in order to obtain a solution
to the original problem, $\GeneralProg$
(Def.~\ref{def:general_problem_programs}).

We begin this section by defining a language for these generated
programs (\S\ref{sec:programming_language}). The language is designed
to be sufficiently expressive that for any discovered execution $X$,
there exists a program in the language that can behave like $X$. In
particular, the program must be able to create arbitrary sequencing
patterns and dependencies. Beyond this, we keep the language as small
as possible to keep code generation simple. The language is also
generic, so that it can be used to generate both assembly and
high-level language tests.

We go on to define a function that obtains litmus tests from
executions (\S\ref{sec:lits}), and show that, under an additional
assumption about executions, the function is total
(\S\ref{sec:nfreedom}). We then define our deadness constraint on
executions (\S\ref{sec:safety}), and conclude with a discussion of
some of the practical aspects of generating litmus tests (\S\ref{sec:practical}).

\subsection{Programming Language}
\label{sec:programming_language}

\begin{figure}[t]
\[
\newcommand\myskip{1.5mm}
\begin{array}{@{}r@{~~}r@{~}l@{}}
\Expr{s} & ::= & 
         s \mid \Expr{s} + 0 \times \Reg 
\\[\myskip]
\Inst     & ::= & 
         \mathbf{st}(\Expr{\Loc},\Expr{\Val}) 
  \mid   \mathbf{ld}(\Reg,\Expr{\Loc}) 
  \mid \mathbf{cas}(\Expr{\Loc},\Val,\Expr{\Val}) 
  \mid   \mathbf{fe}
\\[\myskip]
\Thread     & ::= & 
     \Inst^\ell
\mid \Thread +^\ell \Thread
\mid \Thread \join^\ell \Thread 
\mid \mathbf{if}^\ell~\Reg=\Val~\mathbf{then}~\Thread
\\[\myskip]
\Prog    & ::= & 
         \Thread \parallel\dots\parallel \Thread
\end{array}
\]
\caption{A programming language}
\label{fig:pl}
\end{figure}

The syntax of our generic programming language is defined in
Fig.~\ref{fig:pl}. We postulate a set $\Val$ of values (containing
zero), a set $\Reg$ of registers, and a set $\Loc$ of (shared)
locations, which is subdivided into atomic ($\ALoc$) and non-atomic
($\NALoc$) locations. Every register and location is implicitly
initialised to zero. Let $\Expr{s}$ denote a language of
\emph{expressions} over a set $s$. Observe that since the only
construction is the addition of a register multiplied by zero, these
expressions only evaluate to elements of $s$. We use them to create
artificial dependencies (i.e. \emph{syntactic} but not \emph{semantic}
dependencies). The set $\Inst$ of \emph{instructions} includes stores,
loads, compare-and-swaps (CAS's), and fences. The
$\mathbf{cas}(x,v,v')$ instruction compares $x$ to $v$ (and if the
comparison succeeds, sets $x$ to $v'$), returning the observed value
of $x$ in either case. Observe that, artificial dependencies
notwithstanding, stores and CAS's write only constant values to
locations. Let $\Thread$ denote a set of \emph{components}, each
formed from sequenced ($\join$) or unsequenced ($+$) composition, or
one-armed conditionals that test for a register having a constant
value. We have no need for loops because the executions we generate
are finite. Each component has a globally-unique label, $\ell$. Let
$\Prog$ be the set of \emph{programs}, each a parallel composition of
components. The top-level components in a program are called
\emph{threads}.

Some languages attach extra information to instructions, such as their
atomicity and memory order (in C11), their memory scope (in OpenCL),
or whether they are `locked' (in x86). Accordingly, when using the
generic language in Fig.~\ref{fig:pl} to represent one of these
languages, we introduce an additional function that stores extra
information for each instruction label.

We ensure that the programs we generate are \emph{well-formed} -- this
is necessary for ensuring that they provide valid solutions to
$\GeneralProg$.

\begin{definition}[Well-formed programs]
\label{def:wf_prog} 
%
A program is well-formed if: (1) different stores/CAS's to the same
location store different values, (2) each register is written at most
once, (3) stores/CAS's never store zero, (4) if-statements never test
for zero, and (5) whenever an if-statement tests register $r$, there
is a load/CAS into $r$ sequenced earlier in the same thread.
%
\end{definition}

\subsection{From Executions to Litmus Tests}
\label{sec:lits}

Our strategy for solving $\GeneralProg$, as outlined in \S\ref{sec:intro:constraining_execs}, is summarised by the
following `proof rule':
%
\begin{equation}
\label{eq:proof_rule}
\raisebox{1.5mm}{\small\begin{tabular}{@{}l@{~~}c@{}}
Step 1: & $(X,Y)\in \GeneralExec(M,N,\triangleright)$
\\
Step 2: & 
$(P,\sigma) \in \litmax(X) ~~~~ (Q,\sigma') \in \lit(Y) ~~~
\sigma=\sigma' ~~~ P \blacktriangleright Q$
\\ \midrule
\rlap{Result:} & $(P,Q,\sigma)\in \GeneralProg(M,N,\blacktriangleright)$
\end{tabular}}
\end{equation}
%
The purpose of this subsection is to define the $\lit$ and $\litmax$
functions. We begin by defining a more general constraint, $\litgen$.


\begin{definition} 
\label{def:lit}
%
The predicate $\litgen(X, P, \sigma, disabled, failures)$ serves to
connect executions $X$ with litmus tests $(P,\sigma)$. The $disabled$
argument is a subset of $P$'s components, which we interpret as those
that do not actually execute when creating the execution $X$ (because
they are guarded by an if-statement whose test failed). The $failures$
argument is a subset of $P$'s CAS instructions, which we interpret as
those that fail to carry out the `swap' when creating execution $X$.
(Characterising executions by which instructions fail and which are
disabled is only sensible because of our restriction to loop-free
programs.) The predicate $\litgen(X,P,\sigma,disabled,failures)$ holds
whenever there exists a bijection $\mu$ between $P$'s enabled
instructions and $X$'s events such that the following conditions all
hold (in which we abbreviate $\mu(i)$ as $\mu_i$):
%
\begin{description}

\item[Conditionals] For an if-statement with condition `$r=v$', the
body is enabled iff both the if-statement is enabled and
$\sigma(r) = v$.

\item[Disabled loads] If a load/CAS into register $r$ is disabled,
then $\sigma(r)=0$.

\item[Unguarded components] Any component not guarded by an
if-statement is enabled.

\item[CAS failures] For every enabled CAS $i$ in $P$: $i$ is in
$failures$ iff there is no $j$ with
$(\mu_j,\mu_i)\in{rf}$ that writes the value $i$
expects.

\item[Instruction types] For every enabled instruction $i$ in $P$: $i$
is a store iff $\mu_i\in {W}-{R}$; $i$ is a load or
a failed CAS iff $\mu_i\in {R}-{W}$; $i$ is a
successful CAS iff $\mu_i\in {R}\cap {W}$; and $i$
is a fence iff $\mu_i\in {F}$.

\item[Non-atomic locations] If $i$ is an enabled non-fence instruction
in $P$ then $\mu_i\in {nal}$ iff $i$'s location is in
$\NALoc$.

\item[Threading, locations, and dependencies] For every enabled
instructions $i$ and $j$ in $P$: $(\mu_i,\mu_j)\in {sthd}$ iff $i$ and
$j$ are in the same thread in $P$; $(\mu_i,\mu_j)\in {sloc}$ iff $i$
and $j$ both access the same location; $(\mu_i,\mu_j)\in {cd}$ iff
there is an enabled if-statement, say $T$, such that $i$ is sequenced
before $T$, $i$ writes to the register $T$ tests, and $j$ is in $T$'s
body; $(\mu_i,\mu_j)\in {ad}$ iff the expression for $j$'s location
depends (syntactically) on the register written by $i$; and
$(\mu_i,\mu_j)\in {dd}$ iff $j$ writes an expression that depends
(syntactically) on the register $i$ writes.

\item[Sequenced composition] For every enabled $\join$-operator in
$P$: if $i$ and $j$ are enabled instructions in the left and right
operands (respectively), then $(\mu_i,\mu_j)\in {sb}$.

\item[Unsequenced composition] For every enabled $+$-operator in $P$:
if $i$ and $j$ are enabled instructions in the left and right operands
(respectively), then
$\{(\mu_i,\mu_j), (\mu_j,\mu_i)\} \ncap {sb}$.

\item[Final registers] For every enabled load/CAS $j$ in $P$ on
location $x$ into register $r$: either (1) there exists an enabled
store/CAS $i$ that writes $v$ to $x$, $(\mu_i,\mu_j)\in {rf}$ and
$\sigma(r) = v$, or (2) $\sigma(r) = 0$ and $\mu_j$ is not in the
range of ${rf}$.

\item[Final locations] For every location $x$: either (1) there is an
enabled store/CAS $i$ that writes $v$ to $x$, $\mu_i$ has no
successor in ${co}$, and $\sigma(x) = v$, or (2) $x$ is never written
and $\sigma(x) = 0$.

\end{description}
\end{definition}

\begin{definition}[Obtaining litmus tests]
%
Let $\lit(X)$ be the set of all litmus tests $(P,\sigma)$ for
which $\litgen(X,P,\sigma,disabled,failures)$ holds for some
instantiation of $disabled$ and $failures$.
%
\end{definition}

\begin{definition}[Obtaining minimal litmus tests]
%
Let $\litmax(X)$ be the set of all litmus tests $(P,\sigma)$ for
which $\litgen(X,P,\sigma,\emptyset,\emptyset)$ holds. That is, when
$P$ behaves like $X$, all of its instructions are executed and all of
its CAS's succeed. We say that the litmus test $(P,\sigma)$ is
\emph{minimal} for $X$, or, dually, $X$ is a \emph{maximal}
execution of $P$.
%
\end{definition}

\begin{definition}[Candidate executions]
\label{def:candidate_executions}
%
We define $P$'s candidate executions by inverting $\lit$:
$\cand{P} \eqdef \{X\mid \exists \sigma\ldotp (P,\sigma)\in\lit(X)\}$.
%
\end{definition}

\begin{definition}[Observable final states] We can now formally define the
notion of observation we employed in
Def.~\ref{def:general_problem_programs}: $\obs_M(P)$ is the set of
final states $\sigma$ for which $(P,\sigma) \in \lit(X)$ for some $X \in \sem{P}_M$.
\end{definition}


\subsection{Totality of $\litmax$}
\label{sec:nfreedom}

We now explain why the $\litmax$ function is currently not total (that
is: there exist well-formed executions $X$ for which $\litmax(X)$ is
empty), and how we can impose an additional restriction on executions
to make it become total.

The problem is: our programming language cannot express all of the
sequencing patterns that an execution can capture in a valid $sb$
relation. For instance, it is not possible to write a program that
generates exactly the following $sb$ edges:
%
\begin{equation}
\label{eq:Nshape}
\begin{tikzpicture}[baseline=0.5cm,inner sep=1pt]
\node (a) at (0,0.8) {$\vphantom{Ay}a$};
\node (b) at (1,0.8) {$\vphantom{Ay}b$};
\node (c) at (0,0) {$\vphantom{Ay}c$};
\node (d) at (1,0) {$\vphantom{Ay}d$};
\draw[edgesb] (a) to [auto,swap,pos=0.3] node {$sb$} (c);
\draw[edgesb] (a) to [auto] node {$sb$} (d);
\draw[edgesb] (b) to [auto,pos=0.3] node {$sb$} (d);
\end{tikzpicture}
\end{equation}
%
This is because, armed only with sequenced (`$\join$') and unsequenced
(`$+$') composition (see Fig.~\ref{fig:pl}), it is only possible to
produce $sb$ relations that are in the set of \emph{series--parallel
partial orders}~\cite{mohring89}.\footnote{That said, if we extended our language
to support fork/join parallelism, then the execution
in~\eqref{eq:Nshape} would become possible: "t1=fork($b$); $a$;
t2=fork($c$); join(t1); $d$; join(t2)."} Helpfully, series--parallel
partial orders are characterised exactly by a simple check: that they
do not contain the `N'-shaped subgraph shown
in~\eqref{eq:Nshape}~\cite{valdes+79}. Accordingly, we impose one
further well-formedness constraint on executions:
%
\[
\stack{\nexists a,b,c,d \in
E\ldotp{}\\{}\quad \{(a,c),(a,d),(b,c)\}\in sb
\wedge  \{(a,b),(b,c),(c,d)\}\ncap sb^?.}
\]

Our $\litmax$ function now becomes total (and hence $\lit$ too).
Indeed, it is straightforward to determinise the constraints listed in
Def.~\ref{def:lit} into an algorithm for
\emph{constructing} litmus tests from executions (and we have
implemented this algorithm, in Java).

\subsection{Dead Executions}
\label{sec:safety}

When searching for inconsistent executions, we restrict our search to those that are also
\emph{dead}. 

\begin{definition}[Semantic deadness] The set of executions that are
(semantically) dead under MCM $M$ is given by $semdead_M \eqdef {}$
%
\[
\stack{\{X\in\Exec \mid X \notin consistent_M \Rightarrow \forall P,\sigma,X',\sigma'\ldotp {}\\
\quad ((P,\sigma)\in\litmax(X) \wedge (P,\sigma')\in\lit(X') \wedge X' \in
consistent_M) \\
\quad\quad {} \Rightarrow (X' \in racefree_M \wedge \sigma'\neq\sigma)\}.}
\]
%
That is, for any minimal litmus test $(P,\sigma)$ for $X$,
no consistent candidate execution of $P$ is racy or reaches
$\sigma$. In other words: $(P,\sigma)$ must fail under $M$.
\end{definition}
%


\begin{figure}
\begin{myFrame}{$dead$}
\begin{mathpar}
\labelb{eq:safe:rfcd}~
\mathrm{domain}(cd) \subseteq \mathrm{range}(rf)
\and
\labelb{eq:safe:forcedco}~
imm(co) \join imm(co) \join imm(co^{-1}) \subseteq rf^? \join (sb \join (rf^{-1})^?)^?
\end{mathpar}
\end{myFrame}
\par\vspace*{-2mm}
\begin{myFrame}{$dead_{\rm C11}$}
\begin{mathpar}
\labelb{eq:safe:hw}~
dead
\and
\labelb{eq:safe:ur}~
cnf \cap \sthd \subseteq sb \cup sb^{-1}
\and
\labelb{eq:safe:cnf}~
pdr \subseteq dhb \cup dhb^{-1} \cup (narf\join ssc) \cup (ssc\join narf^{-1})
\end{mathpar}
\end{myFrame}
\begin{mathpar}
pdr \eqdef cnf - A^2
\and
cde \eqdef (rfe \cup cd)^* \join cd % changed from (rf - \sthd) to rfe
\and
drs \eqdef rs - (\stor{R} \join \comp{cde})
\and
dsw \eqdef sw \cap (((Fsb^? \join \stor{rel} \join drs^?) - (cd^{-1} \join \comp{cde}))
\join rf)
\and
dhb \eqdef sb^? \join (dsw \join cd)^*
\and
ssc \eqdef \id \cap cde
\and
narf \eqdef rf \cap nal^2 - hb
\end{mathpar}
\caption{The $dead$ constraint and its specialisation for C11}
\label{fig:c11dead}
\end{figure}

We employ syntactic approximations of semantic deadness.
Figure~\ref{fig:c11dead} shows how this approximation is defined
for any architecture-level MCM that enforces coherence
($dead$), and how it is strengthened to handle the C11 MCM
($dead_{\rm C11}$).

At the architecture level, we need not worry about races, so ensuring
deadness is straightforward. In what follows, let $X$ be an
inconsistent execution, $(P,\sigma)$ be a minimal litmus test of $X$,
and $X'$ be another candidate execution of $P$. First, we require every
event that is the source of a control dependency to read from a
non-initial write~\refb{eq:safe:rfcd}. This ensures that $P$ need not
contain `$\mathbf{if}~"r"=0$'. Such programs are problematic because
we cannot tell whether the condition holds because "r" was set to zero
or because "r" was never assigned. Second, we require that every $co$
edge (except the last) is justified by an $sb$
edge~\refb{eq:safe:forcedco}. This condition ensures that $X'$ cannot
be made consistent simply by inverting one or more of $X$'s $co$
edges. The construction $imm(co)$ obtains event pairs $(e_1,e_2)$ that
are consecutive in $co$; composing this with
`$imm(co)\join imm(co^{-1})$' restricts our attention to those pairs
for which it is possible to take a further $co$ step from $e_2$. The
last $co$ edge does not need justifying with an $sb$ edge because it
is directly observable in $\sigma$.
%
\begin{Example} 
\label{ex:badtest_co}
The basic execution below (left) is inconsistent in any MCM that
imposes coherence (because $co$ is contradicting $sb$), but the litmus
test obtained from it (below right) does not necessarily fail because
its final state can be obtained via a consistent execution that simply
reverses the $co$ edge from $(b,a)$ to $(a,b)$.
%
\begin{center}
\begin{tikzpicture}[inner sep=1pt, baseline=(a.base)]
\node (a) at (0,0.8) {$\evtlbl{$a$}\evW{}{"x"}{2}$};
\node (b) at (0,0) {$\evtlbl{$b$}\evW{}{"x"}{1}$};
\node (c) at (2.0,0.8) {$\evtlbl{$c$}\evW{}{"x"}{3}$};
\node (d) at (2.0,0) {$\evtlbl{$d$}\evR{}{"x"}{3}$};
\draw[edgerf, shift right=0.5mm, inner sep=1mm] (c) to [auto, swap] node {$rf$} (d);
\draw[edgesb, shift left=0.5mm, inner sep=1mm] (a) to [auto] node {$sb$} (b);
\draw[edgesb, shift left=0.5mm, inner sep=1mm] (c) to [auto] node {$sb$} (d);
\draw[edgemo, shift left=0.5mm, inner sep=1mm] (b) to [auto] node {$co$} (a);
\draw[edgemo] (a) to [auto] node {$co$} (c);
\node[sthd, fit=(a)(b)] {};
\node[sthd, fit=(c)(d)] {};
\node at (5.3,0.4) {$\litmustestIsmall{int x=0;}{
\quad\texttt{x=2; x=1;} \\[-1.6mm]
\quad\rule{14.5mm}{0.4pt} \\[-2.7mm]
\quad\rule{14.5mm}{0.4pt} \\
\quad\texttt{x=3; r0=x;} \\
}{x==3 \&\& r0==3}$};
\end{tikzpicture}
\end{center}
\end{Example}
%
At the software level, we must also worry about races. First, we
forbid all unsequenced races~\refb{eq:safe:ur}, because if $X$ does
not have an unsequenced race, then neither will $X'$, because
unsequenced races are not affected by runtime synchronisation
behaviour. Second, condition~\refb{eq:safe:cnf} is concerned with
\emph{potential data races} ($pdr$): events that conflict and are not
both atomic. Although $X$ is inconsistent (which renders any races in
$X$ irrelevant) we worry that $X'$ might be consistent and racy.  So,
we require $pdr$-linked events to be also in the \emph{dependable
happens-before} ($dhb$) relation, or to exhibit a
\emph{self-satisfying cycle} ($ssc$), both of which we explain below.

\paragraph{Dependable happens-before}
This is a restriction of ordinary happens-before ($hb$). Essentially:
if $e_1$ and $e_2$ are in $dhb$ in $X$, and they map to instructions
$i_1$ and $i_2$ respectively in $P$, then if $i_1$ and $i_2$ are
represented by events in $X'$, those events are guaranteed to be
related by happens-before.

\begin{Example}
\label{ex:lb_bad_rs}
The execution below (left) has fixed the shortcoming in
Example~\ref{ex:lb_without_cd} by adding control dependencies, but it
has introduced the C11 \emph{release sequence} as a further
complexity.
%
\begin{center}
\begin{tikzpicture}[inner sep=1pt, baseline=(a.base)]
\node (a) at (0,0.8) {\evtlbl{$a$}$\evR{\na}{"a"}{1}$};
\node (b) at (0,0) {\evtlbl{$b$}$\evW{\morel}{"x"}{1}$};
\node (c) at (0,-0.8) {\evtlbl{$c$}$\evW{\morlx}{"x"}{2}$};
\node (d) at (2,0) {\evtlbl{$d$}$\evR{\moacq}{"x"}{2}$};
\node (e) at (2,-0.8) {\evtlbl{$e$}$\evW{\na}{"a"}{1}$};
\draw[edgerf] (c) to [auto, pos=0.4, swap] node {$rf$} (d);
\draw[edgerf] (e) to [auto, out=160, in=0, swap, pos=0.8] node {$rf$} (a);
\draw[edgesb] (a) to [auto] node {$sb$,$cd$} (b);
\draw[edgesb, shift left=0.5mm,inner sep=1mm] (b) to [auto,pos=0.4] node {$sb$} (c);
\draw[edgemo, shift right=0.5mm,inner sep=1mm] (b) to [auto, swap,pos=0.4] node {$co$} (c);
\draw[edgesb] (d) to [auto] node {$sb$,$cd$} (e);
\node[sthd, fit=(a)(b)(c)] {};
\node[sthd, fit=(d)(e)] {};

\node at (5.3,0.0) {$\litmustestIsmall{int a=0; atomic\_int x=0;}{
\quad\texttt{r0=a;} \\
\quad\texttt{if(r0==1) x.store(1,\morel);} \\
\quad\texttt{x.store(2,\morlx);} \\[-1.6mm]
\quad\rule{36.5mm}{0.4pt} \\[-2.7mm]
\quad\rule{36.5mm}{0.4pt} \\
\quad\texttt{r1=x.load(\moacq);} \\
\quad\texttt{if(r1==2) a=1;} \\
}{r0==1 \&\& r1==1}$};
\end{tikzpicture}
\end{center}
%
Although event $d$ synchronises with $b$, it actually obtains its
value from $c$, in $b$'s release sequence. The execution is not
semantically dead because its litmus test (above right) is racy: if
"r0" is not assigned $1$, then the release store is not executed; this
means that "r1" can read $2$ without synchronisation having occurred;
this leads to a race on "a". By subtracting
$(cd^{-1}\join \comp{cde})$ in the definition of $dsw$, we ensure that
whatever controls $b$ also controls $c$, and this rules out
undesirable executions like the one above.

Moreover, the $(\stor{R}\join \comp{cde})$ in the definition of $drs$
ensures that if $b$ is an RMW, it controls the execution of $c$. The
effect is that $c$ is not executed if the CAS corresponding to the RMW
fails.
%
\end{Example}

% \begin{Example}
% \label{ex:lb_upward_rs}
% The execution below (left) meets the control dependency requirements related discussed in
% Example~\ref{ex:lb_bad_rs}, but still gives rise to a litmus test
% (right) that is racy.
% \begin{center}
% \begin{tikzpicture}[inner sep=1pt, baseline=(a.base)]
% \node (a) at (0,0.8) {\evtlbl{$a$}$\evR{\morlx}{"y"}{1}$};
% \node (b) at (0,0) {\evtlbl{$b$}$\evW{\morlx}{"x"}{2}$};
% \node (c) at (0,-0.8) {\evtlbl{$c$}$\evW{\morel}{"x"}{1}$};
% \node (d) at (2.2,0.8) {\evtlbl{$d$}$\evR{\morlx}{"x"}{1}$};
% \node (e) at (2.2,0) {\evtlbl{$e$}$\evR{\moacq}{"x"}{2}$};
% \node (f) at (2.2,-0.8) {\evtlbl{$f$}$\evW{\na}{"y"}{1}$};
% \draw[edgerf] (c) to [auto, out=20, in=180, pos=0.2, swap] node {$rf$} (d);
% \draw[edgerf] (b) to [auto, swap] node {$rf$} (e);
% \draw[edgerf] (f) to [auto, out=160, in=0, pos=0.2] node {$rf$} (a);
% \draw[edgesb] (a) to [auto] node {$sb$,$cd$} (b);
% \draw[edgesb, shift left=0.5mm, inner sep=1mm] (b) to [auto] node {$sb$} (c);
% \draw[edgemo, shift left=0.5mm, inner sep=1mm] (c) to [auto] node {$co$} (b);
% \draw[edgesb] (d) to [auto] node {$sb$} (e);
% \draw[edgesb] (e) to [auto] node {$sb$,$cd$} (f);
% \node[sthd, fit=(a)(b)(c)] {};
% \node[sthd, fit=(d)(e)(f)] {};

% \node at (5.3,0.0) {$\litmustestIsmall{atomic\_int x=0,y=0;}{
% \quad\texttt{r0=y.load(\morlx);} \\
% \quad\texttt{if(r0==1) x.store(2,\morlx);} \\
% \quad\texttt{x.store(1,\morel);} \\[-1.6mm]
% \quad\rule{36.5mm}{0.4pt} \\[-2.7mm]
% \quad\rule{36.5mm}{0.4pt} \\
% \quad\texttt{r1=x.load(\morlx);} \\
% \quad\texttt{r2=x.load(\moacq);} \\
% \quad\texttt{if(r2==2) y=1;} \\
% }{r0==1 \&\& r1==1}$};
% \end{tikzpicture}
% \end{center}
% %
% The execution's race-freedom, depends on $a$ and $f$ being separated
% by happens-before, which depends on $e$ acquiring from $b$, which is
% in $c$'s release sequence because it is $co$-after $c$. However,
% should the $co$ edge be inverted, $b$ will no longer be in $c$'s
% release sequence. This explains the $((sb \cap \sloc)\join rf^*)$
% conjunct in the definition of $drs$, which provides an alternative
% formulation of the release sequence that does not depend on $co$.
% \JWComment{Cite~\cite{vafeiadis+15} for this.} 
% %
% \end{Example}

\paragraph{Self-satisfying cycles}

An event occurs in a self-satisfying cycle ($ssc$) if it is connected
to itself via a chain of $cd$ and $rf$ edges that ends with a control
dependency. The if-statements that create these dependencies are
constructed such that their bodies are only executed if the desired
$rf$ edges are present (cf.~Def.~\ref{def:lit}).

A potential race between $e_1$ and $e_2$ is deemed dead if both access
a non-atomic location, $e_1$ is observed by $e_2$ but does not happen
before it, and $e_2$ is in a self-satisfying cycle. The reasoning is
as follows. First, note that the execution is inconsistent, because
reads of non-atomic locations cannot observe writes that do not happen
before them (c.f.~\refb{eq:c11c:narf}). Second, the self-satisfying
cycle ensures that $e_2$ reads from $e_1$ in \emph{every} candidate
execution that includes those events. Therefore, every candidate
execution is inconsistent and any data races can be safely ignored. We
illustrate this reasoning in the following example.

\begin{Example}
\label{ex:c11_not_mono}
The execution below (left) is inconsistent because $f$ reads a non-atomic location from a write ($a$) that does
not happen before $f$. It is dead because $f$ is in a self-satisfying
cycle. Its litmus test (below right) is not racy because the
conditionals ensure that the right-hand thread's load of "a" is only
executed if it obtains the value $1$, which means that it reads from
the left-hand thread's store to "a", which means that the execution is
inconsistent and hence that any races can be ignored.
%
\begin{center}
\begin{tikzpicture}[inner sep=1pt, baseline=(a.base)]
\node[event, anchor=west](a) at (0,2.4) 
{\evtlbl{$a$}$\evW{\na}{"a"}{1}$};

\node[event, anchor=west](b) at (0,1.6)
{\evtlbl{$b$}$\evF{\morlx}$};

\node[event, anchor=west](c) at (0,0.8) 
{\evtlbl{$c$}$\evR{\morlx}{"y"}{1}$};

\node[event, anchor=west](d) at (0,0) 
{\evtlbl{$d$}$\evW{\morlx}{"x"}{1}$};

\node[event, anchor=west](e) at (2.1,1.6) 
{\evtlbl{$e$}$\evR{\moacq}{"x"}{1}$};

\node[event, anchor=west](f) at (2.1,0.8) 
{\evtlbl{$f$}$\evR{\na}{"a"}{1}$};

\node[event, anchor=west](g) at (2.1,0) 
{\evtlbl{$g$}$\evW{\morlx}{"y"}{1}$};

\foreach \i/\j in {a/b, b/c}
\draw[edgesb] ([xshift=7mm]\i.south west) to[auto,pos=0.4]
node{$sb$} ([xshift=7mm]\j.north west -| \i.south west);

\foreach \i/\j in {c/d, e/f, f/g}
\draw[edgesb] ([xshift=7mm]\i.south west) to[auto,pos=0.4]
node{$sb$,$cd$} ([xshift=7mm]\j.north west -| \i.south west);

\draw[edgerf] (d) to[auto,pos=0.65, out=20, in=200] node{$rf$} (e);
\draw[edgerf] (g) to[auto, pos=0.3, out=150, in=0] node{$rf$} (c);
\draw[edgerf] (a) to[auto, out=330, in=150, pos=0.45] node{$rf$} (f);

\node[sthd, fit=(a)(b)(c)(d)] {};
\node[sthd, fit=(e)(f)(g)] {};

\node at (6.0,1.2) {$\litmustestIsmall{int a=0; atomic\_int x,y=0;}{
\quad\texttt{a=1; fence(\morlx);} \\
\quad\texttt{r0=y.load(\morlx);} \\
\quad\texttt{if(r0==1) x.store(1,\morlx);} \\[-1.6mm]
\quad\rule{36.5mm}{0.4pt} \\[-2.7mm]
\quad\rule{36.5mm}{0.4pt} \\
\quad\texttt{r1=x.load(\moacq);} \\
\quad\texttt{if(r1==1) r2=a;} \\
\quad\texttt{if(r2==1) y.store(1,\morlx);} \\
}{r0==1 \&\& r1==1 \&\& r2==1}$};
\end{tikzpicture}
\end{center}
%
\end{Example}

\paragraph{Checking deadness} 
The constraints that define $dead_M$ are quite subtle, particularly
for complex MCMs like C11. Fortunately, we can use Alloy to check that
these constraints imply semantic deadness, by seeking elements of
$dead_M - semdead_M$. That is: we search for an execution $X$ that is
deemed dead, but which gives rise to a litmus test $(P,\sigma)$ that
has, among its candidates, a consistent execution $X'$ that is either
racy or reaches $\sigma$. We were able to check that
$dead_M \subseteq semdead_M$ holds for all executions with no
more than 8 events; beyond this, Alloy timed out (after four hours).

\begin{remark} We are using Alloy here to search all candidate
executions of a program, yet in~\S\ref{sec:intro:constraining_execs}
we argued that it is impractical to phrase constraints over programs
because to do so would necessitate an expensive search over all
candidate executions of a program. We emphasise that this is not a
contradiction. The formula to which we objected had a problematic
$\exists\forall$ pattern: `show that \emph{there exists} a program
such that \emph{for all} of its candidate executions, a certain
property holds', whereas satisfying $dead_M - semdead_M$ requires only
existential quantification. \end{remark}

\subsection{Practical Considerations}
\label{sec:practical}

We have found that Step 2 of our proof rule~\eqref{eq:proof_rule} can
still be hard for Alloy to solve, often leading to timeouts,
particularly when $X$ and $Y$ are large. The search space is
constrained quite tightly by the values of $X$ and $Y$, but there are
still many variables involved. One of the degrees of flexibility in
choosing $P$ and $Q$ is in the handling of if-statements. For
instance, both~\refb{eq:if_version1} and~\refb{eq:if_version2} below
give rise to the same executions because of our well-formedness
restrictions on programs (Def.~\ref{def:wf_prog}):
%
\begin{mathpar}
\labelb{eq:if_version1}~
\mathbf{if}~b~\mathbf{then}~(C_1\join C_2)
\and
\labelb{eq:if_version2}~
(\mathbf{if}~b~\mathbf{then}~C_1)\join (\mathbf{if}~b~\mathbf{then}~C_2).
\end{mathpar}

In response to these difficulties, we designed (and implemented in
Java) a \emph{deterministic} algorithm for $\litmax$ (as mentioned in
\S\ref{sec:nfreedom}), called $\litmaxdet$. In particular, it always
chooses option~\refb{eq:if_version2} over
option~\refb{eq:if_version1}. In practice, we find it quicker to
obtain $(P,Q,\sigma)$ by constructing $(P,\sigma) = \litmaxdet(X)$ and
$(Q,\sigma') = \litmaxdet(Y)$, rather than by solving the four
constraints in Step 2 of~\eqref{eq:proof_rule}. This approach
satisfies the third constraint of Step 2 ($\sigma = \sigma'$) because
all of the $\triangleright$ relations that we consider in this work
ensure that $X$ and $Y$ have the same $co$ and $rf$ edges and hence
reach the same final state. It does not, however, guarantee
$P\blacktriangleright Q$. In particular, if a compiler mapping sends
`$"A"$' to `$"B"\join"C"$', then our algorithm would suggest,
unrealistically, that `$\mathbf{if}~b~\mathbf{then}~"A"$' can compile
to
`$(\mathbf{if}~b~\mathbf{then}~"B")\join
(\mathbf{if}~b~\mathbf{then}~"C")$'.  In our experience, the generated
$P$'s and $Q$'s are sufficiently close to being related by
$\blacktriangleright$ that the discrepancy does not matter.

In fact, we do not even prove that \eqref{eq:proof_rule} is guaranteed
to provide a valid solution to $\GeneralProg$, nor that a solution
even necessarily exists. Such a proof would be highly fragile, being
dependent on intricacies of the deadness constraint, which in turn
depends on intricacies of various MCMs, many of which may be revised
in the future. Instead, we follow the `lightweight, automatic'
approach extolled elsewhere in this paper. Besides using Alloy to
check the definition of $dead$ (as described in~\S\ref{sec:safety}),
we also implemented (in Java) a basic MCM simulator to enumerate the
candidate executions of each litmus test we generate, to ensure that
must-fail litmus tests really must fail. We would have preferred to
have used an existing simulator like \textsf{herd}~\cite{alglave+14}
or CppMem~\cite{batty+11}, but we found both tools to be unsuitable
because of restrictions they impose on litmus tests:
\textsf{herd} cannot handle $sb$ being partial within a thread, and
CppMem cannot handle if-statements that test for particular values.




%%% Local Variables:
%%% mode: latex
%%% TeX-master: "paper"
%%% End:

\section{Application: Comparing MCMs (\Q2)}
\label{sec:discrep}

In this section, we use Alloy to generate litmus tests that
distinguish three proposed variants of the C11 MCM
(\S\ref{sec:Q2_c11_sra_simp}, \S\ref{sec:kyndylan},
\S\ref{sec:Q2_c11_simp_orig}). Such distinguishing tests, particularly
the \emph{simplest} distinguishing tests, are difficult to find by
hand, but are very useful for illustrating the proposed changes. We go
on to check two generic properties of MCMs -- multi-copy atomicity
(\S\ref{sec:Q2_mca}) and SC-DRF (\S\ref{sec:scdrf}) -- by encoding the
properties as MCMs themselves and comparing them against software- and
architecture-level MCMs.

\subsection{Strong Release/Acquire Semantics in C11}
\label{sec:Q2_c11_sra_simp}
Lahav et al. have proposed a stronger semantics for release/acquire
atomics~\cite{lahav+16}. For the release/acquire fragment of C11 (no
non-atomics and no relaxed or SC atomics), their semantics amounts to
adding the axiom $\acyclic(sb \cup co \cup rf)$. We used Alloy to
compare their MCM, which we call \mm{C11-SRA}, to \mm{C11} over the
release/acquire fragment.

% Alloy
% demonstrated that the classic `2+2W' litmus test~\cite{?}
% \begin{center}
% \begin{tikzpicture}[inner sep=1pt, baseline=(a.base)]
% \node (a) at (0,0.8) {$\evW{"rel"}{"x"}{1}$};
% \node (b) at (0,0) {$\evW{"rel"}{"y"}{2}$};
% \node (c) at (2.5,0.8) {$\evW{"rel"}{"y"}{1}$};
% \node (d) at (2.5,0) {$\evW{"rel"}{"x"}{2}$};
% \draw[edgemo] (b) to [auto, pos=0.8] node {$co$} (c);
% \draw[edgemo] (d) to [auto, swap, pos=0.8] node {$co$} (a);
% \draw[edgesb] (a) to [auto] node {$sb$} (b);
% \draw[edgesb] (c) to [auto] node {$sb$} (d);
% \node[sthd, fit=(a)(b)] {};
% \node[sthd, fit=(c)(d)] {};
% \node at (5.0,0.4) {$\litmustestIsmall{atomic\_int x=0,y=0;}{
% \quad\texttt{x.store(1,REL);} \\
% \quad\texttt{y.store(2,REL);} \\[-1.6mm]
% \quad\rule{22mm}{0.4pt} \\[-2.7mm]
% \quad\rule{22mm}{0.4pt} \\
% \quad\texttt{y.store(1,REL);} \\
% \quad\texttt{x.store(2,REL);} \\
% }{x==1 \&\& y==1}$};
% \end{tikzpicture}
% \end{center}
% %
% can pass under \mm{C11-Simp} but not under \mm{C11-SRA} -- just as
% Lahav et al. report. We also performed the comparison under the
% additional constraint that the postcondition should not need to refer
% to shared locations, only to registers -- this corresponds to a
% technical requirement in Lahav et al.'s work. The example that Lahav
% et al. provide in this context is a 10-event
% execution~\cite[Fig.~5]{lahav+16}; under the same constraints, Alloy
% found a distinguishing execution with just six events
Lahav et al. provide a 10-event (and 4-location) execution that
distinguishes the MCMs~\cite[Fig.~5]{lahav+16}. Alloy, on the other
hand, found a execution that requires just 6 events (and 2 locations)
and serves the same purpose.\footnote{Lahav et al. impose an additional technical requirement
that postconditions should not need to refer to shared locations (only
to registers), which rules out the even-simpler `2+2W' litmus
test~\cite{sarkar+11}.}
%
\begin{center}
\begin{tikzpicture}[inner sep=1pt, baseline=(a.base)]
\node (a) at (0,0.8) {$\evW{\morel}{"x"}{1}$};
\node (b) at (0,0) {$\evW{\morel}{"y"}{2}$};
\node (c) at (0,-0.8) {$\evR{\moacq}{"y"}{1}$};
\node (d) at (2.5,0.8) {$\evW{\morel}{"y"}{1}$};
\node (e) at (2.5,0) {$\evW{\morel}{"x"}{2}$};
\node (f) at (2.5,-0.8) {$\evR{\moacq}{"x"}{1}$};
\draw[edgemo] (b) to [auto, pos=0.8] node {$co$} (d);
\draw[edgemo] (e) to [auto, swap, pos=0.8] node {$co$} (a);
\draw[edgerf] (a) to [auto, swap, pos=0.7] node {$rf$} (f);
\draw[edgerf] (d) to [auto, pos=0.7] node {$rf$} (c);
\draw[edgesb] (a) to [auto] node {$sb$} (b);
\draw[edgesb] (b) to [auto] node {$sb$} (c);
\draw[edgesb] (d) to [auto] node {$sb$} (e);
\draw[edgesb] (e) to [auto] node {$sb$} (f);
\node[sthd, fit=(a)(b)(c)] {};
\node[sthd, fit=(d)(e)(f)] {};

\node at (5.0,-0.0) {$\litmustestIsmall{atomic\_int x=0,y=0;}{
\quad\texttt{x.store(1,\morel);} \\
\quad\texttt{y.store(2,\morel);} \\
\quad\texttt{r0=y.load(\moacq);} \\[-1.6mm]
\quad\rule{22mm}{0.4pt} \\[-2.7mm]
\quad\rule{22mm}{0.4pt} \\
\quad\texttt{y.store(1,\morel);} \\
\quad\texttt{x.store(2,\morel);} \\
\quad\texttt{r1=x.load(\moacq);} \\
}{r0==1 \&\& r1==1}$};
\end{tikzpicture}
\end{center}

\subsection{Forbidding Reading/Synchronisation Cycles in C11}
\label{sec:kyndylan}

Nienhuis et al.~\cite[Fig.~12]{nienhuis+16} have suggested
strengthening the C11 MCM with the axiom $\acyclic(sw\cup rf)$. Let us
call that MCM \mm{C11-SwRf}. We used Alloy to search for litmus tests
that could witness such a change, and discovered a solution requiring
12 events and 6 threads
%
\begin{equation*}
\small
\begin{tikzpicture}[inner sep=0.5mm, baseline=(a.base)]
\node (a) at (0.0,1.6) {$\evRMW{\morlx}{"y"}{4}{5}$};
\node (b) at (0.0,0.8) {$\evF{\morel}$};
\node (c) at (0.0,0.0) {$\evW{\morlx}{"x"}{1}$};
\node (d) at (2.1,1.6) {$\evRMW{\morel}{"y"}{2}{3}$};
\node (e) at (2.1,0.8) {$\evW{\morlx}{"y"}{4}$};
\node (f) at (2.1,0.0) {$\evRMW{\moacq}{"x"}{1}{2}$};
\node (g) at (4.2,1.6) {$\evRMW{\morel}{"x"}{2}{3}$};
\node (h) at (4.2,0.8) {$\evW{\mosc}{"x"}{4}$};
\node (i) at (4.2,0.0) {$\evRMW{\mosc}{"y"}{1}{2}$};
\node (j) at (6.3,1.6) {$\evRMW{\morel}{"x"}{4}{5}$};
\node (k) at (6.3,0.8) {$\evF{\moar}$};
\node (l) at (6.3,0.0) {$\evW{\morlx}{"y"}{1}$};
\draw[edgesb] (a) to [auto,pos=0.4] node {$sb$} (b);
\draw[edgesb] (b) to [auto,pos=0.4] node {$sb$} (c);
\draw[edgesb, shift left=0.5mm] (d) to [auto,pos=0.4] node {$sb$} (e);
\draw[edgesb, shift right=0.5mm] (g) to [auto,swap,pos=0.4] node {$sb$} (h);
\draw[edgesb] (j) to [auto,pos=0.4] node {$sb$} (k);
\draw[edgesb] (k) to [auto,pos=0.4] node {$sb$} (l);
\draw[edgemo, shift right=0.5mm] (d) to [auto,swap] node {$co$} (e);
\draw[edgemo, shift left=0.5mm] (g) to [auto] node {$co$} (h);
\draw[edgemo,shift left=0.5mm] (e) to [auto] node {$co$} (a);
\draw[edgerf,shift right=0.5mm] (e) to [auto,swap] node {$rf$} (a);
\draw[edgemo,shift right=0.5mm] (c) to [auto,swap, inner sep=1mm] node {$co$} (f);
\draw[edgerf,shift left=0.5mm] (c) to [auto] node {$rf$} (f);
\draw[edgemo,shift right=0.5mm] (f) to [auto,swap,pos=0.2] node {$co$} ([xshift=-5mm]g.south);
\draw[edgerf,shift left=0.5mm] (f) to [auto,pos=0.1] node {$rf$} ([xshift=-5mm]g.south);
\draw[edgemo,shift right=0.5mm] (h) to [auto,swap] node {$co$} (j);
\draw[edgerf,shift left=0.5mm] (h) to [auto] node {$rf$} (j);
\draw[edgemo,shift left=0.5mm] (l) to [auto, inner sep=1mm] node {$co$} (i);
\draw[edgerf,shift right=0.5mm] (l) to [auto,swap] node {$rf$} (i);
\draw[edgemo,shift left=0.5mm] (i) to [auto,pos=0.2] node {$co$} ([xshift=5mm]d.south);
\draw[edgerf,shift right=0.5mm] (i) to [auto,swap,pos=0.1] node {$rf$} ([xshift=5mm]d.south);
\node[sthd_qtight, fit=(a)(b)(c)] {};
\node[sthd_qtight, fit=(d)(e)] {};
\node[sthd_qtight, fit=(f)] {};
\node[sthd_qtight, fit=(g)(h)] {};
\node[sthd_qtight, fit=(i)] {};
\node[sthd_qtight, fit=(j)(k)(l)] {};
\end{tikzpicture}
\end{equation*}
%
that is virtually identical to the one provided by Nienhuis et al., if
a little less symmetrical. We sought smaller solutions, and found none
with fewer than 8 events that could distinguish \mm{C11-SwRf} from
\mm{C11}. For executions with 8 to 11 events, the SAT solver could not
return a result in a reasonable time.

\subsection{Simplifying the SC Axioms in C11}
\label{sec:Q2_c11_simp_orig}
Batty et al. have proposed a change to the C11 consistency axioms that
enables them to be simplified, and also avoids the need for a total
order, $S$, over all SC events~\cite{batty+16}. Having already
incorporated their proposal in our Fig.~\ref{fig:c11_predicates}, let
us call the original MCM \mm{C11-Orig}.

Batty et al. present a litmus test to distinguish \mm{C11}
from \mm{C11-Orig}, which uses 7 instructions across 4
threads~\cite[Example~1]{batty+16}. Alloy, on the other hand, found
one (below) that needs only 5 instructions and 3 threads.
%
\begin{equation*}
\begin{tikzpicture}[inner sep=1pt]
\node[event, anchor=west](a) at (0.2,2) 
{$\evW{\morlx}{"x"}{1}$};

\node[event, anchor=west](b) at (2,2) 
{$\evRMW{\mosc}{"x"}{1}{2}$};

\node[event, anchor=west](c) at (2,1.2) 
{$\evR{\mosc}{"y"}{0}$};

\node[event, anchor=west](d) at (0.2,1.2) 
{$\evW{\mosc}{"y"}{1}$};

\node[event, anchor=west](e) at (0.2,0.4) 
{$\evR{\mosc}{"x"}{1}$};

\foreach \i/\j in {b/c, d/e}
\draw[edgeS] ([xshift=5mm]\i.south west) to[auto,pos=0.4]
node{$S$} ([xshift=5mm]\j.north west -| \i.south west);

\foreach \i/\j in {b/c, d/e}
\draw[edgesb] ([xshift=4mm]\i.south west) to[auto,swap,pos=0.4]
node{$sb$} ([xshift=4mm]\j.north west -| \i.south west);

\draw[edgeS] (c) to[auto,pos=0.5] node{$S$} (d);

\draw[edgemo] ([yshift=.5mm]a.east) to[auto] node{$co$} ([yshift=.5mm]b.west);

\draw[edgerf, overlay] (a.south west) to[auto, bend right=20, pos=0.05] node{$rf$}
(e.north west);

\draw[edgerf] ([yshift=-.5mm]a.east) to[auto, swap] node{$rf$}
([yshift=-.5mm]b.west);

\node[sthd, fit=(a)] {};
\node[sthd, fit=(b)(c)] {};
\node[sthd, fit=(d)(e)] {};

\node at (6.4,1.3) {$\litmustestIsmall{atomic\_int x=0,y=0;}{
\quad\texttt{x.store(1,\morlx);} \\[-1.6mm]
\quad\rule{30.5mm}{0.4pt} \\[-2.7mm]
\quad\rule{30.5mm}{0.4pt} \\
\quad\texttt{r0=x.cas(1,2,\mosc,\morlx);} \\
\quad\texttt{r1=y.load(\mosc);} \\[-1.6mm]
\quad\rule{30.5mm}{0.4pt} \\[-2.7mm]
\quad\rule{30.5mm}{0.4pt} \\
\quad\texttt{y.store(1,\mosc);} \\
\quad\texttt{r2=x.load(\mosc);} \\
}{r0==true\,\&\&\,r1==0\,\&\&\,r2==1}$};
\end{tikzpicture}
\end{equation*}
%
%
% \begin{equation*}
% \litmustestIIIsmall{atomic\_int x,y=0;}{
%    \texttt{x.store(1,RLX);} 
%  & \texttt{r0=x.cas(1,2,SC,RLX);} 
%  & \texttt{y.store(1,SC);} \\
%  & \texttt{r1=y.load(SC);} 
%  & \texttt{r2=x.load(SC);} 
% \\
% }{r0==true \&\& r1==0 \&\& r2==1}
% \end{equation*}
%

\subsection{Multi-copy atomicity}
\label{sec:Q2_mca}
% Cite peter?
The property of \emph{multi-copy atomicity}\footnote{This is also
known as \emph{store atomicity}~\cite{sorin+11} or \emph{write
atomicity}~\cite{adve+96}.} (MCA)~\cite{collier92} ensures that, in
the absence of thread-local reordering, different threads cannot
observe writes in conflicting orders -- i.e., there is a single copy
of the memory that serialises all writes. The canonical MCA violation
is given by the IRIW test~\cite{boehm+08} (below) where thread-local
reordering has been disabled by inducing `preserved program order'
($ppo$) edges, perhaps using dependencies or fences. That the final
reads observe $0$ betrays a disagreement in the order the writes
occurred:
%
\begin{equation}
\label{eq:iriw}
\begin{tikzpicture}[inner sep=0.5mm, baseline=(a.base)]
\node (a) at (0.0,1.6) {$\evW{}{"y"}{1}$};
\node (b) at (2.1,1.6) {$\evR{}{"y"}{1}$};
\node (c) at (2.1,0.8) {$\evR{}{"x"}{0}$};
\node (d) at (4.2,1.6) {$\evR{}{"x"}{1}$};
\node (e) at (4.2,0.8) {$\evR{}{"y"}{0}$};
\node (f) at (6.3,1.6) {$\evW{}{"x"}{1}$};
\draw[edgesb] (b) to [auto,pos=0.4] node {$ppo$} (c);
\draw[edgesb] (d) to [auto,pos=0.4] node {$ppo$} (e);
%\draw[edgesb, shift right=0.5mm] (g) to [auto,swap,pos=0.4] node {$sb$} (h);
\draw[edgerf,shift right=0.5mm] (a) to [auto,swap] node {$rf$} (b);
\draw[edgerf,shift left=0.5mm] (f) to [auto] node {$rf$} (d);
\node[sthd, fit=(a)] {};
\node[sthd, fit=(b)(c)] {};
\node[sthd, fit=(d)(e)] {};
\node[sthd, fit=(f)] {};
\end{tikzpicture}
\end{equation}

\begin{figure}[t]
\begin{myFrame}{$consistent_{\rm MCA}(ppo)$}
\begin{mathpar}
\labelb{eq:mca:uniproc}~
\acyclic((sb \cap \sloc) \cup co)
\and
\labelb{eq:mca:wo}~ 
\acyclic(wo(ppo))
\end{mathpar}
\end{myFrame}
\begin{mathpar}
wo(ppo) \eqdef 
    (((rfe \join ppo \join rfe^{-1}) - \id ) \join co) \cup (rfe \join ppo \join fr_{\rm init})
\end{mathpar}
\caption{Multi-copy atomicity as an MCM}
\label{fig:mca_predicates}
\end{figure}

We formalise MCA -- for the first time in the axiomatic style -- in
Fig.~\ref{fig:mca_predicates}. The model comprises write/write
coherence~\refb{eq:mca:uniproc}~\cite{sorin+11} and the acyclicity of
the \emph{write order} relation, $wo$~\refb{eq:mca:wo}. Write order
captures the intuition that if two reads, say $r_1$ and $r_2$ (with
$(r_1,r_2)\in ppo$), observe two writes, say $w_1$ and $w_2$
respectively, then any write $co$-later than $w_2$ must follow $w_1$
in the single copy of memory. The $wo$ relation is the union of two
cases: in the first, both reads observe write events, and in the
second, one reads from the initial value (reusing our definition of
$fr_{\rm init}$ from Def.~\ref{def:fromread}). Note that MCA is
parameterised by the given model's $ppo$.\footnote{We have an
alternative formulation for happens-before-based models.}

With this formal definition of MCA, we can seek executions allowed by
a given MCM but disallowed by MCA. We tested x86 and Power, and as
expected, Power does not guarantee MCA (Alloy finds a counterexample
similar to \eqref{eq:iriw}) but x86 does.


\subsection{Seeking SC-DRF Violations}
\label{sec:scdrf}

We used Alloy to seek violations of the SC-DRF
guarantee~\cite{adve+90} in an early draft of the C11
MCM~\cite{c++11draft}. The SC-DRF guarantee, in the C11 context, states
that if a program is race-free and contains no non-SC atomic operations,
then it has only SC semantics.

\begin{figure}[t]
\begin{myFrame}{$consistent_{\mm{SC}}$}
\begin{mathpar}
\labelb{eq:sc}~
acyclic(rf \cup co \cup fr \cup sb)
\end{mathpar}
\end{myFrame}
\caption{The \mm{SC} MCM (following Shasha et al.~\cite{shasha+88})}
\label{fig:sc}
\end{figure}

We sought an execution $X$ that is dead (so that its corresponding
litmus test is race-free), uses no non-SC atomics, and is consistent
under the (draft) C11 MCM but inconsistent under \mm{SC}:
\[
\field{X}{A} \subseteq \field{X}{sc}
\wedge X\in dead_{\mm{C11}}
\cap consistent_{\mm{C11-Draft}} - consistent_{\mm{SC}}
\]
where the \mm{SC} MCM is characterised in Fig.~\ref{fig:sc} and
\mm{C11-Draft} has all the axioms listed by Batty et
al.~\cite[Def.~11]{batty+16}, minus their `S4' axiom. Alloy found an
example similar to that reported by Batty et al.~\cite[\S4,
\emph{Sequential consistency for SC atomics}]{batty+11}. The non-SC
execution is consistent under \mm{C11-Draft} because no rule forbids
SC reads to observe initial writes. The `S4' axiom was
added~\cite[\S2.7]{batty+11} to fix exactly this issue.


%%% Local Variables:
%%% mode: latex
%%% TeX-master: "paper"
%%% End:
\section{Application: Checking Monotonicity (\Q3)}
\label{sec:monotonicity}

In this section, we rediscover two monotonicity violations in the C11
MCM, whereby a new behaviour is enabled either by sequentialisation
(\S\ref{sec:Q3_c11_seq}) or by strengthening a memory order
(\S\ref{sec:Q3_c11_mo}).

\begin{figure}
\centering
\begin{myFrame}{$strengthen(X,Y)$}
\begin{mathpar}
\labelb{eq:monot:ev}~
\field{X}{E} = \field{Y}{E}
\and
\labelb{eq:monot:r}~
\field{X}{R} = \field{Y}{R}
\and 
\labelb{eq:monot:w}~
\field{X}{W} = \field{Y}{W}
\and 
\labelb{eq:monot:f}~
\field{X}{F} = \field{Y}{F} 
\and
\labelb{eq:monot:nal}~
\field{X}{nal} = \field{Y}{nal} 
\and
\labelb{eq:monot:sb}~
\field{X}{sb} \subseteq \field{Y}{sb}
\and
\labelb{eq:monot:ad}~
\field{X}{ad} \subseteq \field{Y}{ad}
\and
\labelb{eq:monot:cd}~
\field{X}{cd} \subseteq \field{Y}{cd}
\and
\labelb{eq:monot:dd}~
\field{X}{dd} \subseteq \field{Y}{dd}
\and
\labelb{eq:monot:sthd}~
\field{X}{\sthd} \subseteq \field{Y}{\sthd}
\and
\labelb{eq:monot:sloc}~
\field{X}{\sloc} = \field{Y}{\sloc}
\and
\labelb{eq:monot:rf}~
\field{X}{rf} = \field{Y}{rf}
\and
\labelb{eq:monot:co}~
\field{X}{co} = \field{Y}{co}
\end{mathpar}
\end{myFrame}
\caption{`Strengthening' an execution}
\label{fig:strengthen}
\end{figure}

Checking monotonicity requires the relation defined in
Fig.~\ref{fig:strengthen}, which holds when one execution is
`stronger' than another. There is almost an isomorphism between $X$
and $Y$, except that $Y$ may have extra $sb$ and dependency
edges~\refb{eq:monot:sb}~\refb{eq:monot:ad}~\refb{eq:monot:cd}~\refb{eq:monot:dd},
and it may interleave multiple threads together~\refb{eq:monot:sthd}.

% When Alloy finds $X$ and $Y$ that witness non-monotonicity at the
% execution level ($X$ is inconsistent, $Y$ is consistent, and $Y$ is
% stronger than $X$), we search for programs $P$ and $Q$ that witness
% non-monotonicity at the source-code level. We constrain $P$ and $Q$
% to have $X$ and $Y$ as maximal candidate executions respectively,
% and $Q$ to be stronger than $P$ in the sense that $P$ and $Q$
% contain the same instructions, control dependencies and sequencing
% between instructions in $P$ are preserved in $Q$, and instructions
% in the same thread in $P$ are in the same thread in $Q$.

\subsection{Monotonicity of C11 w.r.t. Sequencing}
\label{sec:Q3_c11_seq}
One way to extend the $strengthen$ relation to the C11 setting is to
add the following constraints:
%
\begin{mathpar}
\field{X}{A} = \field{Y}{A} 
\and
\field{X}{acq} = \field{Y}{acq} 
\and
\field{X}{rel} = \field{Y}{rel} 
\and
\field{X}{sc} = \field{Y}{sc}.
\end{mathpar}
%
With this notion of strengthening, Alloy found a pair of 6-event
executions that witness a monotonicity violation in C11. They are
almost identical to those given by Vafeiadis et
al.~\cite[Fig.~1]{vafeiadis+15}, though slightly less elegant: Alloy
chose one of the writes to be SC when a relaxed write would have
sufficed.

\subsection{Monotonicity of C11 w.r.t. Memory Orders}
\label{sec:Q3_c11_mo}
Another way to extend the $strengthen$ relation is to add
%
\begin{mathpar}
\field{X}{A} = \field{Y}{A} 
\and
\field{X}{acq} \subseteq \field{Y}{acq} 
\and
\field{X}{rel} \subseteq \field{Y}{rel} 
\and
\field{X}{sc} \subseteq \field{Y}{sc}
\and
\field{X}{sb} = \field{Y}{sb}
\and
\field{X}{ad} = \field{Y}{ad}
\and
\field{X}{cd} = \field{Y}{cd}
\and
\field{X}{dd} = \field{Y}{dd}
\and
\field{X}{\sthd} = \field{Y}{\sthd}
\end{mathpar}
%
which allows memory orders to be strengthened but forbids changes
to sequencing or threading. Using this notion of strengthening, Alloy
found a 7-event monotonicity violation in C11: the execution
previously given in Example~\ref{ex:c11_not_mono}. That execution is
inconsistent, but if the relaxed fence is strengthened to a release
fence, it becomes consistent. This example is similar to one due to
Vafeiadis et al.~\cite[\S3, \emph{Strengthening is
Unsound}]{vafeiadis+15}, but it is simpler: where theirs requires 8
events, 4 locations, and 3 threads, our Alloy-generated example
requires only 7 events, 3 locations, and 2 threads.

%%% Local Variables:
%%% mode: latex
%%% TeX-master: "paper"
%%% End:

\section{\!\!\!\!Application: Checking Compiler Mappings (\Q4)}
\label{sec:compilers}

We now report on our use of Alloy to investigate OpenCL
compiler mappings for AMD-style GPUs (\S\ref{sec:Q4_opencl_amd}) and
NVIDIA GPUs (\S\ref{sec:Q4_opencl_ptx}), and a C11 compiler mapping for
Power (\S\ref{sec:Q4_c11_power}).

Checking compiler mappings requires a relation that holds when one
execution `compiles' to another. Of course, real compilers do not
compile on a `per-execution' level, but from a pair of executions
related in this way, we can obtain a pair of programs that are related
by the compiler mapping at the level of program code. Compiler mappings are more complicated than the
strengthenings considered in \S\ref{sec:monotonicity}, because
they may introduce additional events, such as fences. For this
reason, we introduce an additional relation, $\pi$, to connect each
original event to its compiled event(s). (In \S\ref{sec:monotonicity},
$\pi$ is the identity relation.) 

We are able to handle `straight line' mappings like
C11/x86~\cite{batty+11}, OpenCL/AMD (\S\ref{sec:Q4_opencl_amd}) and
OpenCL/PTX (\S\ref{sec:Q4_opencl_ptx}). We can also handle more
complicated mappings like C11/Power~\cite{batty+12, sarkar+12}, which
includes RMWs and introduces control dependencies, but our examples
are limited to the execution level, because at the code level, C11
RMWs map to Power loops, which our loop-free output programming
language cannot express.


\begin{figure}[t]
\begin{myFrame}{$compile(\pi,X,Y)$}
\begin{mathpar}
\labelb{eq:compile:pi1}~
\stor{\field{X}{E}} = \pi \join \pi^{-1}
\and
\labelb{eq:compile:pi2}~
\stor{\field{Y}{E}} \subseteq \pi^{-1} \join \pi
\and
\labelb{eq:compile:sb}~
\pi^{-1} \join \field{X}{sb}^? \join \pi = \field{Y}{sb}^?
\and
\labelb{eq:compile:ad}~
\field{X}{ad} = \pi \join \field{Y}{ad} \join \pi^{-1}
\and
\labelb{eq:compile:dd}~
\field{X}{dd} = \pi \join \field{Y}{dd} \join \pi^{-1}
\and
\labelb{eq:compile:cd}~
\field{X}{cd}\join \pi = \pi \join \field{Y}{cd}
\and
\labelb{eq:compile:sloc}~
\field{X}{sloc} = \pi \join \field{Y}{sloc} \join \pi^{-1}
\and
\labelb{eq:compile:sthd}~
\pi^{-1} \join \field{X}{sthd} \join \pi = \field{Y}{sthd}
\and
\labelb{eq:compile:rf}~
\field{X}{rf} = \pi \join \field{Y}{rf} \join \pi^{-1}
\and
\labelb{eq:compile:co}~
\field{X}{co} = \pi \join \field{Y}{co} \join \pi^{-1}
\end{mathpar}
\end{myFrame}
\caption{Compiling an execution}
\label{fig:compile}
\end{figure}

Execution $X$ maps to execution $Y$ if there exists a $\pi$ for
which $compile(\pi,X,Y)$ holds, as defined in
Fig.~\ref{fig:compile}. Here, $\pi$ is an injective, surjective,
multivalued function from $X$'s events to $Y$'s
events~\refb{eq:compile:pi1}~\refb{eq:compile:pi2}, that preserves
sequencing~\refb{eq:compile:sb},
dependencies~\refb{eq:compile:ad}~\refb{eq:compile:dd}~\refb{eq:compile:cd},
locations~\refb{eq:compile:sloc}, threads~\refb{eq:compile:sthd}, the
reads-from relation~\refb{eq:compile:rf}, and the coherence
order~\refb{eq:compile:co}. The location-preservation and
thread-preservation constraints have different shapes because, for
instance, when a mapping sends source event $e$ to a set
$\{e'_i\}_i$ of target events, it must ensure that \emph{one} $e'_i$
has $e$'s location (the other events will be fences), but that
\emph{every} $e'_i$ is in the same thread.

\subsection{Compiling OpenCL to AMD-Style GPUs}
\label{sec:Q4_opencl_amd}
Orr et al.~\cite{orr+15} describe a compiler mapping from OpenCL to
an AMD-style GPU architecture. Actually, they support OpenCL extended
with `remote-scope promotion', in which a workgroup-scoped event $a$
can synchronise with an event $b$ in a different workgroup if $b$ is
annotated as `remote'. Wickerson et al.~\cite{wickerson+15a} report on
two bugs in this scheme: a failure to implement message-passing, and a
failure of RMW atomicity.

\begin{figure}
\newcommand\ST[1]{\begin{array}{@{}c@{}}#1\end{array}}
\newcommand\LG[2]{\smash{$\scriptsize\left(\ST{#1\\#2}\right)$}}
\centering
\begin{tikzpicture}[inner sep=1pt]

\node[event](a) at (-0.8,2.3) 
{\evtlbl{$a$}$\evRMW{\moar,\mswg}{"x"}{0}{1}$};

\node[event](b) at (0.3,1.3) 
{\evtlbl{$b$}$\evW{\morel,\msdv,"REM"}{"x"}{2}$};


\node[sthd, fit=(a)] (t1) {};
\node[sthd, fit=(b)] (t2) {};
\node[swg, fit=(t1)] {};
\node[swg, fit=(t2)] {};

\draw[edgemo] (b) to[auto, pos=0.4, swap] node{$co$} (a);


\node (rinv) at (4.3,0) 
{\strut\llap{\LG{"x"\mapsto_{\rm vd} 2}{"x"\mapsto_{\rm L} 0}}$\evRemInv$\rlap{\LG{"x"\mapsto_{\rm vd} 2}{"x"\mapsto_{\rm L} 0}}};
\node (st) [above=3mm of rinv]
{\strut\llap{\LG{}{"x"\mapsto_{\rm L} 0}}$\evW{}{"x"}{2}$\rlap{\LG{"x"\mapsto_{\rm vd} 2}{"x"\mapsto_{\rm L} 0}}};
\node (flu) [above=3mm of st] 
{\strut\llap{\LG{}{"x"\mapsto_{\rm L} 0}}$\evFlu$\rlap{\LG{}{"x"\mapsto_{\rm L} 0}}};
\node (lk) [above=3mm of flu] 
{\strut\llap{\LG{}{"x"\mapsto 0}}$\evLk{"x"}$\rlap{\LG{}{"x"\mapsto_{\rm L} 0}}};
\node (uk) [below=3mm of rinv] 
{\strut\llap{\LG{"x"\mapsto_{\rm vd} 2}{"x"\mapsto_{\rm L} 0}}$\evUk{"x"}$\rlap{\LG{"x"\mapsto_{\rm vd} 2}{"x"\mapsto 0}}};
\node (efet) at (0,0) 
{\strut\llap{\LG{}{"x"\mapsto 0}}$\evEFet{"x"}$\rlap{\LG{"x"\mapsto_{\rm vc} 0}{"x"\mapsto 0}}};
\node (inc) [below=3mm of efet] 
{\strut\llap{\LG{"x"\mapsto_{\rm vc} 0}{"x"\mapsto 0}}$\evRMW{}{"x"}{0}{1}$\rlap{\LG{"x"\mapsto_{\rm vd} 1}{"x"\mapsto 0}}};
\node (eflu1) [below=3mm of uk]
{\strut\llap{\LG{"x"\mapsto_{\rm vd} 2}{"x"\mapsto 0}}$\evEFlu{"x"}$\rlap{\LG{"x"\mapsto_{\rm vc} 2}{"x"\mapsto 2}}};
\node (eflu2) [below=3mm of inc]
{\strut\llap{\LG{"x"\mapsto_{\rm vd} 1}{"x"\mapsto 2}}$\evEFlu{"x"}$\rlap{\LG{"x"\mapsto_{\rm vc} 1}{"x"\mapsto 1}}};

\draw[edgethen] (lk) to (flu);
\draw[edgethen] (flu) to (st);
\draw[edgethen] (st) to (rinv);
\draw[edgethen] (rinv) to (uk);
\draw[edgethen] ([xshift=-14mm, yshift=1mm]uk.west) to ([xshift=14mm, yshift=-1mm]efet.east);
\draw[edgethen] (efet) to (inc);
\draw[edgethen] ([xshift=14mm, yshift=-1mm]inc.east) to ([xshift=-14mm, yshift=1mm]eflu1.west);
\draw[edgethen] ([xshift=-14mm]eflu1.west) to ([xshift=14mm]eflu2.east);

\draw[edgemo] (st) to[auto, bend right, swap, pos=0.4] node{$co$} (inc);

\coordinate (wg1n) at ([yshift=1mm]efet.north);
\coordinate (wg1w) at ([xshift=-13mm]inc.west);
\coordinate (wg1e) at ([xshift=13mm]inc.east);
\coordinate (wg1s) at ([yshift=-1mm]eflu2.south);

\coordinate (wg2n) at ([yshift=1mm]lk.north);
\coordinate (wg2w) at ([xshift=-13mm]rinv.west);
\coordinate (wg2e) at ([xshift=13mm]st.east);
\coordinate (wg2s) at ([yshift=-1mm]eflu1.south);

\node[sthd, fit=(wg1n)(wg1w)(wg1e)(wg1s)] (tt1) {};
\node[sthd, fit=(wg2n)(wg2w)(wg2e)(wg2s)] (tt2) {};
\node[swg, fit=(tt1)] {};
\node[swg, fit=(tt2)] {};

\draw[edgepi, overlay] (b) to[out=0, in=210, auto, pos=0.2] 
node{$\pi$} (lk);
\draw[edgepi, overlay] (b) to[out=0, in=175] (flu);
\draw[edgepi, overlay] (b) to[out=0, in=160] (st);
\draw[edgepi, overlay] (b) to[out=0] (rinv);
\draw[edgepi, overlay] (b) to[out=0] (uk);
\draw[edgepi, overlay] (a) to[bend right=40, auto, swap, pos=0.2] node{$\pi$} (inc);


\end{tikzpicture}
\caption{RMW atomicity bug in the OpenCL/AMD mapping}
\label{fig:atomicity_bug}
\end{figure}

We have formalised Orr et al.'s architecture-level MCM and compiler
mapping in Alloy~\cite{popl17supplementary}, following the
formalisation by Wickerson et al., and then used Alloy to search for
bugs -- essentially automating the task that Wickerson et
al. conducted manually. The architecture-level MCM is
\emph{operational}, which means that consistent executions are
obtained constructively, not merely characterised by axioms. This
means that the MCM is more complex to express in Alloy. Essentially,
the MCM involves a single global memory in which entries can be
temporarily locked, several processing elements partitioned into
compute units, and a write-back write-allocate cache per compute unit.

Fig.~\ref{fig:atomicity_bug} depicts the RMW atomicity bug discovered
by Alloy. The top left of the figure shows a 2-event execution that is
inconsistent (and dead) in OpenCL, and the right shows a corresponding
9-event execution that is observable on hardware. (Wickerson et al.'s
bug is similar, but requires an extra `fetch' event.) All events are
grouped by thread (inner dotted rectangles) and by workgroup (outer
dotted rectangles). Events in the architecture-level execution are
totally ordered (thick black arrows). We track the local and global
state before and after each event, writing
$\left(\begin{smallmatrix}\sigma_l\\\sigma_g\end{smallmatrix}\right)$
for local state $\sigma_l$ and global state $\sigma_g$. The global
state comprises global memory entries, some of which are locked
(${\rm L}$). The local state comprises the compute unit's cache
entries, each either valid (${\rm v}$) or invalid (${\rm i}$), and
either clean (${\rm c}$) or dirty (${\rm d}$). The $\pi$ edges show
how the software-level events are mapped to architecture-level events.
The RMW at workgroup scope ($a$) maps to a single RMW, while the
remote ("REM") write to "x" ($b$) is implemented by first flushing all
dirty local cache entries (${\rm Flu}$), then doing the write
(${\rm W}$), then invalidating all entries in all caches
(${\rm InvA}$), all while preventing concurrent access to "x" in the
global memory (${\rm Lk}$, ${\rm Uk}$). Architecture-level executions
also include the actions of the `environment' fetching ($fet$) and
flushing ($flu$) entries to and from global memory, as well as derived
$rf$ and $co$ relations. The undesirable but observable execution, in
which $"x"=1$ in the final state, arises because the mapping fails to
propagate $"x"=2$ to the global memory before releasing the lock on
"x". The fix is to flush immediately after the write.

To make Alloy's search for OpenCL/AMD compiling bugs tractable, we
found it necessary to make several simplifications: we deleted the
QuickRelease buffers~\cite{hechtman+14}; we allowed multiple locations
to be fetched and flushed in a single action (which reduces the total
number of actions required); we hard-coded the system to have exactly
two workgroups with one thread each; and we maximised sharing between
global and local memory entry objects. These changes are not
necessarily semantics-preserving, but any bogus solutions found using
the simplified MCM can be simply tested manually against the full one.


\subsection{Compiling OpenCL to PTX}
\label{sec:Q4_opencl_ptx}

In this subsection, we develop and check a compiler mapping from
OpenCL to PTX. First, using \Q4, we show that a natural mapping is invalid for an existing formalisation of the PTX MCM
(\mm{PTX1})~\cite{alglave+15}, but valid for a stronger model that we
develop (\mm{PTX2}). Then, we use \Q2 to generate litmus
tests that distinguish \mm{PTX2} from \mm{PTX1}, which we use to
confirm that our \mm{PTX2} remains empirically sound for NVIDIA GPUs.

\begin{figure}[t]
\begin{myFrame}{$consistent_{\mm{PTX1}}$}
\begin{mathpar}
\labelb{eq:ptx:uniproc}~
\acyclic((sb \cap sloc - R^2) \cup rf \cup co \cup fr)
\and
\labelb{eq:ptx:cta}~
\acyclic(hb_{\rm wg})
\and
\labelb{eq:ptx:gl}~
\acyclic(hb_{\rm dv})
\end{mathpar}
\end{myFrame}
\begin{mathpar}
rmo \eqdef rfe \cup co \cup fr
\and
f_{\rm dv} \eqdef sb \join \stor{F \cap dv} \join sb
\and
f_{\rm any} \eqdef sb \join \stor{F} \join sb
\and
hb_{\rm wg} \eqdef (rmo \cup f_{\rm any}) \cap swg
\and
hb_{\rm dv} \eqdef rmo \cup f_{\rm dv}
\end{mathpar}
\caption{The \mm{PTX1} MCM}
\label{fig:ptx_predicates}
\end{figure}

Figure~\ref{fig:ptx_predicates} defines the \mm{PTX1} MCM. The model
enforces coherence (but not between reads, \`a la Sparc RMO~\cite{alglave12})~\refb{eq:ptx:uniproc}, allows any fence to restore
SC within a workgroup~\refb{eq:ptx:cta}, and allows
device-scoped fences to restore SC throughout the
device~\refb{eq:ptx:gl}. We omit dependencies because they are
difficult to test: the PTX compiler often removes our artificially-inserted dependencies.

\begin{table}
\centering
\begin{tabular}{@{}l@{~}l@{~}c@{~}l@{}}
\labelb{eq:ptxmap:strlx} & $"store"(\na|\morlx, s)$ & $\rightsquigarrow$ & $"st.cg"$ \\
\labelb{eq:ptxmap:strel} & $"store"(\morel, s)$ & $\rightsquigarrow$ & $"F"_s\join "st.cg"$ \\
\labelb{eq:ptxmap:stsc} & $"store"(\mosc, s)$ & $\rightsquigarrow$ &
$"F"_s\join "st.cg" \join "F"_s$ \\
\labelb{eq:ptxmap:ldna} & $"load"(\na, s)$ & $\rightsquigarrow$ & $"ld.cg"$ \\
\labelb{eq:ptxmap:lda} & $"load"(\morlx|\moacq, s)$ &
$\rightsquigarrow$ & $"ld.cg" \join "F"_s$ \\
\labelb{eq:ptxmap:ldsc} & $"load"(\mosc, s)$ & $\rightsquigarrow$ &
$"F"_s \join "ld.cg" \join "F"_s$ \\
\labelb{eq:ptxmap:f} & $"fence"(s)$ & $\rightsquigarrow$ & $"F"_s$ \\
\end{tabular} 
\hfill \raisebox{4mm}{$\stack{\text{where}\\"F"_{\mswg} \eqdef "membar.cta" \\ "F"_{\msdv} \eqdef "membar.gl"}$}
\caption{OpenCL/PTX compiler mapping, program code level}
\label{tab:opencl_ptx}
\end{table}

Table~\ref{tab:opencl_ptx} defines the OpenCL/PTX mapping we use.
OpenCL stores are mapped to PTX stores
("st.cg")~\refb{eq:ptxmap:strlx}~\refb{eq:ptxmap:strel}~\refb{eq:ptxmap:stsc},
and OpenCL loads are mapped to PTX loads
("ld.cg")~\refb{eq:ptxmap:ldna}~\refb{eq:ptxmap:lda}, with fences
("membar") placed before and/or after the memory access depending on
whether the OpenCL instruction is non-atomic, relaxed, acquire,
release, or SC. PTX fences are required even for relaxed OpenCL loads
because PTX allows consecutive loads from the same address to be
re-ordered (see~\refb{eq:ptx:uniproc}). The PTX fences are scoped to
match the scope $s$ of the OpenCL instruction. In line with prior work
on the PTX MCM~\cite{alglave+15}, we exclude RMWs, local memory, and
multi-GPU interactions.

\begin{figure}[t]
%
\begin{myFrame}{$compile_{\rm OpenCL/PTX}(\pi,X,Y)$}
\begin{mathpar}
\labelb{eq:openclptx:comp}~
compile(\pi,X,Y)
\and
\labelb{eq:openclptx:swg}~
\field{X}{swg} = \pi\tightjoin \field{Y}{swg}\tightjoin \pi^{-1}
\and
\stack{~\\[-2mm]
\labelb{eq:openclptx:Wnarlx}~
\stack{\forall e\in \field{X}{W} - \field{X}{rel}\ldotp {}\\
  \exists e' \in \field{Y}{W} \ldotp {}\\
  \{e\}\tightjoin\pi  = \{e'\}}}
\and \hspace*{-1mm}
% \labelb{eq:openclptx:Wsc}~
% \stack{\forall e\in \Wsc{X}\ldotp {}\\
%   \exists e'_1, e'_3 \in \membar{e}{X}{Y} \ldotp {}\\
%   \exists e'_2 \in \field{Y}{W} \ldotp {}\\ 
%   (e'_1, e'_2) \in imm(\field{Y}{sb}) \wedge {}\\
%   (e'_2, e'_3) \in imm(\field{Y}{sb}) \wedge {}\\
%   \{e\}\tightjoin\pi  = \{e'_1\}{\uplus}\{e'_2\}{\uplus}\{e'_3\}}
% \and
\labelb{eq:openclptx:Wrel}~
\stack{\forall e\in \field{X}{W} \cap \field{X}{rel} - \field{X}{sc} -
\mathrlap{\field{X}{dv}\ldotp} {}\\
  \exists e'_1 \in \field{Y}{F} - \field{Y}{dv} \ldotp %{}\\
  \exists e'_2 \in \field{Y}{W} \ldotp {}\\
  (e'_1, e'_2) \in imm(\field{Y}{sb}) \wedge {}\\
  \{e\}\tightjoin\pi  = \{e'_1\} \uplus \{e'_2\}}
% \and
% \labelb{eq:openclptx:Ra}~
% \stack{\forall e\in \Ra{X}\ldotp {}\\
%   \exists e'_1 \in \field{Y}{R} \ldotp {}\\
%   \exists e'_2 \in \membar{e}{X}{Y} \ldotp {}\\
%   (e'_1, e'_2) \in imm(\field{Y}{sb}) \wedge {}\\
%   \{e\}\tightjoin\pi  = \{e'_1\} \uplus \{e'_2\}}
% \\
% \labelb{eq:openclptx:Rna}~
% \stack{\forall e\in \Rna{X}\ldotp {}\\
%   \exists e' \in \field{Y}{R} \ldotp {}\\
%   \{e\}\tightjoin\pi  = \{e'\}}
% \and
% \labelb{eq:openclptx:F}~
% \stack{\forall e\in \field{X}{F}\ldotp {}\\
%   \exists e' \in \membar{e}{X}{Y} \ldotp {}\\
%   \{e\}\tightjoin\pi  = \{e'\}}
\end{mathpar}
\end{myFrame}

% where \vspace*{-3mm}
% \begin{mathpar}
% \begin{array}{@{}r@{~}c@{~}l@{}}
% \Fwg{X} &\eqdef& \field{X}{F} - \field{X}{dv} \\ 
% \Fdv{X} &\eqdef& \field{X}{F} \cap \field{X}{dv} \\
% \Rna{X} &\eqdef& \field{X}{R} - \field{X}{A} \\
% \Ra{X} &\eqdef& \field{X}{R} \cap \field{X}{A}
% \end{array}
% \hspace*{-1mm}\and
% \begin{array}{@{}r@{~}c@{~}l@{}}
% \Wnarlx{X} &\eqdef& \field{X}{W} - \field{X}{rel} \\
% \Wrel{X} &\eqdef& \field{X}{W} \cap \field{X}{rel} - \field{X}{sc} \\
% \Wsc{X} &\eqdef& \field{X}{W} \cap \field{X}{sc} \\
% \membar{e}{X}{Y} &\eqdef& \stack{\text{$\Fwg{Y}$ if $e\notin
% \field{X}{dv}$} \\ \text{$\Fdv{Y}$ otherwise}}
% \end{array}
% \end{mathpar}

\caption{OpenCL/PTX mapping, execution level (excerpt)}
\label{fig:compile_opencl_ptx}
\end{figure}

Figure~\ref{fig:compile_opencl_ptx} shows how the code-level mappings
in (the first two rows of) Tab.~\ref{tab:opencl_ptx} are encoded at
the execution level. OpenCL workgroups correspond to PTX
workgroups~\refb{eq:openclptx:swg}. An OpenCL write without release
semantics ($e$) maps to a single PTX write
($e'$)~\refb{eq:openclptx:Wnarlx}, while an OpenCL workgroup-scoped
release write ($e$) maps to a workgroup-scoped fence ($e'_1$)
sequenced before a PTX write ($e'_2$)~\refb{eq:openclptx:Wrel}. The
other rows are handled similarly.

\begin{figure}[t]
\centering
\begin{tikzpicture}[inner sep=0.5mm, baseline=(a.base)]
\node (a) at (0.3,2.8) {$\evtlbl{$a$}\evWCL{\morel}{"x"}{1}{\msdv}$};
\node (b) at (2.9,2.8) {$\evtlbl{$b$}\evRCL{\moacq}{"x"}{1}{\msdv}$};
\node (c) at (2.9,2.0) {$\evtlbl{$c$}\evWCL{\morel}{"y"}{1}{\mswg}$};
\node (d) at (5.3,2.8) {$\evtlbl{$d$}\evRCL{\moacq}{"y"}{1}{\mswg}$};
\node (e) at (5.3,2.0) {$\evtlbl{$e$}\evRCL{\moacq}{"x"}{0}{\mswg}$};
\draw[edgesb] (b) to [auto,pos=0.4, swap] node {$sb$} (c);
\draw[edgesb] (d) to [auto,pos=0.5, swap] node {$sb$,$cd$} (e);
\draw[edgerf] (a) to [auto,swap] node {$rf$} (b);
\draw[edgerf] (c) to [auto,pos=0.4,swap] node {$rf$} (d);
\node[sthd, fit=(a)] (t1) {};
\node[sthd, fit=(b)(c)] (t2) {};
\node[sthd, fit=(d)(e)] (t3) {};
\node[swg, fit=(t1)] {};
\node[swg, fit=(t2)(t3)] {};

\node (a') at (1.6,0.8) {$\evtlbl{$a'$}\evF{\msdv}$};
\node (b') at (1.6,0.0) {$\evtlbl{$b'$}\evW{}{"x"}{1}$};
\node (c') at (4.2,0.8) {$\evtlbl{$c'$}\evR{}{"x"}{1}$};
\node (d') at (4.2,0.0) {$\evtlbl{$d'$}\evF{\msdv}$};
\node (e') at (4.2,-0.8) {$\evtlbl{$e'$}\evF{\mswg}$};
\node (f') at (4.2,-1.6) {$\evtlbl{$f'$}\evW{}{"y"}{1}$};
\node (g') at (6.6,0.8) {$\evtlbl{$g'$}\evR{}{"y"}{1}$};
\node (h') at (6.6,0.0) {$\evtlbl{$h'$}\evF{\mswg}$};
\node (i') at (6.6,-0.8) {$\evtlbl{$i'$}\evR{}{"x"}{0}$};
\node (j') at (6.6,-1.6) {$\evtlbl{$j'$}\evF{\mswg}$};
\draw[edgesb] (a') to [auto,pos=0.4] node {$sb$} (b');
\draw[edgesb] (c') to [auto,pos=0.4] node {$sb$} (d');
\draw[edgesb] (d') to [auto,pos=0.4] node {$sb$} (e');
\draw[edgesb] (e') to [auto,pos=0.4] node {$sb$} (f');
\draw[edgesb] (g') to [auto,pos=0.4] node {$sb$} (h');
\draw[edgesb] (g') to [pos=0.4, bend left=55] (i');
\draw[edgesb] (g') to [auto,pos=0.15, bend left=55] node {$cd$} (j');
\draw[edgesb] (h') to [auto,pos=0.4] node {$sb$} (i');
\draw[edgesb] (i') to [auto,pos=0.4] node {$sb$} (j');
\draw[edgerf] (b') to [auto,swap,pos=0.4] node {$rf$} (c');
\draw[edgerf] (f') to [auto,swap,pos=0.56, inner sep=0] node {$rf$} (g');
\node[sthd, fit=(a')(b')] (t1') {};
\node[sthd, fit=(c')(d')(e')(f')] (t2') {};
\node[sthd, fit=(g')(h')(i')(j')] (t3') {};
\node[swg, fit=(t1')] {};
\node[swg, fit=(t2')(t3')] {};

\draw[edgepi] (a) to[out=260, in=180, auto, pos=0.2] node{$\pi$} (a');
\draw[edgepi] (a) to[out=260, in=180] (b');
\draw[edgepi] (b) to[out=290, in=180, auto, pos=0.6] node{$\pi$} (c');
\draw[edgepi] (b) to[out=290, in=180] (d');
\draw[edgepi] (c) to[out=260, in=180] (e');
\draw[edgepi] (c) to[out=260, in=180, auto, pos=0.13, swap] node{$\pi$} (f');
\draw[edgepi] (d) to[out=290, in=180, auto, pos=0.6] node{$\pi$} (g');
\draw[edgepi] (d) to[out=290, in=180] (h');
\draw[edgepi] (e) to[out=260, in=180] (i');
\draw[edgepi] (e) to[out=260, in=180, auto, pos=0.13, swap] node{$\pi$} (j');
\end{tikzpicture}
\caption{A WRC bug in the OpenCL/PTX1 mapping}
\label{fig:opencl_ptx1_bug}
\end{figure}

%\paragraph{A stronger PTX model} 
We used Alloy to check this mapping against \mm{PTX1}, and found an
execution that is disallowed by OpenCL
(Fig.~\ref{fig:opencl_ptx1_bug}, top) but allowed, after compilation,
by \mm{PTX1} (Fig.~\ref{fig:opencl_ptx1_bug}, bottom). Note that the
outer dotted rectangles group threads by workgroup. The crux here is
\emph{cumulative} synchronisation across
scopes~\cite[\S1.7.1]{power206}. The left thread synchronises at
workgroup scope with the middle thread (via $a$ and $b$), which
synchronises at device scope with the right thread (via $c$ and
$d$). If \mm{PTX1} supported cumulative synchronisation across scopes,
the left and right threads would now be synchronised, and the
execution above, in which $e$ observes a stale "y", would be
forbidden, just as it is in OpenCL.

We could fix the mapping for \mm{PTX1} by upgrading all the PTX fences
in Tab.~\ref{tab:opencl_ptx} to device scope. However, because
widely-scoped fences incur high performance overhead on NVIDIA
GPUs~\cite{sorensen+16}, we opt instead to strengthen \mm{PTX1} to
support our mapping, by enforcing cumulative synchronisation
across scopes (obtaining \mm{PTX2}). This entails changing the
definition of $hb_{\rm dv}$ (Fig.~\ref{fig:ptx_predicates}) to
%
\[
hb_{\rm dv} \eqdef rmo \cup (hb_{\rm wg}^* \join f_{\rm dv}
 \join hb_{\rm wg}^*)
\]
%
so that device-scoped fences ($f_{\rm dv}$) can restore SC throughout
the device even if they are preceded or followed by workgroup-scoped
synchronisation ($hb_{\rm wg}$). After this change, Alloy finds no
bugs in our mapping with up to 5 software-level events. It times out
(after four hours) when checking larger executions.
% \TSComment{I currently have this running for 6,18 events, we'll see
% if it finishes!}

%\paragraph{Testing our stronger PTX model}
It remains to show that \mm{PTX2} is sound w.r.t. empirical GPU
testing data. To do this, we use \Q2 to find litmus tests $(P,\sigma)$
that are allowed under \mm{PTX1} but disallowed under \mm{PTX2}. We
wish to find not just a single litmus test (as
in~\S\ref{sec:discrep}), but as many as possible, so we run Alloy
iteratively, each time disallowing the execution shape found
previously, until it is unable to find more. We then check testing
results (or if results do not exist for a given test, we run new
tests) to confirm that $(P,\sigma)$ cannot pass on actual GPUs.

Using this method, Alloy finds all 14 distinguishing 7-event tests
(e.g. WRC~\cite{mador-haim+12}), plus 12 of the distinguishing 8-event
tests (e.g. IRIW~\cite{boehm+08}) before timing out. We are able to
query Alglave et al.'s experimental results~\cite{alglave+15} for 22
of these 26 tests. The rest are single-address tests (which arise
because PTX does not guarantee coherence in general). These we ran on
an NVIDIA GTX Titan using the GPU-litmus tool~\cite{alglave+15}. We
found no behaviours that are allowed by \mm{PTX1}, disallowed by
\mm{PTX2}, and empirically observable on a
GPU~\cite{popl17supplementary}. Thus, subject to the available testing
data, strengthening \mm{PTX1} to \mm{PTX2} is sound, and thus the
natural OpenCL/PTX compiler mapping can be used.

\subsection{Compiling C11 to Power}
\label{sec:Q4_c11_power}

Work in progress by Lahav et al.~\cite{lahav+16b} has uncovered a bug
in the C11/Power mapping previously thought to have been proven sound
by Batty et al.~\cite{batty+12}.\footnote{Concurrent work by
Manerkar et al.~\cite{manerkar+16} has shown the C11/ARMv7
mapping to be similarly flawed.} Before becoming aware of their work, we
had used Alloy to verify the soundness of the mapping for up to 6
software-level events and up to 6 architecture-level
events. Incrementing these bounds any further resulted in intractable
solving times, which explains why the bug, which requires 6
software-level events and 13 architecture-level events, had not
previously been found by Alloy. In order to recreate Lahav et al.'s
result, we modified the C11/Power mapping so as not to place fences at
the start or the end of a thread. Removing these redundant fences allows
the bug to be expressed using just 8 architecture-level events, and
found automatically by Alloy in a reasonable time (see
Tab.~\ref{tab:tasks} in the next section).

%%% Local Variables:
%%% mode: latex
%%% TeX-master: "paper"
%%% End:

%\section{Application: Generic Properties (\Q5, \Q2)}
In this section we use our tool to validate two generic properties of
memory models: SC-DRF and multi-copy atomicity.



\subsection{Seeking SC-DRF Violations (\Q5)}
\label{sec:scdrf}

This section reports on our use of Alloy to search for violations of
the SC-DRF guarantee in an early draft of the C11 MCM. The
SC-DRF guarantee, in the C11 context, is that if a program is
race-free and involves no atomic operations that are not
"memory\_order\_seq\_cst", then it has only SC semantics.

We search for an execution that satisfies
\[
\stack{\exists X\in\Exec\ldotp \field{X}{A} \subseteq \field{X}{sc}
\wedge {}\\{} \quad X\in dead_{\mm{C11}}
\cap consistent_{\mm{C11-Draft}} - consistent_{\mm{SC}}.}
\]
That is: we seek an execution that is dead (so that its corresponding
litmus test is race-free), uses only SC atomics, and is consistent
under the (draft) C11 MCM but inconsistent under \mm{SC}. The
\mm{SC} MCM is characterised by
%
\begin{eqnarray*}
consistent_{\mm{SC}} &\eqdef& acyclic(rf \cup co \cup fr \cup sb) \\
racefree_{\mm{SC}} &\eqdef& true
\end{eqnarray*}
while the \mm{C11-Draft} MCM has the same axioms as those listed
by Batty et al.~\cite{batty+16}, minus the `S4' axiom that was added
by Batty et al.~\cite{batty+11} to fix exactly this issue.

The execution found by Alloy
%
\begin{equation}
\begin{tikzpicture}[inner sep=1pt, baseline=(a.base)]
\node (a) at (0,0.8) {$\evW{\mosc}{"x"}{1}$};
\node (b) at (0,0) {$\evW{\mosc}{"y"}{1}$};
\node (c) at (1.8,0.8) {$\evW{\mosc}{"y"}{2}$};
\node (d) at (1.8,0) {$\evR{\mosc}{"x"}{0}$};
\draw[edgemo, shift right=0.5mm, inner sep=1mm] (b) to [auto, swap, pos=0.3] node {$co$} (c);
\draw[edgesb, shift right=0.5mm, inner sep=1mm] (a) to [auto, swap] node {$sb$} (b);
\draw[edgesb, shift right=0.5mm, inner sep=1mm] (c) to [auto, swap] node {$sb$} (d);
\draw[edgeS, shift left=0.5mm, inner sep=1mm] (a) to [auto] node {$S$} (b);
\draw[edgeS, shift left=0.5mm, inner sep=1mm] (b) to [auto, pos=0.7] node {$S$} (c);
\draw[edgeS, shift left=0.5mm, inner sep=1mm] (c) to [auto] node {$S$} (d);
\node[sthd, fit=(a)(b)] {};
\node[sthd, fit=(c)(d)] {};
\end{tikzpicture}
\end{equation}
%
is similar that reported by Batty et al.~\cite{batty+11}. It is not SC
because the read of "x" observes a stale value, but it is consistent
under \mm{C11-Draft} because \JWComment{need to finish this sentence
somehow!}.


\input{mca.tex}

%%% Local Variables:
%%% mode: latex
%%% TeX-master: "paper"
%%% End:

\section{Performance Evaluation}
\label{sec:eval}

In this section, we report on an empirical investigation of how our
design decisions affect Alloy's SAT-solving performance.

\begin{table}
\newcommand\Gluc{{\sf G}}
\newcommand\Plin{{\sf P}}
\newcommand\Mini{{\sf M}}
\newcommand\TICK{\textcolor{Red}{\cmark}}
\newcommand\CROSS{\textcolor{Green}{\xmark}}
\centering
\renewcommand\tabcolsep{0.9mm}
\begin{tabularx}{\linewidth}{@{\,}rX@{}rrrcc@{\,}}
\toprule
\multicolumn{2}{@{\,}l}{Task} & $\lvert\E\rvert$ & $t_{\rm enc} / {\rm s}$ & 
$t_{\rm sol} / {\rm s}$ & & \\
\midrule
1 & \Q2 \mm{C11-SRA} vs. \mm{C11} (\S\ref{sec:Q2_c11_sra_simp}) & $6$ 
& $0.7$ & $0.6\phantom{0}$ & \Gluc & \TICK \\
2 & \Q2 \mm{C11-SwRf} vs. \mm{C11} (\S\ref{sec:kyndylan}) & $7$ 
& $0.8$ & $625\phantom{.00}$ & \Gluc & \CROSS \\
3 & \Q2 \mm{C11-SwRf} vs. \mm{C11} (\S\ref{sec:kyndylan}) & $12$ 
& $2\phantom{.0}$ & $214\phantom{.00}$ & \Plin & \TICK \\
4 & \Q2 \mm{C11} vs. \mm{C11-Orig} (\S\ref{sec:Q2_c11_simp_orig})&$5$
& $0.4$ & $0.3\phantom{0}$ & \Gluc & \TICK \\
5 & \Q2 \mm{MCA} vs. \mm{x86} (\S\ref{sec:Q2_mca}) & $9$ 
& $0.8$ & $607\phantom{.00}$ & \Plin & \CROSS \\
6 & \Q2 \mm{MCA} vs. \mm{Power} (\S\ref{sec:Q2_mca}) & $6$ 
& $2\phantom{.0}$ & $0.06$ & \Gluc & \TICK \\
7 & \Q2 \mm{SC} vs. \mm{C11-Draft} (\S\ref{sec:scdrf}) & $4$ 
& $0.4$ & $0.04$ & \Gluc & \TICK \\
8 & \Q2 \mm{PTX2} vs. \mm{PTX1} (\S\ref{sec:Q4_opencl_ptx}) & $7$ 
& $0.7$ & $4\phantom{.00}$ & \Gluc & \TICK \\
9 & \Q3 \mm{C11} (sequencing) (\S\ref{sec:Q3_c11_seq}) & $5$ 
& $0.5$ & $163\phantom{.00}$ & \Mini & \CROSS \\
10 &  \Q3 \mm{C11} (sequencing) (\S\ref{sec:Q3_c11_seq}) & $6$ 
& $0.7$ & $5\phantom{.00}$ & \Plin & \TICK \\
11 & \Q3 \mm{C11} (mem. orders) (\S\ref{sec:Q3_c11_mo}) & $7$ 
& $0.9$ & $51\phantom{.00}$ & \Gluc & \TICK \\
12 & \Q4 \mm{C11} / \mm{x86} & $5+5$ 
& $0.7$ & $13029\phantom{.00}$ & \Plin & \CROSS \\
13 & \Q4 \mm{C11} / \mm{Power} (\S\ref{sec:Q4_c11_power}) & $6+8$ 
& $8\phantom{.0}$ & $91\phantom{.00}$ & \Plin & \TICK \\
14 & \Q4 \mm{OpenCL} / \mm{AMD} (\S\ref{sec:Q4_opencl_amd}) & $2+9$ 
& $6\phantom{.0}$ & $1355\phantom{.00}$ & \Gluc & \TICK \\
15 & \Q4 \mm{OpenCL} / \mm{AMD} (\S\ref{sec:Q4_opencl_amd}) & $4+10$ 
& $16\phantom{.0}$ & $4743\phantom{.00}$ & \Plin & \TICK \\
16 & \Q4 \mm{OpenCL} / \mm{PTX1} (\S\ref{sec:Q4_opencl_ptx}) & $5+8$ 
& $2\phantom{.0}$ & $11\phantom{.00}$ & \Plin & \TICK \\
17 & \Q4 \mm{OpenCL} / \mm{PTX2} (\S\ref{sec:Q4_opencl_ptx}) & $5+15$
& $4\phantom{.0}$ & $9719\phantom{.00}$ & \Plin & \CROSS 
\end{tabularx}
%
\caption{Summary of tasks. We record the number ($\lvert\E\rvert$) of
events in the search space, the time taken (in seconds) to encode ($t_{\rm enc}$)
and then solve ($t_{\rm sol}$) the SAT query, whether the fastest
solver was MiniSat (\Mini), Glucose (\Gluc), or Plingeling (\Plin),
and whether a solution was found (\TICK) or not (\CROSS). For \Q4
tasks, $m+n$ means that $\E$ is partitioned into $m$ software-level and $n$ architecture-level events.}

\label{tab:tasks}
\end{table}

Table~\ref{tab:tasks} summarises the tasks on which we have evaluated
our technique. We used a machine with four 16-core 2.1 GHz AMD
Opteron processors and 128 GB of RAM.

%%%%%
% C11/Power results (6 SE + 8 HE)
% Without simplified compiler: 25383s (7 hours)
% With:
% Plingeling: 55s (+8s of processing), 138s, 70s, 102s
% Glucose: 1723s, 2274s, 2600s, 2364s 
% Minisat: 171s, 146s, 168s, 149s
%%%%%

\tikzset{
  glucosebar/.style={draw=none, fill=white!30!red},
  minisatbar/.style={draw=none, fill=white!30!blue},
  plingelingbar/.style={draw=none, fill=white!30!green},
}

\subsection{Choice of SAT Solver}

Figure~\ref{fig:solver_results} summarises the performance of three
SAT solvers on our tasks: Glucose (version~2.1)~\cite{audemard+09},
MiniSat (version~2.2)~\cite{een+03}, and Plingeling~\cite{biere10}. Each bar shows
the mean solve time over 4 runs, plus the minimum and maximum
time.\footnote{A more thorough comparison of SAT solvers would control for the
order of clauses, which greatly influences
performance~\cite{nikolic10}.}
Plingeling is able to complete all tasks, whereas Glucose and MiniSat
time out on three and five of them respectively. On the tasks all
three solvers complete, MiniSat takes an average of 9 minutes, Glucose
takes 8, and Plingeling takes 6. Plingeling's relatively high startup
overhead is evident on the quicker tasks.

% 12 tasks where all solvers complete: 1,2,4,5 ,6,7,9 ,10,11,13,14,16
% Averages: Glucose=8min, Plingeling=6min, Minisat=9min.

\begin{figure}

\def\mybarwidth{1.5pt}
\def\timeout{14400}

\centering

\begin{tikzpicture}
\begin{semilogyaxis}[
  legend pos=north west,
  xtick={1,...,17},
  xticklabels={1,2,3,4,5,6,7,8,9,10,11,12,13,14,15,16,17},
  xmin=0.3,
  xmax=17.7,
  x label style={at={(axis description cs:-0.14,0.155)},anchor=north
  west},
  log ticks with fixed point,
  ytick={1,100,10000},
  yticklabels={$1$, $100$, \llap{$10000$}},
  ylabel near ticks,
  xlabel={Task:},
  ylabel={Solve time /s\hspace*{2mm}},
  xtick style={draw=none},
  ybar=0.5pt,
  bar width=\mybarwidth,
  log origin=infty,
  height=4cm,
  width=8.9cm,
  legend columns=-1,
]
\addplot[minisatbar]
plot [error bars/.cd, y dir=both, y explicit, error mark options={mark
size=\mybarwidth/2, rotate=90}] 
table [x=x, y=y, y error minus expr=\thisrow{y}-\thisrow{ymin}, y
error plus expr=\thisrow{ymax}-\thisrow{y}, col sep=comma]
{MiniSat.csv};

\addplot[glucosebar]
plot [error bars/.cd, y dir=both, y explicit, error mark options={mark
size=\mybarwidth/2, rotate=90}] 
table [x=x, y=y, y error minus expr=\thisrow{y}-\thisrow{ymin}, y
error plus expr=\thisrow{ymax}-\thisrow{y}, col sep=comma]
{Glucose.csv};

\addplot[plingelingbar]
plot [error bars/.cd, y dir=both, y explicit, error mark options={mark
size=\mybarwidth/2, rotate=90}] 
table [x=x, y=y, y error minus expr=\thisrow{y}-\thisrow{ymin}, y
error plus expr=\thisrow{ymax}-\thisrow{y}, col sep=comma]
{Plingeling.csv};

\draw [black!50, dashed] ({rel axis cs:0,0}|-{axis cs:12,\timeout}) -- ({rel axis
cs:1,0}|-{axis cs:12,\timeout}) node [inner sep=0pt,pos=0.55,
below=0.3mm] {\scriptsize 4-hour timeout};

%\legend{MiniSat\hspace*{2mm}, Glucose\hspace*{2mm}, Plingeling}
\end{semilogyaxis}
\end{tikzpicture}
\def\fooo{MiniSat 
(\,{\tikz\draw[minisatbar,/tikz/.cd,yshift=-0.25em]
        (0cm,0cm) rectangle (\mybarwidth,0.8em);}\,), Glucose
(\,{\tikz\draw[glucosebar,/tikz/.cd,yshift=-0.25em]
        (0cm,0cm) rectangle (\mybarwidth,0.8em);}\,) and Plingeling (\,{\tikz\draw[plingelingbar,/tikz/.cd,yshift=-0.25em]
        (0cm,0cm) rectangle (\mybarwidth,0.8em);}\,)
}
\vspace*{-3mm}
\caption{Comparing \protect\fooo} 
\label{fig:solver_results}
\end{figure}
 

\subsection{Combined vs. Separate Event Sets}

For \Q4 tasks (\S\ref{sec:compilers}), we can choose to draw
software-level and architecture-level events either from a single set
$\E$ or from disjoint partitions $\SE$ and $\HE$. The former approach
is attractive because it gives the user one fewer parameter to
control, and it minimises the total number of events required to find
solutions because a single element of $\E$ can represent both a
software-level event and an architecture-level event. However, as
shown in Tab.~\ref{tab:tasks}, we choose the latter approach. To see
why, consider Task 17. To validate the OpenCL/PTX mapping for OpenCL
executions of up to 5 events, we must consider PTX executions of up to
15 events (since each OpenCL event can map to up to three PTX events).
We found that setting $\lvert\SE\rvert = 5$ and $\lvert\HE\rvert = 15$
led to a tractable constraint-solving problem for Alloy, but that
merely setting $\lvert\E\rvert = 15$, which involves fewer events in
total but includes OpenCL executions with more than 5 events, rendered the problem insoluble.

\subsection{Recursive Definitions vs. Fixed Unrolling}
\label{sec:fixpoints}

\begin{figure}

\def\mybarwidth{6.5pt}

\centering

\begin{tikzpicture}
\begin{semilogyaxis}[
  legend pos=north west,
  xtick={2,3,4,5,6,8},
  xticklabels={$2$,$3$,$4$,$5$,$6$,$\infty$},
  xmin=1.5,
  xmax=8.5,
  log ticks with fixed point,
  ytick={200,1000,5000},
  yticklabels={$200$, $1000$, $5000$},
  x label style={at={(axis description cs:-0.02,0.23)},anchor=north west},
  ylabel near ticks,
  xlabel={$k$:},
  ylabel={Solve time /s},
  xtick style={draw=none},
  ybar=0.5pt,
  bar width=\mybarwidth,
  log origin=infty,
  height=3.3cm,
  width=8.7cm,
  legend columns=-1,
]
\addplot[glucosebar]
plot [error bars/.cd, y dir=both, y explicit, error mark options={mark
size=\mybarwidth/2, rotate=90}] 
table [x expr=\thisrow{k}+1, y=mean, y error minus expr=\thisrow{mean}-\thisrow{min}, y
error plus expr=\thisrow{max}-\thisrow{mean}, col sep=comma]
{Fixpoints.csv};
\end{semilogyaxis}
\end{tikzpicture}
\vspace*{-3mm}
\caption{Performance of Task 13 with fixpoints unrolled $k$ times} 
\label{fig:fixpoint_results}
\end{figure}

Alglave et al.'s formalisation of the Power MCM in the ".cat" format
involves several recursively-defined relations, but Alloy does not
support definitions of the form `"let rec $r = f(r)$"'. Accordingly,
in our formalisation of the Power MCM~\cite{popl17supplementary}, we
expand the recursive construction explicitly, by requiring the
existence of an $r$ satisfying $f(r) \subseteq r$ and
$\forall r'\ldotp f(r')\subseteq r' \Rightarrow r \subseteq r'$. The
latter constraint involves universal quantification over relations and
hence requires the higher-order AlloyStar solver~\cite{milicevic+15}.
A first-order alternative is simply to unroll the recursive
definitions a few times; that is, to set $r \eqdef f^k(\emptyset)$ for a
fixed $k$. We found, for the small search scopes involved in our work,
that $k\ge 2$ is sufficient for avoiding false positives when checking
a compiler mapping from C11 (Task~13).
Figure~\ref{fig:fixpoint_results} shows that the proper fixpoint
construction (i.e., $k=\infty$) is much more expensive than a
fixed unrolling.


%%% Local Variables:
%%% mode: latex
%%% TeX-master: "paper"
%%% End:

\section{Related Work}
\label{sec:related}

Existing tools for MCM reasoning typically take a concurrent program
as input, and produce all of the executions allowed under a given MCM.
% Some tools, such as MemSAT~\cite{torlak+10},
% Nitpick~\cite{blanchette+11}, Nemos~\cite{yang+04}, and
% SATCheck~\cite{demsky+15}, rely on SAT solvers to discover
% executions, while others, such as memevents~\cite{sarkar+09},
% CDSChecker~\cite{norris+13}, CBMC~\cite{alglave+13},
% \textsf{herd}~\cite{alglave+14}, CppMem~\cite{batty+11},
% PPCMEM~\cite{sarkar+11}, enumerate executions explicitly.
Some rely on SAT solvers to discover executions~\cite{torlak+10,
blanchette+11, yang+04, demsky+15}, while others enumerate them
explicitly~\cite{sarkar+09, norris+13, alglave+13, alglave+14,
batty+11, sarkar+11}.
%
Our work tackles, in a sense, the `inverse' problem: we start with
executions that witness interesting MCM properties, and go on to
produce programs that can give rise to these executions.

\paragraph{Other works tackling \Q1--\Q4}

Regarding \Q1, Darbari et al.~\cite{darbari+16} have automatically
generated conformance tests for HSA~\cite{hsa-foundation15} from an
Event-B specification~\cite{abrial+10} of its MCM. Alglave et al.'s
\textsf{diy} tool~\cite{alglave+10} generates conformance tests for a
range of MCMs based on \emph{critical cycles}~\cite{shasha+88}. We
find that our Alloy-based approach has several advantages over
\textsf{diy} when generating conformance tests. First, we can
straightforwardly handle custom MCM constructs (e.g., C11 memory
orders) while \textsf{diy} currently does not. Second, we generate
only tests needed for conformance, while \textsf{diy} generates
many more. Third, \textsf{diy} can also miss some tests required to
distinguish two MCMs (such as the single-address tests we saw in
\S\ref{sec:Q4_opencl_ptx}), if not guided carefully through
user-supplied critical cycles.

Regarding \Q2, there is a long history of unifying frameworks for
comparing MCMs~\cite{adve93, gharachorloo95, higham+97, higham+07,
shasha+88, collier92}, and of manual proofs that different
formulations of the same MCM are equivalent~\cite{owens+09,
mador-haim+12}. On the automation side, Mador-Haim et
al.~\cite{mador-haim+10} have, like ourselves, used a SAT solver as
part of a tool for generating litmus tests that distinguish MCMs.
However, where we use the solver to \emph{generate} tests, they just
use it to \emph{check} the behaviour of pre-generated tests. Given
that generating all 6-instruction litmus tests takes them `a few
minutes' (in 2010), we expect that their approach of explicit
enumeration would not scale to find the 12-instruction litmus tests
that are sometimes necessary to distinguish MCMs
(see~\S\ref{sec:kyndylan}), and which Alloy is able to find in just
4 minutes.

Prior work has proved (manually) the validity of compiler
optimisations in non-SC MCMs~\cite{sevcik+08, sevcik11}. Since
optimisations should not introduce new behaviours, this problem is
related to monotonicity (\Q3). On the automation side, the Traver
tool~\cite{burckhardt+10} uses an automated theorem prover to
verify/falsify a given program transformation against a non-SC
MCM. Unlike our work, Traver does not support multi-threaded
optimisations such as linearisation. Chakraborty et al. have built a
tool~\cite{chakraborty+16} that automatically verifies that LLVM
optimisations preserve C11 semantics. Like Traver, their tool only
checks specific \emph{instances} of an optimisation, while our
approach is able to check optimisations themselves. Morisset et
al.~\cite{morisset+13} use random testing to validate optimisations
(albeit those not involving atomic operations) against the C11 MCM;
Vafeiadis et al.~\cite{vafeiadis+15} then show (manually) that several
of these optimisations are invalid when atomic operations are
involved. Our work, in turn, shows how several of Vafeiadis et al.'s
results can be recreated automatically, often in a simpler form
(\S\ref{sec:monotonicity}).

Regarding \Q4, prior work has proved (manually) the correctness of
both compiler mappings~\cite{batty+11, batty+12, wickerson+15a} and
full compilers~\cite{sevcik+11} in a non-SC context. These proofs all
involve intensive manual effort, in contrast to our lightweight
automatic checking, though our checking can of course only guarantee
the absence of bugs within Alloy's search scope. On the automation
side, Lustig et al.\ have built tools for finding and verifying
semantics-preserving translations, but where we focus on
language/architecture translation, they focus on
architecture/microarchitecture~\cite{lustig+14} and
architecture/architecture~\cite{lustig+15} translation. Very recent
work by Trippel et al.~\cite{trippel+16} has produced a framework that
can check language/architecture mappings; this works by enumerating
standard litmus tests and then simulating each one against both the
language-level MCM and, after compilation, the architecture-level MCM.
 
\paragraph{Reflections on Alloy}
Alloy is a mature, open-source, and widely-used modelling application,
and its trio of features -- a modular and object-oriented modelling
language, an automatic constraint solver, and a graphical visualiser
-- makes it ideal for MCM development. Although Tab.~\ref{tab:tasks}
shows several lengthy solving times, those figures are obtained once the
search space has been set as large as computational feasibility
allows. Given smaller search spaces, as would be appropriate during
MCM prototyping, Alloy is suitable for interactive use.

When does Alloy's failure to distinguish two MCMs become a proof that
they are equivalent? Mador-Haim et al.~\cite{mador-haim+11} prove that
6 events are enough, but their result applies only to multi-copy
atomic, architecture-level MCMs (see~\S\ref{sec:Q2_mca}). Momtahan~\cite{momtahan05} gives a result for
general Alloy models, but imposes strong restrictions on
quantifiers that our models do not meet.

Ivy~\cite{padon+16} defines a relational modelling language similar to
Alloy's. Where Alloy ensures the decidability of instance-finding by
restricting to a finite search scope (which limits its usefulness for
verification), Ivy instead restricts formulas to the form
$\exists\bar{x}\ldotp\forall\bar{y}\ldotp\phi$ (for quantifier-free
$\phi$). If our models can be rephrased to fit into Ivy's restricted
language, there is the potential not just to `debug' MCM properties,
but to \emph{verify} them. Another language related to Alloy,
Rosette~\cite{torlak+13a}, is used in very recent work by Bornholt et
al.~\cite{bornholt+16} to solve the problem of synthesising a MCM from
a set of desired litmus test outcomes.

We find that Alloy has several advantages over other frameworks that
have been used to reason about MCMs, such as Isabelle
(e.g.,~\cite{batty+11}), Lem~\cite{mulligan+14}
(e.g.,~\cite{batty+12}), Coq (e.g.,~\cite{vafeiadis+15}), and
".cat"~\cite{alglave+14} (e.g.,~\cite{batty+16}). A key advantage is
that the entire memory modelling process can be conducted within
Alloy: the Alloy modelling language can express programming languages,
compiler mappings, MCMs, and properties to test, the Alloy Analyzer
can discover solutions, and the Alloy Visualizer can display solutions
using theming customised for the model at hand. Alglave et al.'s
".cat" framework, like Alloy, allows MCM axioms to be expressed in the
concise propositional relation calculus~\cite{tarski41}, but Alloy
also supports the more powerful predicate calculus as a fallback. As
such, Alloy is expressive enough to capture both axiomatic and
operational MCMs, while remaining sufficiently restrictive that
fully-automatic property checking is computationally tractable.
% Unlike ".cat", Alloy does not support recursive definitions (though
% these can be encoded, \S\ref{sec:fixpoints}), and its operator
% precedence rules mean that models tend to be cluttered with more
% parentheses than ".cat" models.


%%% Local Variables:
%%% mode: latex
%%% TeX-master: "paper"
%%% End:

\section{Conclusion}
\label{sec:conc}

By solving relational constraints between executions and then lifting
solutions to litmus tests, Alloy can generate conformance tests,
compare MCMs (against each other and against general properties like
SC-DRF and multi-copy atomicity), and check monotonicity and compiler
mappings. As such, we believe that Alloy should become an essential
component of the memory modeller's toolbox. Indeed, we are already
working with two large processor vendors to apply our technique to
their recent and upcoming architectures and languages. Other future
work includes applying our technique to more recent MCMs that
are defined in a non-axiomatic style~\cite{jeffrey+16,
pichon-pharabod+16, flur+16, kang+17}.

Although Alloy's lightweight, automatic approach cannot give the same
universal assurance as fully mechanised theorems, we have found it
invaluable in practice, because even (and perhaps \emph{especially})
in the complex and counterintuitive world of non-SC MCMs, Jackson's
maxim~\cite{jackson12a} holds true: \emph{Most bugs have small
counterexamples}.


% Acknowledgements:
%%%%%%%%%%%%%%%%%%%%%%%%%%%%%%%%
% Alastair Donaldson
% Matthew Parkinson
% Carsten Fuhs
% Daniel Jackson?
% Ali Sezgin?
% Ganesh Gopalakrishnan?
% Stefan Kahrs?
% Luc Maranget

\subsection*{Acknowledgements}

We thank Alastair Donaldson, Carsten Fuhs, Daniel Jackson, Luc
Maranget, and Matthew Parkinson for their valuable feedback. We
acknowledge the support of a Research Fellowship from the Royal
Academy of Engineering and the Lloyd's Register Foundation (Batty), a
Research Chair from the Royal Academy of Engineering and Imagination
Technologies (Constantinides), an EPSRC Impact Acceleration Award,
EPSRC grants EP/I020357/1, EP/K034448/1, and EP/K015168/1, and a gift
from Intel Corporation.


%\clearpage
\printbibliography

%\clearpage\appendix
\section{Bonus stuff, probably won't fit in}

\subsection{Monotonicity violation in C11}
Alloy finds the following pair of 6-event executions that witness a
monotonicity violation in C11:
\begin{equation*}
\begin{tikzpicture}[inner sep=1pt]
\node[event, anchor=west](a) at (1,2.5) 
{$\evR{\morlx}{"x"}{1}$};

\node[event, anchor=west](b) at (1,1.5) 
{$\evR{"na"}{"a"}{1}$};

\node[event, anchor=west](c) at (1,0.5) 
{$\evW{\mosc}{"y"}{1}$};

\node[event, anchor=west](d) at (2.7,2.5) 
{$\evR{\morlx}{"y"}{1}$};

\node[event, anchor=west](e) at (2.7,1.5) 
{$\evW{\morlx}{"x"}{1}$};

\node[event, anchor=west](f) at (-0.5,2) 
{$\evW{"na"}{"a"}{1}$};

\node[event, anchor=west](a') at (5,2) 
{$\evR{\morlx}{"x"}{1}$};

\node[event, anchor=west](b') at (5,1) 
{$\evR{"na"}{"a"}{1}$};

\node[event, anchor=west](c') at (5,0) 
{$\evW{\mosc}{"y"}{1}$};

\node[event, anchor=west](d') at (6.5,2) 
{$\evR{\morlx}{"y"}{1}$};

\node[event, anchor=west](e') at (6.5,1) 
{$\evW{\morlx}{"x"}{1}$};

\node[event, anchor=west](f') at (5,3) 
{$\evW{"na"}{"a"}{1}$};

\foreach \i/\j in {a/b, b/c, d/e, a'/b', b'/c', d'/e'} {
\draw[edgesb] ([xshift=5mm]\i.south west) to[auto,swap,pos=0.4]
node{$sb$} ([xshift=5mm]\j.north west -| \i.south west);
\draw[edgesb] ([xshift=6mm]\i.south west) to[auto,pos=0.4]
node{$cd$} ([xshift=6mm]\j.north west -| \i.south west);
}
\draw[edgesb] ([xshift=5mm]f'.south west) to[auto,swap,pos=0.4]
node{$sb$} ([xshift=5mm]a'.north west -| a'.south west);

\draw[edgerf] (e) to[auto, swap, pos=0.7] node{$rf$} (a);
\draw[edgerf] (c) to[auto, pos=0.3, swap, out=30, in=210] node{$rf$} (d);
\draw[edgerf] (f) to[auto, pos=0.4] node{$rf$} (b);

\draw[edgerf] (e') to[auto, swap, pos=0.7] node{$rf$} (a');
\draw[edgerf] (c') to[auto, swap, pos=0.3, out=30, in=210] node{$rf$} (d');
\draw[edgerf] (f') to[auto, swap, pos=0.2, bend right=70] node{$rf$} (b');

\draw[edgepi] (a) to[auto,swap,pos=0.8] node{$\pi$} (a');
\draw[edgepi] (b) to[auto,swap,pos=0.8] node{$\pi$} (b');
\draw[edgepi] (c) to[auto] node{$\pi$} (c');
\draw[edgepi] (d) to[auto, pos=0.15] node{$\pi$} (d');
\draw[edgepi] (e) to[auto, pos=0.1] node{$\pi$} (e');
\draw[edgepi] (f) to[auto, pos=0.7,out=45,in=180] node{$\pi$} (f');

\node[sthd, fit=(f)] {};
\node[sthd, fit=(a)(b)(c)] {};
\node[sthd, fit=(d)(e)] {};
\node[sthd, fit=(f')(a')(b')(c')] {};
\node[sthd, fit=(d')(e')] {};
\end{tikzpicture}
\end{equation*}
where left-hand execution is inconsistent (because the write to "a"
does not happen-before the read that observes its value) but the
right-hand execution (obtained by sequentialising two of the threads)
is consistent (since "a"'s write now happens-before its read). This is
essentially the same execution as that discovered by Vafeiadis et
al.~\cite[Fig.~1]{vafeiadis+15}; it is just a little less elegant:
Alloy chose the write to "y" to be sequentially-consistent when a
relaxed write would suffice.

\subsection{Generating monotonic programs} We rely on the following
property of our program-generating function:
\[
\stack{
\text{if}~"strengthen"(X, Y, rf, co) \\
\text{then}~ltest(X,rf,co) \sqsubseteq ltest(Y,rf,co)
}
\]
The $\sqsubseteq$ relation is defined as the smallest partial order
that is compatible with the syntax of litmus tests and satisfies:
\begin{mathpar}
\inferrule{ }{\mathbf{st}(x,v,\mathbf{rel}) \sqsubseteq \mathbf{st}(x,v,\mathbf{rlx})}
\and
\inferrule{ }{\mathbf{st}(x,v,\mathbf{sc}) \sqsubseteq \mathbf{st}(x,v,\mathbf{rel})}
\and
\inferrule{ }{\mathbf{ld}(r,x,\mathbf{sc}) \sqsubseteq \mathbf{ld}(r,x,\mathbf{acq})}
\and
\inferrule{ }{\mathbf{ld}(r,x,\mathbf{acq}) \sqsubseteq \mathbf{ld}(r,x,\mathbf{rlx})}
\and
\inferrule{ }{\mathbf{xc}(r,x,v,\mathbf{sc}) \sqsubseteq \mathbf{xc}(r,x,v,\mathbf{ar})}
\and
\inferrule{ }{\mathbf{xc}(r,x,v,\mathbf{ar}) \sqsubseteq \mathbf{xc}(r,x,v,\mathbf{acq})}
\and
\inferrule{ }{\mathbf{xc}(r,x,v,\mathbf{ar}) \sqsubseteq \mathbf{xc}(r,x,v,\mathbf{rel})}
\and
\inferrule{ }{\mathbf{xc}(r,x,v,\mathbf{acq}) \sqsubseteq \mathbf{xc}(r,x,v,\mathbf{rlx})}
\and
\inferrule{ }{\mathbf{xc}(r,x,v,\mathbf{rel}) \sqsubseteq \mathbf{xc}(r,x,v,\mathbf{rlx})}
\and
\inferrule{ }{C_1 ; C_2 \sqsubseteq C_1 + C_2}
\and
\inferrule{ }{C_1 ; C_2 \sqsubseteq C_1 \parallel C_2}
\end{mathpar}

\subsection{Compiling C11 to x86}
A mapping $\pi$ implements the C11-to-x86 compilation scheme if it
satisfies the following constraints:

\paragraph{\upshape $"mapping"(\pi,X,Y)$:}
\begin{mathpar}
\labelb{eq:c11x86:pibij}~
(\pi^{-1}\join\pi) = \stor{Y."ev"}
\and
\labelb{eq:c11x86:sb}~
(X."sb" \join \pi) = (\pi \join Y."sb")
\and
\labelb{eq:c11x86:sthd}~
(X.\sthd \join \pi) = (\pi\join Y.\sthd)
\and
\labelb{eq:c11x86:sloc}~
(X.\sloc \join \pi) = (\pi\join Y.\sloc)
\and
\labelb{eq:c11x86:cd}~
(X."cd" \join \pi) = (\pi\join Y."cd")
\and
\labelb{eq:c11x86:ad}~
(X."ad" \join \pi) = (\pi\join Y."ad")
\and
\labelb{eq:c11x86:dd}~
(X."dd" \join \pi) = (\pi\join Y."dd")
\and
\labelb{eq:c11x86:r}~
\pi (X."R") = Y."R"
\and 
\labelb{eq:c11x86:w}~
\pi(X."W") = Y."W"
\and
\labelb{eq:c11x86:a}~
\pi(X."sc") = Y."A"
\and 
\labelb{eq:c11x86:scf}~
\pi(X."F" \cap X."sc") = Y."F" 
\and
\labelb{eq:c11x86:notscf}~
\pi(X."F" - X."sc") \ncap Y."F" 
\end{mathpar}

This mapping is one-to-one~\refb{eq:c11x86:pibij}. It preserves
sequenced-before~\refb{eq:c11x86:sb}, threads~\refb{eq:c11x86:sthd},
locations~\refb{eq:c11x86:sloc}, and
dependencies~\refb{eq:c11x86:cd}~\refb{eq:c11x86:ad}~\refb{eq:c11x86:dd}.
Reads are mapped to reads~\refb{eq:c11x86:r} and writes are mapped to
writes~\refb{eq:c11x86:w}. SC events are mapped to atomic
events~\refb{eq:c11x86:a}. SC fences are mapped to
fences~\refb{eq:c11x86:scf}, and non-SC fences are mapped to
no-ops~\refb{eq:c11x86:notscf}.


\subsection{Generating executable code}

\begin{figure}
\[
\renewcommand\arraystretch{1.2}
\begin{array}{@{}r@{~~}c@{~~}l@{}}
\cof{x} &=& \C{\&} x \\
\cof{v} &=& v \\
\cof{e + 0 \times r} &=& \C{(} \cof{e} \C{+0*} r \C{)} \\
\cof{\mathbf{st}(e_x, e_v, \mathrm{None})} &=& \C{*} \cof{e_x} \C{=} \cof{e_v} 
\\
\cof{\mathbf{st}(e_x, e_v, \mathrm{Some}~mo)} &=& \\
\multicolumn{3}{@{}r@{}}{
\C{(atomic\_store((atomic\_int*)} \cof{e_x} \C{,}
\cof{e_v} \C{,} mo \C{),0)}} 
\\
\cof{\mathbf{ld}(r,e_x,\mathrm{None})} &=& r \C{=((int*)}\cof{e_x} \C{)} \\
\cof{\mathbf{ld}(r,e_x,\mathrm{Some}~mo)} &=& r \C{=atomic\_load(}
\cof{e_x} \C{,} mo \C{)} 
\\
\cof{\mathbf{cas}(r,e_x,v,e_{v'},\mathrm{Some}~mo)} &=& \\
\multicolumn{3}{@{}r@{}}{\stack{r \C{=(atomic\_compare\_exchange\_strong(} \\\quad \C{(atomic\_int*)}
\cof{e_x} \C{,} v \C{,} \cof{e_{v'}} \C{,} mo \C{,relaxed))?} v \C{:0}}} 
\\
\cof{\mathbf{fence}(\mathrm{Some}~mo)} &=& \C{atomic\_thread\_fence(} mo \C{)}
\\
\cof{\mathbf{if}~r = v~\mathbf{then}~a} &=& r\C{==}v\C{?}\cof{a}\C{:0}
\\
\cof{C_1 + C_2} &=& \cof{C_1}\C{+}\cof{C_2} 
\\
\cof{C_1 \join C_2} &=& \C{(}\cof{C_1}\C{,}\cof{C_2}\C{)}
\end{array}
\]
\caption{Generating executable C code \JWComment{Needs checking}}
\label{fig:generating_C}
\end{figure}

It is quite straightforward to convert from a term in $\Prog{\tau}$ to
a piece of executable C11 or assembly code. For example,
Fig.~\ref{fig:generating_C} demonstrates how we generate C11 code.

\subsection{Monotonicity results for x86} 
To extend the $strengthen$ relation to
the x86 setting, we add the constraint $\field{X}{locked} \subseteq
\field{Y}{locked}$ -- this allows the stronger execution to have
additional `locked' events. We used Alloy to search for monotonicity
violations in the x86-TSO model, as formalised by Alglave et
al.~\cite{alglave+14} -- that is, we sought solutions for
$\GeneralExec(\mm{x86-TSO},\mm{x86-TSO}, strengthen_{\rm x86})$. Alloy
discovered the following pair of executions:
%
\begin{equation*}
\begin{tikzpicture}[inner sep=1pt]
\node[event, anchor=west](a) at (2,2) 
{\evtlbl{$a$}$\evW{}{"x"}{1}$};

\node[event, anchor=west](b) at (2,1) 
{\evtlbl{$b$}$\evR{"lock"}{"y"}{0}$};

\node[event, anchor=west](c) at (4,2) 
{\evtlbl{$c$}$\evW{"lock"}{"y"}{1}$};

\node[event, anchor=west](d) at (4,1) 
{\evtlbl{$d$}$\evR{}{"x"}{0}$};

\node[event, anchor=west](a') at (6.5,2) 
{\evtlbl{$a'$}$\evW{}{"x"}{1}$};

\node[event, anchor=west](b') at (6.5,1) 
{\evtlbl{$b'$}$\evR{"lock"}{"y"}{0}$};

\node[event, anchor=west](c') at (8.5,2) 
{\evtlbl{$c'$}$\evW{"lock"}{"y"}{1}$};

\node[event, anchor=west](d') at (8.5,1) 
{\evtlbl{$d'$}$\evR{"lock"}{"x"}{0}$};

\foreach \i/\j in {a/b, a'/b',c/d, c'/d'}
\draw[edgesb] ([xshift=8mm]\i.south west) to[auto,swap,pos=0.4]
node{$sb$} ([xshift=8mm]\j.north west -| \i.south west);

\node[sthd, fit=(a)(b)] {};
\node[sthd, fit=(c)(d)] {};
\node[sthd, fit=(a')(b')] {};
\node[sthd, fit=(c')(d')] {};
\end{tikzpicture}
\end{equation*}
%
The left-hand execution is forbidden because a write cannot be
reordered with a later read if one of them is locked. The right-hand
execution is allowed because of a bug in Alglave et al.'s
formalisation of \mm{x86-TSO}: they neglect to forbid write/read
reordering when both the write ($c'$) and the read ($d'$) are
locked. Upon amending their model and rerunning our test, Alloy could
find no further monotonicity violations.

\subsection{Amusing remark about Prolog}

\begin{remark}Recalling Prolog's `green cuts' and `red
cuts'~\cite{bratko86}, we could describe a postcondition as `red' if
it rules out an execution that is consistent and racy, and `green'
otherwise. Herd and cppmem allow users the freedom to generate red
postconditions, for pedagogical reasons. In our work, we must only
generate green postconditions. \end{remark}

\subsection{Compiling C11 to Power}

\begin{figure}[t]
\begin{myFrame}{$compile_{\rm C11/PPC}(\pi,X,Y)$}
\begin{mathpar}
\labelb{eq:c11ppc:pi1}~
\stor{\field{X}{E}} = (\pi \join \pi^{-1})
\and
\labelb{eq:c11ppc:pi2}~
\stor{\field{Y}{E}} \subseteq (\pi^{-1} \join \pi)
\and
\labelb{eq:c11ppc:sb}~
\stack{
\forall (e_1,e_2)\in \field{X}{sb} \ldotp (\{e_1\}\join \pi) \times
(\{e_2\}\join\pi) \subseteq \field{Y}{sb}
}
\and
\labelb{eq:c11ppc:sloc}~
\field{X}{sloc} = \pi \join \field{Y}{sloc} \join \pi^{-1}
\and
\labelb{eq:c11ppc:sthd}~
\field{Y}{sthd} = \pi^{-1} \join \field{X}{sthd} \join \pi
\and
\labelb{eq:c11ppc:rf}~
\field{X}{rf} = \pi \join \field{Y}{rf} \join \pi^{-1}
\and
\labelb{eq:c11ppc:co}~
\field{X}{co} = \pi \join \field{Y}{co} \join \pi^{-1}
\and
\labelb{eq:c11ppc:Rrlx}~
\forall e\in \field{X}{R} - \field{X}{W} - \field{X}{acq}\ldotp
  \exists e' \in \field{Y}{R} \ldotp 
  (\{e\}\join\pi)  = \{e'\} 
\and
\labelb{eq:c11ppc:Racq}~
\stack{\forall e\in \field{X}{R} \cap \field{X}{acq} - \field{X}{W} -
\field{X}{sc}\ldotp {}\\{}
  \exists e'_1 \in \field{Y}{R}\ldotp 
  \exists e'_2 \in \field{Y}{isync}\ldotp
  (\{e\}\join\pi)  = \{e'_1\}\uplus\{e'_2\} \wedge {}\\{}
  (e'_1,e'_2) \in imm(\field{Y}{sb}) \cap \field{Y}{cd} -
  \field{Y}{ad} - \field{Y}{dd} \wedge {}\\{}
  (\forall e'\ldotp (e'_1,e') \in
  \field{Y}{sb} \Rightarrow (e'_1,e') \in \field{Y}{cd})}
\and
\labelb{eq:c11ppc:Rsc}~
\labelb{eq:c11ppc:Wrlx}~
\labelb{eq:c11ppc:Wrel}~
\labelb{eq:c11ppc:Wsc}~
\labelb{eq:c11ppc:RMWrlx}~
\labelb{eq:c11ppc:RMWacq}~
\labelb{eq:c11ppc:RMWrel}~
\labelb{eq:c11ppc:RMWar}~
\labelb{eq:c11ppc:RMWsc}~
\labelb{eq:c11ppc:Frlx}~
\labelb{eq:c11ppc:Far}~
\labelb{eq:c11ppc:Fsc}~\text{[see full version]}
\and
\labelb{eq:c11ppc:ad}~
\field{X}{ad} = \pi \join \field{Y}{ad} \join \pi^{-1}
\and
\labelb{eq:c11ppc:dd}~
\field{X}{dd} = \pi \join \field{Y}{dd} \join \pi^{-1}
\and
\labelb{eq:c11ppc:cd1}~
\stack{
\forall (e_1,e_2) \in \field{X}{cd} \ldotp \exists e'_1 \in
(\{e_1\}\join\pi)\ldotp {}\\{} (\{e'_1\}\times
(\{e_2\}\join\pi)) \subseteq \field{Y}{cd}
}
\and
\labelb{eq:c11ppc:cd2}~
\stack{
\forall (e'_1,e'_2) \in \field{Y}{cd} \ldotp \exists (e_1,e_2) \in
(\field{X}{cd})^{?} \ldotp {}\\{}
e'_1 \in (\{e_1\}\join\pi) \wedge 
e'_2 \in (\{e_2\}\join\pi)
}
\end{mathpar}
\end{myFrame}
\caption{Compilation scheme from C11 to Power}
\label{fig:c11_ppc}
\end{figure}

\label{sec:Q4_c11_ppc}
A C execution $X$ is deemed to compile to a Power execution $Y$ if
there exists a relation $\pi$ between the events of the former and
those of the latter such that $compile_{\rm C11/PPC}(\pi,X,Y)$ holds,
as defined in Fig.~\ref{fig:c11_ppc}.

The $\pi$ relation is injective, surjective and left-total (i.e., a
multivalued function)~\refb{eq:c11ppc:pi1}~\refb{eq:c11ppc:pi2}. 

The mapping must not remove $sb$ edges, but can add
them~\refb{eq:c11ppc:sb}. New $sb$ edges are created either when a
partial $sb$ is linearised, or when a single software-level event maps
to a chain of architecture-level events.

A non-atomic or relaxed read maps to a single
read~\refb{eq:c11ppc:Rrlx}. An acquire read maps to a pair of
architecture-level events, say $e'_1$ and $e'_2$, where $e'_1$ is a
read, $e'_2$ is an "isync", $e_1$ and $e_2$ are sequenced
consecutively, and every event in $Y$ that is sequenced after $e'_1$
becomes control-dependent on $e'_1$~\refb{eq:c11ppc:Racq}. The full
version of our mapping also includes SC reads~\refb{eq:c11ppc:Rsc},
relaxed/release/SC
writes~\refb{eq:c11ppc:Wrlx}~\refb{eq:c11ppc:Wrel}~\refb{eq:c11ppc:Wsc},
RMWs~\refb{eq:c11ppc:RMWrlx}~\refb{eq:c11ppc:RMWacq}~\refb{eq:c11ppc:RMWrel}~\refb{eq:c11ppc:RMWar}~\refb{eq:c11ppc:RMWsc},
and
fences~\refb{eq:c11ppc:Frlx}~\refb{eq:c11ppc:Far}~\refb{eq:c11ppc:Fsc}.

The mapping does not affect address or data
dependencies~\refb{eq:c11ppc:ad}~\refb{eq:c11ppc:dd}. \JWComment{Need
to think more about control
dependencies~\refb{eq:c11ppc:cd1}~\refb{eq:c11ppc:cd2}.} The mapping
does not change locations~\refb{eq:c11ppc:sloc} or
threading~\refb{eq:c11ppc:sthd}, and the $rf$ and $co$ maps are
unaffected~\refb{eq:c11ppc:rf}~\refb{eq:c11ppc:co}.

\paragraph{Results} We used Alloy to search for C11/Power compilation
bugs involving software-level executions of up to four events and
architecture-level executions of up to eight events. (Beyond this, the
SAT solver timed out.) We found no bugs.

We also used Alloy to search for bugs in two other, broadly similar,
C11 compilation schemes. The target memory models, which cannot be
named, are currently being developed by two major chip companies.
Alloy found a bug in one of the models, which led us to the
observation that the model was failing to include control dependencies
(though we have been unable to confirm this with the developers as an
actual bug).

\subsection{Results for initial writes / predicate calculus}

\begin{center}
\begin{tikzpicture}
\begin{semilogxaxis}[
  legend pos=south east,
  xtickten={3,4,5},
  ytick={1,3,5},
  x label style={at={(axis description cs:0,0.335)},anchor=north west},
  ylabel near ticks,
  ylabel={Task},
  xlabel={Solve time /s:},
  minor tick num=0,
  height=3cm,
  width=9.1cm,
]
\addplot[only marks, mark size=0pt]
plot [error bars/.cd, x dir=both, x explicit, error bar style={line
width=3pt, draw=green}] 
table [y=task, x=init, x error minus expr=\thisrow{init}-\thisrow{initmin}, x
error plus expr=\thisrow{initmax}-\thisrow{init}, col sep=comma]
{../models/tests/results_init_exp/Sheet 1-Table 2.csv};

\addplot[only marks, mark size=0pt]
plot [error bars/.cd, x dir=both, x explicit, error bar style={line
width=3pt, draw=red}] 
table [y expr=\thisrow{task}-0.5, x=normal, x error minus expr=\thisrow{normal}-\thisrow{normalmin}, x
error plus expr=\thisrow{normalmax}-\thisrow{normal}, col sep=comma]
{../models/tests/results_init_exp/Sheet 1-Table 2.csv};

\addplot[only marks, mark size=0pt]
plot [error bars/.cd, x dir=both, x explicit, error bar style={line
width=3pt, draw=blue}] 
table [y expr=\thisrow{task}+0.5, x=exp, x error minus
expr=\thisrow{exp}-\thisrow{expmin}, x
error plus expr=\thisrow{expmax}-\thisrow{exp}, col sep=comma]
{../models/tests/results_init_exp/Sheet 1-Table 2.csv};
%\legend{initial writes, normal, predicate}
\end{semilogxaxis}
\end{tikzpicture}
\end{center}

\subsection{Other nice things about Alloy}

Alloy has several qualities that
make it ideal, we believe, for this kind of work:

\begin{description}

\item[Modularity] Alloy's type system supports subtyping, which we
exploit by having, for instance, \emph{x86 executions} inherit the
constraints of \emph{architecture-level executions}, which inherit the
constraints of \emph{executions}. This, together with Alloy's simple
module system that allows generic definitions, preserves the
readability of our models. 

\item[Debuggability] If Alloy fails, it fails `visibly' (to coin a
phrase by Padon et al.~\cite{padon+16}). That is, if Alloy produces a
bogus solution, it is typically straightforward to identify
constraints to strengthen. And if Alloy fails to find a known
solution, it helpfully highlights the minimal subset of unsatisfiable
constraints, which simplifies the subsequent search for constraints to
weaken.

\item[Maturity] Having been in development for over fifteen
years~\cite{jackson+01}, Alloy has a large user base and support
community. It is also open-source and offers an easy-to-use API. 
%
\end{description}

\subsection{Architecture-level executions}

\begin{figure}[t]
\paragraph{Sets of events ($\subseteq \E$):} ~\\
\quad\begin{tabular}{ll}
$atom$ & relates the components of an RMW
\end{tabular} \\~

\paragraph{Well-formedness constraints:}
\begin{mathpar}
\labelb{eq:hwexec:sbtot}~ \sthd \subseteq sb^? \cup (sb^{-1})^?
\and
\labelb{eq:hwexec:nona}~ nal = \emptyset
\and
\labelb{eq:hwexec:noRMW}~ R \ncap W
\and
\labelb{eq:hwexec:atom}~ atom \subseteq (R\times W) \cap \imm(sb)
\end{mathpar}
\caption{Architecture-level executions, $\ExecH$ (extending $\Exec$)}
\label{fig:hw_exec}
\end{figure}

An architecture-level execution is a structure that inherits all the
fields and well-formedness conditions from basic executions, and
adds those listed in Fig.~\ref{fig:hw_exec}. Since the structure of
assembly code does not permit ambiguous evaluation orders,
sequenced-before becomes a total relation within each
thread~\refb{eq:hwexec:sbtot}. There is no concept of a non-atomic
location at the architecture level~\refb{eq:hwexec:nona}. It is
commonplace in architecture-level MCMs to separate the read and the
write of an RMW into separate events~\refb{eq:hwexec:noRMW}, and
connect them with the $atom$
relation~\refb{eq:hwexec:atom}~\cite{alglave+14}. This differs from
the C11 approach, where an RMW is a single event belonging to both $R$
and $W$.

\subsection{Relation calculus vs. predicate calculus}

Alloy supports concise relation calculus expressions like
$irreflexive(rf\join hb)$ and more powerful predicate calculus
expressions like $\forall e_1,e_2\ldotp (e_1,e_2)\in rf \Rightarrow
(e_2,e_1)\notin hb$. We measured the impact on solving time when a few
tasks (1, 3, and 5) were converted to use the latter style, and found
it to be negligible. 

\subsection{Bit where we work through a few examples of `generating
litmus tests'}

The following example illustrates
how this technique can obtain conformance tests (\Q1) for the C11
memory model.

\begin{Example}
Consider the following C11 execution, in which ${\rm W}_{"rlx"}$ and
${\rm R}_{"rlx"}$ denote atomic write and read events with `relaxed'
memory order.
%
\begin{equation}
\label{exec:goodtest}
\begin{tikzpicture}[inner sep=1pt, baseline=(a.base)]
\node (a) at (2,0.8) {$\evR{"rlx"}{"x"}{1}$};
\node (b) at (2,0) {$\evR{"rlx"}{"x"}{0}$};
\node (c) at (0,0.8) {$\evW{"rlx"}{"x"}{1}$};
\draw[edgerf] (c) to [auto] node {$rf$} (a);
\draw[edgesb] (a) to [auto] node {$sb$} (b);
\node[sthd, fit=(a)(b)] {};
\node[sthd, fit=(c)] {};
\end{tikzpicture}
\end{equation}
The execution is inconsistent in C11, because of a coherence violation. It can be
reverse-engineered into a litmus test
\begin{equation*}
\litmustestII{atomic\_int x=0;}{
\texttt{x.store(1,RLX);} & \texttt{r0=x.load(RLX);} \\
                         & \texttt{r1=x.load(RLX);} \\
}{r0==1 \&\& r1==0}
\end{equation*}
that is both useful and conclusive.
\end{Example}

Unfortunately, litmus tests generated in this way are not always
\emph{conclusive}. If a particular execution is \emph{allowed} under a
memory model, we can easily construct a litmus test that \emph{can
pass}, but if a particular execution is \emph{forbidden} under a
memory model, it is not straightforward to construct a litmus test
that \emph{must fail}, because the constructed litmus test may still
be able to pass via a different execution that is allowed. The
following two examples demonstrate why it is not enough for the
execution merely to be inconsistent (and hence why we further impose
deadness).

\begin{Example}
\label{ex:badtest_racy}
Consider the following C11 execution, in which "na" indicates events
corresponding to non-atomic operations.
\begin{equation}
\label{exec:badtest_racy}
\begin{tikzpicture}[inner sep=1pt, baseline=(a.base)]
\node (a) at (0,0) {$\evW{"na"}{"a"}{1}$};
\node (b) at (2,0) {$\evR{"na"}{"a"}{1}$};
\draw[edgerf] (a) to [auto] node {$rf$} (b);
\node[sthd, fit=(a)] {};
\node[sthd, fit=(b)] {};
\end{tikzpicture}
\end{equation}
It is inconsistent in C11 (because a read of a non-atomic location
must only observe a write that happens before it), but the litmus test
obtained from it
\begin{equation*}
\litmustestII{int a=0;}{
\texttt{a=1;} & \texttt{r0=a;} \\
}{r0==1}
\end{equation*}
is not conclusive because it admits the execution
\begin{equation*}
\begin{tikzpicture}[inner sep=1pt]
\node (a) at (0,0) {$\evW{"na"}{"a"}{1}$};
\node (b) at (2,0) {$\evR{"na"}{"a"}{0}$};
%\draw[edgerf] (a) to [auto] node {$rf$} (b);
\node[sthd, fit=(a)] {};
\node[sthd, fit=(b)] {};
\end{tikzpicture}
\end{equation*}
%
which is both consistent and racy. (Any behaviour is allowed for a
program that admits a consistent and racy execution.) \end{Example}

\begin{Example} 
\label{ex:badtest_co}
Consider the following C11 execution, in which $co$
denotes the coherence order on writes.
\begin{equation}
\label{exec:badtest_co}
\begin{tikzpicture}[inner sep=1pt, baseline=(a.base)]
\node (a) at (0,0.8) {$\evtlbl{$a$}\evW{"rlx"}{"x"}{2}$};
\node (b) at (0,0) {$\evtlbl{$b$}\evW{"rlx"}{"x"}{1}$};
\node (c) at (2.5,0.8) {$\evtlbl{$c$}\evW{"rlx"}{"x"}{3}$};
\node (d) at (2.5,0) {$\evtlbl{$d$}\evR{"rlx"}{"x"}{3}$};
\draw[edgerf] (c) to [auto, bend right, swap] node {$rf$} (d);
\draw[edgesb] (a) to [auto, bend left] node {$sb$} (b);
\draw[edgesb] (c) to [auto, bend left] node {$sb$} (d);
\draw[edgemo] (b) to [auto, bend left] node {$co$} (a);
\draw[edgemo] (a) to [auto] node {$co$} (c);
\node[sthd, fit=(a)(b)] {};
\node[sthd, fit=(c)(d)] {};
\end{tikzpicture}
\end{equation}
It is inconsistent in C11 (because of a coherence violation), but
the litmus test obtained from it
\begin{equation*}
\litmustestII{atomic\_int x=0;}{
\texttt{x.store(2,RLX);} & \texttt{x.store(3,RLX);} \\
\texttt{x.store(1,RLX);} & \texttt{r0=x.load(RLX);} \\
}{x==3 \&\& r0==3}
\end{equation*}
is not conclusive because its final state can be obtained via
a consistent execution that simply reverses the $co$ edge
in~\eqref{exec:badtest_co} from $(b,a)$ to $(a,b)$. 
\end{Example}

The purpose of the `$dead$' condition in
Def.~\ref{def:general_problem_executions} is to rule out executions
like \eqref{exec:badtest_racy} that give rise to litmus tests that can
pass because they are racy, as well as executions like
\eqref{exec:badtest_co} whose litmus tests can pass via another
consistent execution, but to permit executions
like~\eqref{exec:goodtest}. With a suitable definition of `$dead$'
(\S\ref{sec:safety}), and careful restrictions on the target
programming language (\S\ref{sec:language}), our approach ensures that
if $(X,Y)$ solves $\GeneralExec$, and if $X$ and $Y$ are maximal
candidate executions of programs $P$ and $Q$ (respectively) that reach
$\sigma$, then $(P,Q,\sigma)$ solves the original $\GeneralProg$
problem, as required. We find that obtaining $(P,Q,\sigma)$ in this
indirect way is a much more efficient strategy than solving
$\GeneralProg$ directly.

\subsection{Constructing coherence}
\label{sec:sezgin}

Sezgin~\cite{sezgin04} has asked whether $co$ can constructed from the
other relations of an execution, such as $sb$ and $rf$, rather than
simply declared to exist. He has shown that an attempt to characterise
SC executions in this way falls short, by exhibiting a 16-event
execution~\cite[Fig.~4.3]{sezgin04} that is judged SC according to the
`constructed $co$' approach, but is nonetheless not SC. We encoded his
$co$ construction into Alloy and sought a discrepancy with the SC
MCM. Unfortunately, Sezgin's 16-event example proved to be out of
Alloy's reach. This problem is much more computationally expensive
than the problem considered in \S\ref{sec:kyndylan}, because to show
that an execution is \emph{not} SC, we must universally quantify over
all possible $co$ relations.


\subsection{Handling (mutually) recursive definitions}
\label{sec:fixpoints}

The axiomatic Power MCM derives four relations (called $ic$, $ii$,
$ci$ and $cc$) via mutual recursion. In the ".cat"
format~\cite{alglave+14}, one can write `"let rec $ic = f_0(ic,ii,ci,cc)$ and $ii =
f_1(ic,ii,ci,cc)$ and $ci = f_2(ic,ii,ci,cc)$ and $cc = f_3(ic,ii,ci,cc)$"', but Alloy does not provide recursive
definitions. Accordingly, we expand the recursive construction explicitly, using the constraints
\begin{mathpar}
\labelb{ppcrec:0}~
f_0(ic,ii,ci,cc) \subseteq ic
\and
\labelb{ppcrec:1}~
f_1(ic,ii,ci,cc) \subseteq ii
\and
\labelb{ppcrec:2}~
f_2(ic,ii,ci,cc) \subseteq ci
\and
\labelb{ppcrec:3}~
f_3(ic,ii,ci,cc) \subseteq cc
\and
\labelb{ppcrec:all}~
\stack{\forall ic',ii',ci',cc'\ldotp\\
\quad (f_0(ic',ii',ci',cc') \subseteq ic' \wedge
f_1(ic',ii',ci',cc') \subseteq ii' \wedge {}\\
\quad \phantom(f_2(ic',ii',ci',cc') \subseteq ci' \wedge
f_3(ic',ii',ci',cc') \subseteq cc') \Rightarrow {}\\
\quad\quad (ci \subseteq ci' \wedge ii \subseteq ii' \wedge cc \subseteq cc' \wedge ic \subseteq ic')}
\end{mathpar}
%
to express that $(ic,ii,ci,cc)$ is both a
pre-fixpoint~\refb{ppcrec:0}~\refb{ppcrec:1}~\refb{ppcrec:2}~\refb{ppcrec:3}
and the least such~\refb{ppcrec:all}. This last constraint involves
universal quantification over relations and hence requires the
higher-order AlloyStar solver. A first-order alternative is simply to
unroll the recursive definitions a few times; that is, to define
$ic = f_0^k(\emptyset,\emptyset,\emptyset,\emptyset)$ (and $ii$, $ci$,
and $cc$ similarly) for a fixed $k$. We found, for a search scope of 4
software events and 8 hardware events, that $k\ge 2$ is sufficient for
avoiding false positives. Figure~\ref{fig:fixpoint_results} shows that
the proper fixpoint construction (i.e., $k=\infty$) is considerably
more expensive than a fixed unrolling.

\subsection{MP bug in OpenCL/AMD}

\begin{figure}
\newcommand\ST[1]{\begin{array}{@{}c@{}}#1\end{array}}
\newcommand\LG[2]{{$\scriptsize\left(\ST{\sigma_{#1}\\
\sigma_{#2}}\right)$}}
\begin{tikzpicture}[inner sep=1pt]

\node[event, anchor=west](a) at (-4.9,0) 
{$\evW{\na}{"x"}{1}$};

\node[event, anchor=west](b) at (-4.9,-0.8) 
{$\evW{\morel,\msdv}{"y"}{1}$};

\node[event, anchor=west](c) at (-3.3,0) 
{$\evR{\moacq,\msdv}{"y"}{1}$};

\node[event, anchor=west](d) at (-3.3,-0.8) 
{$\evR{\na}{"x"}{0}$};

\node[sthd, fit=(a)(b)] {};
\node[sthd, fit=(c)(d)] {};

\draw[edgerf] (b) to[auto, pos=0.4] node{$rf$} (c);
\foreach \i/\j in {a/b}
\draw[edgesb] ([xshift=8mm]\i.south west) to[auto,swap,pos=0.4]
node{$sb$} ([xshift=8mm]\j.north west -| \i.south west);
\foreach \i/\j in {c/d}
\draw[edgesb] ([xshift=8mm]\i.south west) to[auto,swap,pos=0.6]
node{$sb$,$cd$} ([xshift=8mm]\j.north west -| \i.south west);

\node (inv1) at (2.3,0) {\LG04$\evInv$\LG04};
\node (efet1) [below=4mm of inv1] {\LG04$\evEFet{"x"}$\LG34};
\node (sto1) at (0,0) {\LG04$\evW{}{"x"}{1}$\LG14};
\node (eflu1) [below=4mm of sto1] {\LG14$\evEFlu{"x"}$\LG25};
\node (flu1) [below=4mm of eflu1] {\LG25$\evFlu$\LG25};
\node (sto2) [below=4mm of flu1] {\LG25$\evW{}{"y"}{1}$\LG65};
\node (eflu2) [below=4mm of sto2] {\LG65$\evEFlu{"y"}$\LG77};
\node (efet2) [below=4mm of efet1] {\LG37$\evEFet{"y"}$\LG87};
\node (loa1) [below=4mm of efet2] {\LG87$\evR{}{"y"}{1}$\LG87};
\node (loa2) [below=4mm of loa1] {\LG87$\evR{}{"x"}{0}$\LG87};

\draw[edgerf] (sto2) to[auto, pos=0.1, bend left=40] node[inner sep=0]{$rf$} (loa1);

\coordinate (wg1nw) at ([yshift=2mm]sto1.north west);
\coordinate (wg1ne) at ([yshift=2mm]sto1.north east);
\coordinate (wg1sw) at ([yshift=-2mm]eflu2.south west);
\coordinate (wg1se) at ([yshift=-2mm]eflu2.south east);

\coordinate (wg2nw) at ([yshift=2mm]inv1.north west);
\coordinate (wg2ne) at ([yshift=2mm]inv1.north east);
\coordinate (wg2sw) at ([yshift=-2mm]loa2.south west);
\coordinate (wg2se) at ([yshift=-2mm]loa2.south east);

\node[sthd_tight, fit=(sto1)(eflu2)] {};
\node[sthd_tight, fit=(inv1)(loa2)] {};

\draw[edgepi, overlay] (a) to[auto, out=30, in=180,pos=0.8] node{$\pi$} (sto1);
\draw[edgepi] (b) to[auto, out=330, in=180,pos=0.4] node{$\pi$} (flu1);
\draw[edgepi] (b) to[auto, out=330, in=160] (sto2);
\draw[edgepi] (c) to[auto, out=350, in=200, swap, pos=0.2] node{$\pi$} (inv1);
\draw[edgepi] (c) to[auto, out=350, in=120] (loa1);
\draw[edgepi] (d) to[auto, bend right=10, pos=0.1] node{$\pi$} (loa2);

\draw[edgethen] (inv1) to (efet1);
\draw[edgethen] (efet1) to[out=175,in=360] (sto1);
\draw[edgethen] (sto1) to (eflu1);
\draw[edgethen] (eflu1) to (flu1);
\draw[edgethen] (flu1) to (sto2);
\draw[edgethen] (sto2) to (eflu2);
\draw[edgethen] (eflu2) to[out=0,in=180] (efet2);
\draw[edgethen] (efet2) to (loa1);
\draw[edgethen] (loa1) to (loa2);

\node[anchor=north west] at (-5.0,-1.25) {$\begin{array}{@{}r@{\,}c@{\,}l@{}} 
\sigma_0 &=& \{\} \\
\sigma_1 &=& \{\ST{"x"\mapsto_{\rm dv}1}\} \\
\sigma_2 &=& \{\ST{"x"\mapsto_{\rm cv}1}\} \\
\sigma_3 &=& \{\ST{"x"\mapsto_{\rm cv}0}\} \\
\sigma_4 &=& \{\ST{"x"\mapsto_{\rm cv}0, 
                   "y"\mapsto_{\rm cv}0}\} \\
\sigma_5 &=& \{\ST{"x"\mapsto_{\rm cv}1, 
                   "y"\mapsto_{\rm cv}0}\} \\
\sigma_6 &=& \{\ST{"x"\mapsto_{\rm cv}1, 
                   "y"\mapsto_{\rm dv}1}\} \\
\sigma_7 &=& \{\ST{"x"\mapsto_{\rm cv}1, 
                   "y"\mapsto_{\rm cv}1}\} \\
\sigma_8 &=& \{\ST{"x"\mapsto_{\rm cv}0, 
                   "y"\mapsto_{\rm cv}1}\}
\end{array}$};
\end{tikzpicture}
\caption{Message-passing bug in OpenCL/AMD compilation}
\label{fig:mp_bug}
\end{figure}

See Fig.~\ref{fig:mp_bug}.

\subsection{Other related work}

\begin{itemize}

\item Horn et al.~\cite{horn+15} have a paper that is something to do
with SAT-solving and weak memory, and also mentions the N-free
property.

\item Lau et al. have looked at how to reverse-engineer a program
by examining executions~\cite{lau+03}.

\item Maybe mention future application to crash-consistency models
(Bornholt et al, ASPLOS 16, http://homes.cs.washington.edu/~bornholt/papers/ferrite-asplos16.pdf)

\item We remark that consistency models for distributed systems often
resemble those for memory systems, and we expect that our comparison
techniques may prove helpful in that domain too -- potentially using
Cerone et al.'s axiomatic treatment of various notions of consistency
as a starting point~\cite{cerone+15}.

\item Regarding the SC-DRF guarantee (\Q2): Batty et al. have proved it
manually for C11, while Bouajjani et al.~\cite{bouajjani+11} have
analysed the complexity of automatically checking it for the TSO
memory model~\cite{owens+09}.

\item Jade's SFM lecture notes?~\cite{alglave15}

\end{itemize}


\end{document}

