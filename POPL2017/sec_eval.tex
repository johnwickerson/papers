\section{Performance Evaluation}
\label{sec:eval}

In this section, we report on an empirical investigation of how our
design decisions affect Alloy's SAT-solving performance.

\begin{table}
\newcommand\Gluc{{\sf G}}
\newcommand\Plin{{\sf P}}
\newcommand\Mini{{\sf M}}
\newcommand\TICK{\textcolor{Red}{\cmark}}
\newcommand\CROSS{\textcolor{Green}{\xmark}}
\centering
\renewcommand\tabcolsep{0.9mm}
\begin{tabularx}{\linewidth}{@{\,}rX@{}rrrcc@{\,}}
\toprule
\multicolumn{2}{@{\,}l}{Task} & $\lvert\E\rvert$ & $t_{\rm enc} / {\rm s}$ & 
$t_{\rm sol} / {\rm s}$ & & \\
\midrule
1 & \Q2 \mm{C11-SRA} vs. \mm{C11} (\S\ref{sec:Q2_c11_sra_simp}) & $6$ 
& $0.7$ & $0.6\phantom{0}$ & \Gluc & \TICK \\
2 & \Q2 \mm{C11-SwRf} vs. \mm{C11} (\S\ref{sec:kyndylan}) & $7$ 
& $0.8$ & $625\phantom{.00}$ & \Gluc & \CROSS \\
3 & \Q2 \mm{C11-SwRf} vs. \mm{C11} (\S\ref{sec:kyndylan}) & $12$ 
& $2\phantom{.0}$ & $214\phantom{.00}$ & \Plin & \TICK \\
4 & \Q2 \mm{C11} vs. \mm{C11-Orig} (\S\ref{sec:Q2_c11_simp_orig})&$5$
& $0.4$ & $0.3\phantom{0}$ & \Gluc & \TICK \\
5 & \Q2 \mm{MCA} vs. \mm{x86} (\S\ref{sec:Q2_mca}) & $9$ 
& $0.8$ & $607\phantom{.00}$ & \Plin & \CROSS \\
6 & \Q2 \mm{MCA} vs. \mm{Power} (\S\ref{sec:Q2_mca}) & $6$ 
& $2\phantom{.0}$ & $0.06$ & \Gluc & \TICK \\
7 & \Q2 \mm{SC} vs. \mm{C11-Draft} (\S\ref{sec:scdrf}) & $4$ 
& $0.4$ & $0.04$ & \Gluc & \TICK \\
8 & \Q2 \mm{PTX2} vs. \mm{PTX1} (\S\ref{sec:Q4_opencl_ptx}) & $7$ 
& $0.7$ & $4\phantom{.00}$ & \Gluc & \TICK \\
9 & \Q3 \mm{C11} (sequencing) (\S\ref{sec:Q3_c11_seq}) & $5$ 
& $0.5$ & $163\phantom{.00}$ & \Mini & \CROSS \\
10 &  \Q3 \mm{C11} (sequencing) (\S\ref{sec:Q3_c11_seq}) & $6$ 
& $0.7$ & $5\phantom{.00}$ & \Plin & \TICK \\
11 & \Q3 \mm{C11} (mem. orders) (\S\ref{sec:Q3_c11_mo}) & $7$ 
& $0.9$ & $51\phantom{.00}$ & \Gluc & \TICK \\
12 & \Q4 \mm{C11} / \mm{x86} & $5+5$ 
& $0.7$ & $13029\phantom{.00}$ & \Plin & \CROSS \\
13 & \Q4 \mm{C11} / \mm{Power} (\S\ref{sec:Q4_c11_power}) & $6+8$ 
& $8\phantom{.0}$ & $91\phantom{.00}$ & \Plin & \TICK \\
14 & \Q4 \mm{OpenCL} / \mm{AMD} (\S\ref{sec:Q4_opencl_amd}) & $2+9$ 
& $6\phantom{.0}$ & $1355\phantom{.00}$ & \Gluc & \TICK \\
15 & \Q4 \mm{OpenCL} / \mm{AMD} (\S\ref{sec:Q4_opencl_amd}) & $4+10$ 
& $16\phantom{.0}$ & $4743\phantom{.00}$ & \Plin & \TICK \\
16 & \Q4 \mm{OpenCL} / \mm{PTX1} (\S\ref{sec:Q4_opencl_ptx}) & $5+8$ 
& $2\phantom{.0}$ & $11\phantom{.00}$ & \Plin & \TICK \\
17 & \Q4 \mm{OpenCL} / \mm{PTX2} (\S\ref{sec:Q4_opencl_ptx}) & $5+15$
& $4\phantom{.0}$ & $9719\phantom{.00}$ & \Plin & \CROSS 
\end{tabularx}
%
\caption{Summary of tasks. We record the number ($\lvert\E\rvert$) of
events in the search space, the time taken (in seconds) to encode ($t_{\rm enc}$)
and then solve ($t_{\rm sol}$) the SAT query, whether the fastest
solver was MiniSat (\Mini), Glucose (\Gluc), or Plingeling (\Plin),
and whether a solution was found (\TICK) or not (\CROSS). For \Q4
tasks, $m+n$ means that $\E$ is partitioned into $m$ software-level and $n$ architecture-level events.}

\label{tab:tasks}
\end{table}

Table~\ref{tab:tasks} summarises the tasks on which we have evaluated
our technique. We used a machine with four 16-core 2.1 GHz AMD
Opteron processors and 128 GB of RAM.

%%%%%
% C11/Power results (6 SE + 8 HE)
% Without simplified compiler: 25383s (7 hours)
% With:
% Plingeling: 55s (+8s of processing), 138s, 70s, 102s
% Glucose: 1723s, 2274s, 2600s, 2364s 
% Minisat: 171s, 146s, 168s, 149s
%%%%%

\tikzset{
  glucosebar/.style={draw=none, fill=white!30!red},
  minisatbar/.style={draw=none, fill=white!30!blue},
  plingelingbar/.style={draw=none, fill=white!30!green},
}

\subsection{Choice of SAT Solver}

Figure~\ref{fig:solver_results} summarises the performance of three
SAT solvers on our tasks: Glucose (version~2.1)~\cite{audemard+09},
MiniSat (version~2.2)~\cite{een+03}, and Plingeling~\cite{biere10}. Each bar shows
the mean solve time over 4 runs, plus the minimum and maximum
time.\footnote{A more thorough comparison of SAT solvers would control for the
order of clauses, which greatly influences
performance~\cite{nikolic10}.}
Plingeling is able to complete all tasks, whereas Glucose and MiniSat
time out on three and five of them respectively. On the tasks all
three solvers complete, MiniSat takes an average of 9 minutes, Glucose
takes 8, and Plingeling takes 6. Plingeling's relatively high startup
overhead is evident on the quicker tasks.

% 12 tasks where all solvers complete: 1,2,4,5 ,6,7,9 ,10,11,13,14,16
% Averages: Glucose=8min, Plingeling=6min, Minisat=9min.

\begin{figure}

\def\mybarwidth{1.5pt}
\def\timeout{14400}

\centering

\begin{tikzpicture}
\begin{semilogyaxis}[
  legend pos=north west,
  xtick={1,...,17},
  xticklabels={1,2,3,4,5,6,7,8,9,10,11,12,13,14,15,16,17},
  xmin=0.3,
  xmax=17.7,
  x label style={at={(axis description cs:-0.14,0.155)},anchor=north
  west},
  log ticks with fixed point,
  ytick={1,100,10000},
  yticklabels={$1$, $100$, \llap{$10000$}},
  ylabel near ticks,
  xlabel={Task:},
  ylabel={Solve time /s\hspace*{2mm}},
  xtick style={draw=none},
  ybar=0.5pt,
  bar width=\mybarwidth,
  log origin=infty,
  height=4cm,
  width=8.9cm,
  legend columns=-1,
]
\addplot[minisatbar]
plot [error bars/.cd, y dir=both, y explicit, error mark options={mark
size=\mybarwidth/2, rotate=90}] 
table [x=x, y=y, y error minus expr=\thisrow{y}-\thisrow{ymin}, y
error plus expr=\thisrow{ymax}-\thisrow{y}, col sep=comma]
{MiniSat.csv};

\addplot[glucosebar]
plot [error bars/.cd, y dir=both, y explicit, error mark options={mark
size=\mybarwidth/2, rotate=90}] 
table [x=x, y=y, y error minus expr=\thisrow{y}-\thisrow{ymin}, y
error plus expr=\thisrow{ymax}-\thisrow{y}, col sep=comma]
{Glucose.csv};

\addplot[plingelingbar]
plot [error bars/.cd, y dir=both, y explicit, error mark options={mark
size=\mybarwidth/2, rotate=90}] 
table [x=x, y=y, y error minus expr=\thisrow{y}-\thisrow{ymin}, y
error plus expr=\thisrow{ymax}-\thisrow{y}, col sep=comma]
{Plingeling.csv};

\draw [black!50, dashed] ({rel axis cs:0,0}|-{axis cs:12,\timeout}) -- ({rel axis
cs:1,0}|-{axis cs:12,\timeout}) node [inner sep=0pt,pos=0.55,
below=0.3mm] {\scriptsize 4-hour timeout};

%\legend{MiniSat\hspace*{2mm}, Glucose\hspace*{2mm}, Plingeling}
\end{semilogyaxis}
\end{tikzpicture}
\def\fooo{MiniSat 
(\,{\tikz\draw[minisatbar,/tikz/.cd,yshift=-0.25em]
        (0cm,0cm) rectangle (\mybarwidth,0.8em);}\,), Glucose
(\,{\tikz\draw[glucosebar,/tikz/.cd,yshift=-0.25em]
        (0cm,0cm) rectangle (\mybarwidth,0.8em);}\,) and Plingeling (\,{\tikz\draw[plingelingbar,/tikz/.cd,yshift=-0.25em]
        (0cm,0cm) rectangle (\mybarwidth,0.8em);}\,)
}
\vspace*{-3mm}
\caption{Comparing \protect\fooo} 
\label{fig:solver_results}
\end{figure}
 

\subsection{Combined vs. Separate Event Sets}

For \Q4 tasks (\S\ref{sec:compilers}), we can choose to draw
software-level and architecture-level events either from a single set
$\E$ or from disjoint partitions $\SE$ and $\HE$. The former approach
is attractive because it gives the user one fewer parameter to
control, and it minimises the total number of events required to find
solutions because a single element of $\E$ can represent both a
software-level event and an architecture-level event. However, as
shown in Tab.~\ref{tab:tasks}, we choose the latter approach. To see
why, consider Task 17. To validate the OpenCL/PTX mapping for OpenCL
executions of up to 5 events, we must consider PTX executions of up to
15 events (since each OpenCL event can map to up to three PTX events).
We found that setting $\lvert\SE\rvert = 5$ and $\lvert\HE\rvert = 15$
led to a tractable constraint-solving problem for Alloy, but that
merely setting $\lvert\E\rvert = 15$, which involves fewer events in
total but includes OpenCL executions with more than 5 events, rendered the problem insoluble.

\subsection{Recursive Definitions vs. Fixed Unrolling}
\label{sec:fixpoints}

\begin{figure}

\def\mybarwidth{6.5pt}

\centering

\begin{tikzpicture}
\begin{semilogyaxis}[
  legend pos=north west,
  xtick={2,3,4,5,6,8},
  xticklabels={$2$,$3$,$4$,$5$,$6$,$\infty$},
  xmin=1.5,
  xmax=8.5,
  log ticks with fixed point,
  ytick={200,1000,5000},
  yticklabels={$200$, $1000$, $5000$},
  x label style={at={(axis description cs:-0.02,0.23)},anchor=north west},
  ylabel near ticks,
  xlabel={$k$:},
  ylabel={Solve time /s},
  xtick style={draw=none},
  ybar=0.5pt,
  bar width=\mybarwidth,
  log origin=infty,
  height=3.3cm,
  width=8.7cm,
  legend columns=-1,
]
\addplot[glucosebar]
plot [error bars/.cd, y dir=both, y explicit, error mark options={mark
size=\mybarwidth/2, rotate=90}] 
table [x expr=\thisrow{k}+1, y=mean, y error minus expr=\thisrow{mean}-\thisrow{min}, y
error plus expr=\thisrow{max}-\thisrow{mean}, col sep=comma]
{Fixpoints.csv};
\end{semilogyaxis}
\end{tikzpicture}
\vspace*{-3mm}
\caption{Performance of Task 13 with fixpoints unrolled $k$ times} 
\label{fig:fixpoint_results}
\end{figure}

Alglave et al.'s formalisation of the Power MCM in the ".cat" format
involves several recursively-defined relations, but Alloy does not
support definitions of the form `"let rec $r = f(r)$"'. Accordingly,
in our formalisation of the Power MCM~\cite{popl17supplementary}, we
expand the recursive construction explicitly, by requiring the
existence of an $r$ satisfying $f(r) \subseteq r$ and
$\forall r'\ldotp f(r')\subseteq r' \Rightarrow r \subseteq r'$. The
latter constraint involves universal quantification over relations and
hence requires the higher-order AlloyStar solver~\cite{milicevic+15}.
A first-order alternative is simply to unroll the recursive
definitions a few times; that is, to set $r \eqdef f^k(\emptyset)$ for a
fixed $k$. We found, for the small search scopes involved in our work,
that $k\ge 2$ is sufficient for avoiding false positives when checking
a compiler mapping from C11 (Task~13).
Figure~\ref{fig:fixpoint_results} shows that the proper fixpoint
construction (i.e., $k=\infty$) is much more expensive than a
fixed unrolling.


%%% Local Variables:
%%% mode: latex
%%% TeX-master: "paper"
%%% End:
