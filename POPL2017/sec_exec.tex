\section{Executions}
\label{sec:executions}

The semantics of a program is a set of \emph{executions}. This section
describes our formalisation of executions, both in general
(\S\ref{sec:form_executions}) and specifically for C11, OpenCL, and
PTX (\S\ref{sec:language-specific-executions}), and then
explains how MCMs decide which executions are allowed
(\S\ref{sec:c11_consistency}).

Program executions are composed of \emph{events}, each representing
the execution of a program instruction. Most existing MCM frameworks
embed several pieces of information within each event, such as the
location it accesses, the thread it belongs to, and the value it reads
and/or writes (e.g.,~\cite{adve+90, alglave+10, batty+11, morisset+13,
sarkar+09}). In our work, events are \emph{pure}, in Needham's
sense~\cite{needham89}: they are given meaning simply by their
membership of, for instance, the `read events' set, or the `same
location' relation. This formulation brings three benefits. First, it
means that we can easily build a hierarchy of executions to unify
several levels of abstraction. For instance, starting with a top-level
`execution' class, we can obtain a class of `C11 executions' just by
adding extra fields such as `the set of events with acquire
semantics'; we need not modify the type of events. Second, it means
that the same events can appear (with different meanings) in two
executions, thus reducing the total number of events needed in our
search space. Third, we avoid the need to define (and therefore, in a
bounded search query, set the number of) locations, threads, and
values.

\paragraph{Notation} Our MCMs are written in Alloy's modelling
language, but in this paper we opt for more conventional set-theoretic
notation. For a binary relation $r$, $\comp{r}$ is its complement,
$r^{-1}$ is its inverse, $r^?$ is its reflexive closure, $r^+$ is its
transitive closure, and $r^*$ is its reflexive, transitive
closure. The $\spo$ predicate holds of binary relations that are
acyclic and transitive, and $\equivalence(r,s)$ holds when $r$ is a
subset of $s^2$, reflexive over $s$, symmetric, and transitive. We
compose an $m$-ary relation $r_1$ with an $n$-ary relation
$r_2$ (where $m,n\ge 1$) via
$r_1\join r_2 \eqdef \{(x_1,\dots, x_{m-1},z_1,\dots, z_{n-1})\mid \exists
y\ldotp (x_1,\dots, x_{m-1}, y)\in r_1 \wedge (y,z_1,\dots, z_{n-1})\in
r_2\}$,
and we lift a set to a subset of the identity relation via
$\stor{s} \eqdef \{(e,e)\mid e\in s\}$. We write $\imm(r)$ for
$r-(r\join r^+)$, and $s_1\ncap s_2$ for $s_1\cap s_2 = \emptyset$.

\begin{figure}[t]
\begin{tabular}{@{}l@{~}l@{~~~~~~~~}l}
$E$ & $\subseteq \E$ & all events in the execution \\
$R$, $W$ & $\subseteq \E$ & events that read (resp. write) a location \\
$F$ & $\subseteq \E$ & fence events \\
$nal$ & $\subseteq \E$ & events that access a non-atomic
location \\
$sb$ & $\subseteq \E^2$ & sequenced-before \\
$ad$, $cd$, $dd$ & $\subseteq \E^2$ & address/control/data dependency \\
$\sthd$, $sloc$ & $\subseteq \E^2$ & same thread, same location \\
$rf$, $co$ & $\subseteq \E^2$ & reads-from, coherence order \\
$rfe$ & $\subseteq \E^2$ & $\eqdef rf - sthd$
\end{tabular}
%\par\vspace*{0.5mm}\rule{\linewidth}{0.4pt}
\begin{mathpar}
\labelb{eq:basicexec:ev}~ R \cup W \cup F \cup nal \subseteq E
\and
\labelb{eq:basicexec:fences}~ (R\cup W) \ncap F
\and
\labelb{eq:basicexec:sbthd}~ sb \subseteq \sthd
\and
\labelb{eq:basicexec:sbspo}~ \spo(sb)
\and
\labelb{eq:basicexec:ad}~ ad \subseteq \stor{R}\join
sb\join\stor{R\cup W}
\and
\labelb{eq:basicexec:dd}~ dd \subseteq \stor{R}\join sb\join\stor{W}
\and
\labelb{eq:basicexec:cd}~ cd \subseteq \stor{R}\join sb
\and
\labelb{eq:basicexec:thdequiv}~ \equivalence(\sthd, E)
\and
\labelb{eq:basicexec:locequiv}~ \equivalence(\sloc, R\cup W)
\and
\labelb{eq:basicexec:locnal}~ nal \join \sloc = nal
\and
\labelb{eq:rfdom}~ rf \subseteq (W \times R) \cap \sloc
\and
\labelb{eq:rfunique}~ rf \join rf^{-1} \subseteq \id
\and
\labelb{eq:cospo}~ \spo(co)
\and
\labelb{eq:cotot}~ co \cup co^{-1} = (W - nal)^2 \cap \sloc - \id 
\end{mathpar}\par\vspace*{-2mm}
\caption{Basic executions, $\Exec$}
\label{fig:basic_exec}
\end{figure}

\subsection{Basic Executions}
\label{sec:form_executions}
Let $\E$ be a set of events. 

\begin{definition}[Basic executions] 
%
The set $\Exec$ of basic executions comprises structures with fourteen
fields, governed by well-formed\-ness constraints
(Fig.~\ref{fig:basic_exec}). We write $f_X$ for the field $f$ of
execution $X$, omitting the subscript when it is clear from the
context. The constraints can be understood as follows. The subsets
$R$, $W$, $F$, and $nal$ are all drawn from the events $E$ that appear
in the execution~\refb{eq:basicexec:ev}. In particular,
compound read-modify-write (RMW) events belong to \emph{both} $R$ and $W$.
Fences are distinct from reads and writes~\refb{eq:basicexec:fences}.
\emph{Sequenced before} is an intra-thread strict partial
order~\refb{eq:basicexec:sbthd}~\refb{eq:basicexec:sbspo}. (We allow
$sb$ within a thread to be partial because in C-like languages, the
evaluation order of certain components, such as the operands of the
"+"-operator, is not specified.) Address dependencies are either
read-to-read or read-to-write~\refb{eq:basicexec:ad}, data
dependencies are read-to-write~\refb{eq:basicexec:dd}, and control
dependencies originate from reads~\refb{eq:basicexec:cd}. The $\sthd$
relation forms an equivalence among all
events~\refb{eq:basicexec:thdequiv}, while $\sloc$ forms an
equivalence among reads and writes~\refb{eq:basicexec:locequiv}. We
allow a distinction between `atomic' and `non-atomic' locations; MCMs
that do not make this distinction (such as architecture-level MCMs) simply constrain the set of
non-atomic locations to be empty. The $nal$ set consists only of
complete $\sloc$-classes~\refb{eq:basicexec:locnal}. A relation $rf$
is a well-formed \emph{reads-from} if it relates writes to reads at
the same location~\refb{eq:rfdom} and is injective~\refb{eq:rfunique}.
The inter-thread reads-from ($rfe$) is derived from $rf$. A relation
$co$ is a well-formed \emph{coherence order} if when restricted to
writes on a single atomic location it forms a strict total order; that
is, it is acyclic and transitive~\refb{eq:cospo}, and it relates every
pair of distinct writes to the same atomic location~\refb{eq:cotot}.
\end{definition}

\begin{remark}
%
We emphasise that elements of $\E$ have no intrinsic structure, only
identity. Nevertheless, when drawing executions, we tag events with
their type: ${\rm R}$ for elements of $R-W$, ${\rm W}$ for elements of
$W-R$, and ${\rm C}$ (for `compound') for elements of $R\cap W$. We
indicate $sthd$ equivalence classes with dotted rectangles, and $sloc$
equivalence classes using named representatives, e.g. "x" and "y". We
also tag events with the values read/written, but this is purely for
readability.
%
\end{remark}

\begin{remark}[Initial writes] %
\label{rem:initial_writes} %
Like some prior MCM formalisations~\cite{sezgin04,mador-haim+12}, but
unlike most (e.g.~\cite{batty+11, alglave+13, alglave+10, lahav+16}),
our executions exclude initial writes. We found that Alloy's
performance degrades rapidly as the upper bound on $\lvert \E \rvert$
increases; by avoiding initial writes, this bound can be
lowered. Removing initial writes makes $rf^{-1}$ a \emph{partial}
function, which complicates the definition of \emph{from-read}, as
described below. %
\end{remark}
%
\begin{definition}
\label{def:fromread}
\emph{From-read} relates each read to all of the writes that are
$co$-later than the write that the read observed~\cite{ahamad+93}:
%
\[
fr \eqdef ((rf^{-1} \join co) \cup fr_{\rm init}) - \id
\]
%
where
$fr_{\rm init} \eqdef (\stor{R} - (rf^{-1} \join rf)) \join \sloc
\join \stor{W}$.
In the absence of initial writes, $fr_{\rm init}$ handles reads that observe the initial value.
%
\end{definition}
%
\subsection{Language-Specific Executions} 
\label{sec:language-specific-executions}

We can obtain executions for various languages as subclasses of
$\Exec$.

\begin{figure}[t]
\begin{tabular}{@{}l@{~}l@{~~~~}l}
$A$   & $\subseteq\E$ & atomic events \\
$acq$, $rel$ & $\subseteq\E$ & events that have acquire (resp. release) semantics \\
$sc$ & $\subseteq\E$ & events that have SC semantics
\end{tabular}
%\par\vspace*{0.5mm}\rule{\linewidth}{0.4pt}
\begin{mathpar}
\labelb{eq:c11exec:a}~ acq \cup rel \cup sc \cup (R\cap W) \cup F \subseteq A \subseteq E
\and
\labelb{eq:c11exec:acq}~ R \cap sc \subseteq acq \subseteq
R \cup F
\and
\labelb{eq:c11exec:rel}~ W \cap sc \subseteq rel \subseteq W \cup F
\and
\labelb{eq:c11exec:scf}~ F \cap sc \subseteq acq \cap rel
\and
\labelb{eq:c11exec:anal}~R-A \subseteq nal \subseteq E-A
\end{mathpar}
\par\vspace*{-2mm}
\caption{C11 executions, $\ExecC$ (extending $\Exec$)}
\label{fig:c11_exec}
\end{figure}

\begin{definition}[C11 executions] C11 executions ($\ExecC$) are
structures that inherit all the fields and well-formedness conditions
from basic executions, and add those listed in
Fig.~\ref{fig:c11_exec}. The new fields originate from the `memory
orders' that are attached to atomic operations in
C11~\cite[§7.17.3]{c11}. Acquire events, release events, SC events,
RMWs, and fences are all atomic~\refb{eq:c11exec:a}. Atomic events
that are neither acquires nor releases correspond to C11's `relaxed'
memory order. Acquire semantics
is given to \emph{all} SC reads and \emph{only} reads and
fences~\refb{eq:c11exec:acq}. Release semantics is given to \emph{all}
SC writes and \emph{only} writes and fences~\refb{eq:c11exec:rel}. SC
fences have both acquire and release semantics~\refb{eq:c11exec:scf}.
Non-atomic reads access only non-atomic locations, and atomic
operations never access non-atomic locations~\refb{eq:c11exec:anal}.
\end{definition}

\begin{figure}[t]
\begin{tabular}{@{}l@{~}l@{~~~~~~~~}l}
$dv$ & $\subseteq \E$ & events that have whole-device scope \\
$swg$ & $\subseteq \E^2$ & same workgroup \\
\end{tabular}
%\par\vspace*{0.5mm}\rule{\linewidth}{0.4pt}
%\par\vspace*{-5mm}
\begin{mathpar}
\labelb{eq:openclexec:swg}~ 
sthd \subseteq swg
\and
\labelb{eq:openclexec:swgequiv}~ 
\equivalence(swg, E)
\and
\labelb{eq:openclexec:dv}~
dv \subseteq A
\end{mathpar}
\par\vspace*{-2mm}
\caption{OpenCL executions, $\ExecOpenCL$ (extending $\ExecC$)}
\label{fig:opencl_exec}
\end{figure}

\begin{definition}[OpenCL executions] 
%
OpenCL~\cite{opencl20} extends C11 with hierarchical models of
execution and memory that reflect the GPU architectures it was
primarily designed to target. Accordingly, OpenCL executions
($\ExecOpenCL$, Fig.~\ref{fig:opencl_exec}) extend C11 executions
first by partitioning threads into one or more \emph{workgroups} via
the $swg$
equivalence~\refb{eq:openclexec:swg}~\refb{eq:openclexec:swgequiv},
and second by allowing some atomic operations to be tagged as `device
scope'~\refb{eq:openclexec:dv}, which ensures they are visible
throughout the device. Other atomics (i.e., with only `workgroup
scope') can efficiently synchronise threads within a workgroup but
are unsuitable for inter-workgroup synchronisation. We do not
consider OpenCL's local memory in this work, and we restrict our
attention to the single-device case.
%
\end{definition}

\begin{figure}[t]
\begin{tabular}{@{}l@{~}l@{~~~~~~~~}l}
$dv$ & $\subseteq \E$ & events that have whole-device scope \\
$swg$ & $\subseteq \E^2$ & same workgroup (`co-operative thread array') \\
\end{tabular}
%\par\vspace*{0.5mm}\rule{\linewidth}{0.4pt}
\begin{mathpar}
\labelb{eq:ptxexec:swg}~ 
sthd \subseteq swg
\and
\labelb{eq:ptxexec:swgequiv}~ 
\equivalence(swg, E)
\and
\labelb{eq:ptxexec:dvF}~ 
dv \subseteq F
\and
\labelb{eq:ptxexec:nonal}~ 
nal = \emptyset
\and
\labelb{eq:ptxexec:totalsb}~ 
sthd - \id \subseteq sb \cup sb^{-1}
\and
\labelb{eq:ptxexec:rmw}~ 
R \ncap W
\end{mathpar}
\par\vspace*{-2mm}
\caption{PTX executions, $\ExecPTX$ (extending $\Exec$)}
\label{fig:ptx_exec}
\end{figure}

\begin{definition}[PTX executions] 
%
Like OpenCL executions, PTX executions ($\ExecPTX$,
Fig.~\ref{fig:ptx_exec}) gather threads into
groups~\refb{eq:ptxexec:swg}~\refb{eq:ptxexec:swgequiv}. Some PTX
fences ("membar.gl") have whole-device scope~\refb{eq:ptxexec:dvF},
and others ("membar.cta") have only workgroup scope. There are no
non-atomic locations~\refb{eq:ptxexec:nonal}, sequencing is total
within each thread~\refb{eq:ptxexec:totalsb}, and we do not consider
RMWs~\refb{eq:ptxexec:rmw}.
%
\end{definition}

\begin{remark} When drawing C11 executions, we identify the sets $A$,
$acq$, $rel$ and $sc$ by tagging events in $E-A$ with $\na$, those in
$A - acq - rel$ with $\morlx$, those in $acq - rel - sc$ with
$\moacq$, those in $rel - acq - sc$ with $\morel$, those in
$acq \cap rel - sc$ with $\moar$, and those in $sc$ with $\mosc$. In
OpenCL or PTX executions, we tag events in $A - dv$ with $\mswg$, and
those in $dv$ with $\msdv$. \end{remark}

\subsection{Consistent, Race-Free, and Allowed Executions}
\label{sec:c11_consistency}

Each MCM $M$ defines sets $consistent_M$ and $racefree_M$ of
executions. (For architecture-level MCMs, which do not define races,
the latter contains all executions.) The sets work together to define
the executions allowed under $M$, as follows.

\begin{definition}[Allowed executions] 
\label{def:allowed_execs}
The executions of a program $P$
that are allowed under an MCM $M$ are
%
\[
\sem{P}_M \eqdef \begin{cases} 
\mathrlap{\cand{P} \cap consistent_M}\quad\quad\quad\quad \\ & \text{if}~\cand{P}\cap
consistent_M \subseteq racefree_M \\
\top & \text{otherwise}
\end{cases}
\]
%
where $\top$ stands for an appropriate universal set. Here, $\cand{P}$
is the set of $P$'s \emph{candidate executions}. These can be thought
of as the executions allowed under an MCM that imposes no constraints,
and are discussed separately (see
Def.~\ref{def:candidate_executions}). The equation above says that the
allowed executions of $P$ are its consistent candidates, unless a
consistent candidate is racy, in which case \emph{any} execution is
allowed. (This is sometimes called `catch-fire'
semantics~\cite{boehm+08}.) \end{definition}

\newcommand\dashboxed[1]{\begin{tikzpicture}[baseline=(a.base)]
\node[anchor=base, draw, dashed, inner sep=1mm, rounded corners](a)
{$#1$};
\end{tikzpicture}}

\begin{figure}[t]
\begin{myFrame}{$consistent_{\mm{C11}}$}
\begin{mathpar}
\labelb{eq:c11c:uniproc}~
\acyclic((hb \cap \sloc) \cup rf \cup co \cup fr)
\and
\labelb{eq:c11c:narf}~ 
rf \join \stor{nal} \subseteq \imm(\stor{W} \join (hb \cap \sloc))
\and
\labelb{eq:c11c:ssimp}~
\acyclic((Fsb^? \join (co \cup fr \cup hb) \join
    sbF^?) \cap sc^2 \dashboxed{{}\cap incl})
\end{mathpar}
\end{myFrame}
\vspace*{-2mm}
\begin{myFrame}{$racefree_{\mm{C11}}$}
\begin{mathpar}
\labelb{eq:c11r:dr}~
cnf - A^2 - \sthd \subseteq hb \cup hb^{-1}
\and
\labelb{eq:c11r:ur}~ 
cnf \cap \sthd \subseteq sb \cup sb^{-1}
\and
\dashboxed{\labelb{eq:c11r:hr}~ 
cnf - incl \subseteq hb \cup hb^{-1}}
\end{mathpar}
\end{myFrame}
\begin{mathpar}
Fsb \eqdef \stor{F}\join sb
\and
sbF \eqdef sb \join \stor{F}
\and
rs \eqdef (sb \cap \sloc)^* \join rf^* 
\and
sw \eqdef \stor{rel} \join Fsb^? \join \stor{W\cap A} \join rs
\join rf \join \stor{R\cap A} \join sbF^? \join \stor{acq}
\and
\dashboxed{incl \eqdef dv^2 \cup swg}
\and
hb \eqdef ((sw\, \dashboxed{{}\cap incl} - \sthd) \cup sb)^+
\and
cnf \eqdef ((W \times W) \cup (R \times W) \cup (W \times R)) \cap \sloc - \id
\end{mathpar}
\caption{The C11 \protect\dashboxed{\mbox{and OpenCL}} MCMs}
\label{fig:c11_predicates}
\end{figure}

The consistency and race-freedom axioms for C11 and OpenCL (minus
`consume' atomics) are defined in Figure~\ref{fig:c11_predicates} and
explained below. We have included some recently-proposed
simplifications. In particular, the simpler release sequence (proposed
by Vafeiadis et al.~\cite{vafeiadis+15}) makes deadness easier to
define (\S\ref{sec:safety}), and omitting the total order `$S$' over
SC events (as proposed by Batty et al.~\cite{batty+16}) avoids having
to iterate over all total orders when showing an execution to be
inconsistent.

\emph{Happens before} ($hb$) edges are created by sequencing and by a
release \emph{synchronising with} ($sw$) an acquire in a different
thread. Synchronisation begins with a release write (or a release
fence that precedes an atomic write) and ends with an acquire read (or
an acquire fence that follows an atomic read) and the read observes
either that write or a member of the write's \emph{release sequence}
($rs$). An event's release sequence comprises all the writes to the
same location that are sequenced after the event, as well as the RMWs
(which may be in another thread) that can be reached from one of those
writes via a chain of $rf$ edges~\cite[\S4.3]{vafeiadis+15}. In
OpenCL, synchronisation only occurs between events with
\emph{inclusive scopes} ($incl$), which means that if the events are
in different workgroups they must both be annotated with `device' scope.
Happens-before edges between events accessing the same location,
together with $rf$, $co$, and $fr$ edges, must not form
cycles~\refb{eq:c11c:uniproc}~\cite[\S5.3]{vafeiadis+15}. A read of a
non-atomic location must observe a write that is still \emph{visible}
($vis$)~\refb{eq:c11c:narf}. The \emph{sequential consistency} (SC)
axiom~\refb{eq:c11c:ssimp} requires, essentially, that the $co$, $hb$
and $fr$ edges between $sc$ events do not form
cycles~\cite[\S3.2]{batty+16} An execution has a \emph{data race}
unless every pair of conflicting ($cnf$) events in different threads,
not both atomic, is in $hb$~\refb{eq:c11r:dr}. It has an
\emph{unsequenced race} unless every pair of conflicting events in the
same thread is in $sb$~\refb{eq:c11r:ur}. An OpenCL execution has a
\emph{heterogeneous race}~\cite{hower+14} unless every pair of
conflicting events with non-inclusive scopes is in
$hb$~\refb{eq:c11r:hr}.

% \begin{remark}
% \label{rem:S}
% The SC axiom above is from a revised version of the C11 MCM
% suggested by Batty et al.~\cite{batty+16}. In the original MCM, the
% consistency of an execution depends on the existence of a suitable
% total order, say $S$, over SC events. We prefer the revised version in
% this work because when we later search for an \emph{inconsistent} C11
% execution, we need not iterate over all $S$ orders.
% \end{remark}

% \begin{remark}
% \label{rem:c11_initial_writes}
% We have made a few departures from Batty et al.'s presentation of
% these axioms~\cite{batty+16}, as necessitated by our omission of
% initial writes (Rem.~\ref{rem:initial_writes}). First, happens-before
% ($hb$) no longer needs to explicitly connect initial writes to other
% events. Second, now that $rf^{-1}$ is no longer a total function, the
% coherence axiom can no longer be phrased as Batty et al. had it
% ($\irreflexive((rf^{-1})^?\join co \join rf^? \join hb)$), and must
% instead be stated using
% from-read~\refb{eq:c11c:coh}.\footnote{Somewhat more obscurely: the
% `S4' axiom in the original C11 MCM~\cite{batty+16} must also be
% updated to account separately for initial reads. That is, we require
% not just
% $\irreflexive((S - (co\join S)) \join rf^{-1} \join (hb \cap \sloc)
% \join [W])$,
% but now
% $\irreflexive((S - (co\join S)) \join (\stor{R} - (rf^{-1} \join rf))
% \join fr)$ as well.}
% \end{remark}


%%% Local Variables:
%%% mode: latex
%%% TeX-master: "paper"
%%% End:
